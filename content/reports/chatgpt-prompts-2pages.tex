\documentclass[a4paper,9pt]{article}
\usepackage{fontspec}
\setmainfont{Helvetica Neue}[
  BoldFont=Helvetica Neue Bold,
  ItalicFont=Helvetica Neue Italic
]
\usepackage[top=18mm, bottom=15mm, left=18mm, right=18mm]{geometry}
\usepackage{parskip}
\setlength{\parskip}{3pt}
\setlength{\parindent}{0pt}
\usepackage{xcolor}
\definecolor{primary}{HTML}{2563EB}
\definecolor{darkbg}{HTML}{0A0F1E}
\definecolor{bodytext}{HTML}{374151}
\definecolor{heading}{HTML}{0A0F1E}
\definecolor{subtitle}{HTML}{64748B}
\definecolor{lightgray}{HTML}{F8F9FA}
\definecolor{gold}{HTML}{C8AA50}
\definecolor{lightblue}{HTML}{EFF6FF}
\definecolor{lightgreen}{HTML}{F0FDF4}
\definecolor{darkgreen}{HTML}{15803D}
\definecolor{border}{HTML}{E5E7EB}
\usepackage{titlesec}
\titleformat{\section}{\fontsize{12}{14}\selectfont\bfseries\color{heading}}{}{0em}{}[\vspace{-3pt}]
\titleformat{\subsection}{\fontsize{9}{11}\selectfont\bfseries\color{primary}}{}{0em}{}
\titlespacing*{\section}{0pt}{8pt}{3pt}
\titlespacing*{\subsection}{0pt}{6pt}{2pt}
\usepackage{tabularx}
\usepackage{booktabs}
\usepackage{colortbl}
\usepackage{enumitem}
\setlist[itemize]{leftmargin=1em, itemsep=0pt, parsep=0pt, topsep=1pt, label={\color{primary}\textbullet}}
\setlist[enumerate]{leftmargin=1.4em, itemsep=0pt, parsep=0pt, topsep=1pt}
\usepackage{hyperref}
\hypersetup{colorlinks=true, linkcolor=primary, urlcolor=primary}
\usepackage{tikz}
\usetikzlibrary{calc}
\usepackage{multicol}
\usepackage{microtype}
\usepackage{tcolorbox}
\tcbuselibrary{skins}

\newtcolorbox{promptbox}{
  colback=lightgray, colframe=border,
  leftrule=3pt, rightrule=0pt, toprule=0pt, bottomrule=0pt,
  arc=0pt, boxsep=2pt, left=8pt, right=6pt, top=4pt, bottom=4pt,
  fontupper=\fontsize{7.5}{10}\selectfont\ttfamily\color{darkbg}
}

\newtcolorbox{tipbox}{
  colback=lightgreen, colframe=darkgreen,
  leftrule=3pt, rightrule=0pt, toprule=0pt, bottomrule=0pt,
  arc=0pt, boxsep=2pt, left=8pt, right=6pt, top=4pt, bottom=4pt,
  fontupper=\fontsize{8}{10.5}\selectfont\color{darkgreen}
}

\pagestyle{empty}
\color{bodytext}

\begin{document}

% ============================================
% PAGE 1: PROMOTING ON CHATGPT
% ============================================

\begin{tikzpicture}[remember picture, overlay]
  \fill[darkbg] ($(current page.north west)$) rectangle ($(current page.north east)+(0,-2cm)$);
  \node[anchor=west] at ($(current page.north west)+(1.8cm,-1cm)$) {
    {\fontsize{14}{16}\selectfont\bfseries\color{white}How to Get Your Brand Recommended by ChatGPT}
    \hfill
  };
  \node[anchor=east] at ($(current page.north east)+(-1.8cm,-1cm)$) {
    {\fontsize{7}{9}\selectfont\color{gold}Ainary Ventures · 2026}
  };
\end{tikzpicture}

\vspace{12mm}

{\fontsize{8.5}{11}\selectfont\color{subtitle}1 billion+ queries/day flow through ChatGPT. When someone asks ``best AI consultant for manufacturing,'' will it recommend you? Here's how to make that happen.}

\vspace{4pt}

\begin{multicols}{2}
\fontsize{8}{10.5}\selectfont

\section{The New Game: GEO}

\textbf{Generative Engine Optimization (GEO)} is SEO for AI. Instead of ranking in Google's blue links, you're optimizing to be \textbf{cited in AI-generated answers}.

\textbf{Key difference:} ChatGPT doesn't rank pages. It \textbf{synthesizes} answers from sources it trusts. You need to be one of those sources.

\subsection{Where ChatGPT Gets Its Data}

{\fontsize{7.5}{10}\selectfont
\begin{tabularx}{\linewidth}{@{}X r@{}}
\toprule
\rowcolor{lightgray} \textbf{Source} & \textbf{ChatGPT Citations} \\
\midrule
Wikipedia & 47.9\% \\
Reddit & 11.3\% \\
News outlets & 8.2\% \\
Niche authority sites & 15.4\% \\
Other & 17.2\% \\
\bottomrule
\end{tabularx}
}

{\fontsize{7}{9}\selectfont\color{subtitle}Source: Profound Study, June 2025 (30M+ citations analyzed)}

\subsection{7 Tactics That Work}

\begin{enumerate}
  \item \textbf{Structured Q\&A content} — Write FAQ pages that directly answer questions ChatGPT users ask. Use ``What is...'', ``How to...'', ``Best...'' headers.
  \item \textbf{Schema markup (JSON-LD)} — Add structured data: Person, Organization, FAQ, Article, HowTo. This is how AI ``reads'' your site.
  \item \textbf{Wikipedia presence} — Create/update your Wikipedia article (47.9\% of ChatGPT citations!). Or get mentioned on relevant Wiki pages.
  \item \textbf{Authoritative backlinks} — ChatGPT trusts sites that other trusted sites link to. Guest posts on industry publications.
  \item \textbf{Long-form depth content} — 2,000+ word guides on your expertise. AI prefers comprehensive sources it can quote from.
  \item \textbf{Consistent entity signals} — Same name, title, company across LinkedIn, Substack, website, Crunchbase, Twitter. AI cross-references these.
  \item \textbf{Reddit \& community presence} — Answer questions on Reddit, Stack Overflow, Quora with genuine expertise. Perplexity uses Reddit for 46.7\% of citations.
\end{enumerate}

\subsection{Quick Wins (This Week)}

\begin{tipbox}
\textbf{1.} Add JSON-LD structured data to your website (Person + Organization schema)\\
\textbf{2.} Write 3 FAQ-style blog posts answering questions in your domain\\
\textbf{3.} Update LinkedIn with exact same bio/credentials as website\\
\textbf{4.} Answer 5 relevant Reddit/Quora questions with real expertise\\
\textbf{5.} Ask ChatGPT about yourself --- see what it says, then fix gaps
\end{tipbox}

\subsection{What NOT to Do}

\begin{itemize}
  \item \textbf{Don't} stuff keywords --- AI detects this and ignores it
  \item \textbf{Don't} use AI-generated filler content --- AI recognizes its own output
  \item \textbf{Don't} ignore traditional SEO --- Google ranking still feeds ChatGPT
  \item \textbf{Don't} neglect freshness --- update content with current stats/dates
\end{itemize}

\subsection{The Ainary Advantage}

We already have: JSON-LD schema ✅, Substack ✅, LinkedIn ✅, Twitter ✅. Next: Wikipedia presence, Reddit strategy, FAQ content series.

\end{multicols}

\vspace{2pt}
{\color{border}\rule{\linewidth}{0.3pt}}
\begin{center}
{\fontsize{7}{9}\selectfont\color{subtitle}Florian Ziesche · \href{https://ainaryventures.com}{ainaryventures.com} · \href{mailto:f.ziesche.us@gmail.com}{f.ziesche.us@gmail.com}}
\end{center}

% ============================================
% PAGE 2: BEST PROMPTS FOR RESEARCH & LEGAL
% ============================================
\clearpage

\begin{tikzpicture}[remember picture, overlay]
  \fill[darkbg] ($(current page.north west)$) rectangle ($(current page.north east)+(0,-2cm)$);
  \node[anchor=west] at ($(current page.north west)+(1.8cm,-1cm)$) {
    {\fontsize{14}{16}\selectfont\bfseries\color{white}Best Prompts for Research \& Legal Analysis}
  };
  \node[anchor=east] at ($(current page.north east)+(-1.8cm,-1cm)$) {
    {\fontsize{7}{9}\selectfont\color{gold}Ainary Ventures · 2026}
  };
\end{tikzpicture}

\vspace{12mm}

{\fontsize{8.5}{11}\selectfont\color{subtitle}Production-tested prompts for deep research, competitive analysis, legal review, and contract analysis. Use with Claude, GPT-4, or Gemini.}

\vspace{4pt}

\begin{multicols}{2}
\fontsize{8}{10.5}\selectfont

\section{Deep Research Prompts}

\subsection{Market \& Competitive Analysis}

\begin{promptbox}
You are a senior strategy analyst. Research [INDUSTRY/COMPANY]. Deliver:\\
1. Market size \& growth (cite sources)\\
2. Top 5 competitors with positioning\\
3. Key trends (last 12 months)\\
4. Unmet needs / white spaces\\
5. 3 contrarian insights others miss\\
Format: Executive brief, bullet points, sources cited.
\end{promptbox}

\subsection{Deep Dive on a Topic}

\begin{promptbox}
Act as a research scientist. Analyze [TOPIC] comprehensively:\\
- State of the art (what's known)\\
- Key debates (where experts disagree)\\
- Recent breakthroughs (last 6 months)\\
- Open questions (what's unsolved)\\
- Implications for [YOUR DOMAIN]\\
Cite specific papers, studies, or data. Flag uncertainty levels.
\end{promptbox}

\subsection{Due Diligence (Startup/Fund)}

\begin{promptbox}
Conduct due diligence on [COMPANY]:\\
- Business model \& unit economics\\
- Team background \& track record\\
- Market timing: why now?\\
- Technical moat assessment\\
- Top 3 risks (be brutally honest)\\
- Comparable exits in this space\\
Write as if advising a VC partner making a \$5M decision.
\end{promptbox}

\subsection{Red Team / Devil's Advocate}

\begin{promptbox}
I'm about to [DECISION/STRATEGY]. Your job: destroy this idea.\\
Find every weakness, every way it could fail, every assumption\\
that might be wrong. Be specific. Use data where possible.\\
Then: which of your objections are fatal vs. manageable?\\
End with: "If you still proceed, mitigate these 3 things."
\end{promptbox}

\subsection{Trend Synthesis}

\begin{promptbox}
Synthesize these sources: [PASTE 3-5 URLS OR SUMMARIES]\\
- What do they agree on?\\
- Where do they contradict?\\
- What's the signal vs. noise?\\
- What are they all missing?\\
Write a 500-word brief I can share with my team.
\end{promptbox}

\section{Legal Analysis Prompts}

\subsection{Contract Review}

\begin{promptbox}
Review this contract as an experienced corporate attorney.\\
Focus on:\\
1. Liability exposure (uncapped? indemnification?)\\
2. Termination rights (who can exit, when, how?)\\
3. IP ownership (who owns what's created?)\\
4. Non-compete / non-solicit scope\\
5. Payment terms \& penalties\\
6. Governing law \& dispute resolution\\
Flag: RED (high risk), YELLOW (negotiate), GREEN (standard).\\
End with: Top 3 clauses to negotiate before signing.
\end{promptbox}

\subsection{Regulatory Compliance Check}

\begin{promptbox}
Analyze whether [PRODUCT/SERVICE] complies with [REGULATION].\\
- Which specific articles/sections apply?\\
- Where are we compliant? Where are gaps?\\
- What documentation is required?\\
- What are the penalties for non-compliance?\\
- Recommended remediation steps (prioritized)\\
Note: This is preliminary analysis, not legal advice.
\end{promptbox}

\subsection{Legal Risk Assessment}

\begin{promptbox}
We're planning to [ACTION]. Assess legal risks:\\
- Jurisdictions affected (US, EU, other?)\\
- Applicable regulations (GDPR, SOX, industry-specific)\\
- Precedent cases (similar situations, outcomes)\\
- Risk level: Low / Medium / High / Critical\\
- Recommended: proceed / proceed with mitigation / don't proceed\\
Write as a memo to the CEO. Be direct.
\end{promptbox}

\subsection{NDA / Agreement Drafting}

\begin{promptbox}
Draft a [TYPE] agreement between [PARTY A] and [PARTY B].\\
Context: [DESCRIBE RELATIONSHIP \& PURPOSE]\\
Include: standard protections for [MY SIDE].\\
Tone: Professional but not adversarial.\\
Flag any clauses where I should consult an attorney\\
before finalizing.
\end{promptbox}

\vspace{4pt}

\begin{tipbox}
\textbf{Pro Tips:}\\
• Always add: ``Cite sources. Flag uncertainty.''\\
• Chain prompts: Research → Analyze → Red Team → Decide\\
• For legal: Always add ``This is not legal advice'' disclaimer\\
• Use Claude for nuance, GPT-4 for breadth, Gemini for data\\
• Save your best prompts as templates --- they compound
\end{tipbox}

\end{multicols}

\vspace{2pt}
{\color{border}\rule{\linewidth}{0.3pt}}
\begin{center}
{\fontsize{7}{9}\selectfont\color{subtitle}Florian Ziesche · \href{https://ainaryventures.com}{ainaryventures.com} · \href{mailto:f.ziesche.us@gmail.com}{f.ziesche.us@gmail.com}}
\end{center}

\end{document}
