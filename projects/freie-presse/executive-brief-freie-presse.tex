\documentclass[a4paper,11pt]{article}

% Pakete
\usepackage{fontspec}
\usepackage[top=30mm, bottom=35mm, left=28mm, right=28mm]{geometry}
\usepackage{xcolor}
\usepackage{fancyhdr}
\usepackage{titlesec}
\usepackage{tabularx}
\usepackage{booktabs}
\usepackage{enumitem}
\usepackage{tikz}
\usepackage{tcolorbox}
\usepackage{hyperref}
\usepackage{microtype}
\usepackage{needspace}
\usepackage{parskip}
\usepackage{polyglossia}

% Sprache
\setdefaultlanguage{german}

% Fonts
\setmainfont{Helvetica Neue}

% Farben (reduziert, professionell)
\definecolor{heading}{HTML}{0A0F1E}
\definecolor{bodytext}{HTML}{374151}
\definecolor{subtitle}{HTML}{64748B}
\definecolor{gold}{HTML}{c8aa50}
\definecolor{lightgold}{HTML}{e8d89f}
\definecolor{lightgray}{HTML}{F8F9FA}
\definecolor{border}{HTML}{E5E7EB}

% Typografie
\tolerance=2000
\emergencystretch=15pt
\widowpenalty=10000
\clubpenalty=10000
\setlength{\parskip}{8pt}
\setlength{\parindent}{0pt}

% Titel-Formate
\titleformat{\section}{\fontsize{20}{24}\bfseries\color{heading}}{}{0em}{}[\vspace{2pt}\rule{\linewidth}{0.5pt}]
\titlespacing*{\section}{0pt}{16pt}{10pt}

\titleformat{\subsection}{\fontsize{14}{18}\bfseries\color{heading}}{}{0em}{}
\titlespacing*{\subsection}{0pt}{12pt}{6pt}

% Header/Footer
\pagestyle{fancy}
\fancyhf{}
\renewcommand{\headrulewidth}{0pt}
\renewcommand{\footrulewidth}{0.4pt}
\fancyfoot[L]{\footnotesize\color{subtitle}Florian Ziesche \textperiodcentered{} KI-Strategie für Freie Presse}
\fancyfoot[R]{\footnotesize\color{subtitle}\thepage}

% Hyperlinks
\hypersetup{
    colorlinks=true,
    linkcolor=bodytext,
    urlcolor=gold,
    pdftitle={KI-Transformation: Freie Presse},
    pdfauthor={Florian Ziesche}
}

% Custom Commands
\newcommand{\statbox}[2]{%
    \begin{tcolorbox}[colback=lightgray,colframe=border,boxrule=0.5pt,arc=3pt,left=8pt,right=8pt,top=8pt,bottom=8pt]
        \centering
        {\fontsize{24}{28}\selectfont\bfseries\color{gold}#1}\\[4pt]
        {\footnotesize\color{subtitle}\textbf{#2}}
    \end{tcolorbox}
}

\newcommand{\keypoint}[1]{%
    \begin{tcolorbox}[colback=lightgold!30,colframe=gold,boxrule=1pt,arc=4pt,left=10pt,right=10pt,top=8pt,bottom=8pt]
        {\color{heading}#1}
    \end{tcolorbox}
}

\begin{document}
\thispagestyle{empty}

% COVER
\begin{center}
{\fontsize{32}{38}\selectfont\bfseries\color{heading}
KI-Transformation\\[4pt]
für die Freie Presse}

\vspace{12pt}
{\Large\color{subtitle}Vom Effizienzproblem zur Marktführerschaft}

\vspace{20pt}
\rule{0.5\linewidth}{0.5pt}

\vspace{20pt}
\begin{minipage}[t]{0.3\linewidth}\centering
    \statbox{2-3x}{Content-Output}
\end{minipage}\hfill
\begin{minipage}[t]{0.3\linewidth}\centering
    \statbox{€1-3M}{Zusatz-Revenue}
\end{minipage}\hfill
\begin{minipage}[t]{0.3\linewidth}\centering
    \statbox{< 6 Mo.}{Payback}
\end{minipage}

\vspace{24pt}
{\color{subtitle}
Erstellt für Dr. Daniel Daum, Geschäftsführer\\
Freie Presse Mediengruppe
}

\vspace{30pt}
{\footnotesize\color{subtitle}
\textbf{Florian Ziesche}\\
KI-Strategie \& Implementierung\\[6pt]
+49 151 230 39 208\\
florian.ziesche@hotmail.com\\[12pt]
\textit{55 Seiten Strategie, 9 Demos mit echten Daten, 15.500 Wörter Content.\\
In unter 45 Minuten. Mit KI. Das ist der Proof of Concept.}
}
\end{center}

\clearpage

% SEITE 1: Problem & Chance
\section*{Das Problem ist nicht Relevanz --- es ist Effizienz}

Regionalzeitungen stehen unter Druck. Die Zahlen sind bekannt: Sinkende Auflagen, steigende Kosten, Fachkräftemangel. Aber das eigentliche Problem ist nicht, dass niemand mehr liest. Das Problem ist, dass die Produktion von qualitativ hochwertigem Journalismus zu teuer geworden ist.

\textbf{Die Freie Presse hat 750 Mitarbeiter, 19 Redaktionen und einen geschätzten Umsatz von €80-120 Millionen.} Das bedeutet: Jeder Effizienzgewinn hat eine massive Hebelwirkung. Jede Stunde, die ein Reporter spart, multipliziert sich durch die gesamte Organisation.

Die gute Nachricht: Das ist ein \textit{lösbares} Problem.

\subsection*{Der Markt kippt --- jetzt}

Die Technologie ist reif. Nicht in 2 Jahren. Jetzt. KI-gestützte Recherche, Faktencheck und Content-Optimierung funktionieren heute --- produktionsreif, DSGVO-konform, journalistisch verantwortbar.

\keypoint{\textbf{Die Frage ist nicht mehr OB, sondern WER ZUERST.} Wer in 12 Monaten nicht KI-native arbeitet, wird von denen überholt, die es tun. Die Sächsische Zeitung wird folgen. Der MDR wird folgen. Die Frage ist: Wer ist der First Mover?}

\subsection*{Warum die Freie Presse prädestiniert ist}

\begin{itemize}[itemsep=4pt,parsep=0pt,leftmargin=18pt]
    \item \textbf{19 Redaktionen} -- Effizienzgewinn multipliziert sich durch die gesamte Struktur
    \item \textbf{Etablierte Marke} -- Leser vertrauen der Freie Presse, keine Kaltakquise nötig
    \item \textbf{Digitale Erfahrung} -- Daniel Daum kennt digitale Transformation aus erster Hand (Rheinische Post)
    \item \textbf{Regionale Verwurzelung} -- lokale Themen sind KI-resistent (keine Konkurrenz durch überregionale KI-Newsseiten)
\end{itemize}

\textbf{Das ist keine \grqq{}Digital Transformation Journey\grqq{}. Das ist ein konkreter, umsetzbarer Effizienzgewinn mit messbarem ROI.}

\clearpage

% SEITE 2: Prozess Heute vs. Morgen
\section*{Wie ändert sich der Alltag konkret?}

Das Ziel ist nicht, Journalisten zu ersetzen. Das Ziel ist, jeden Reporter mit einem unsichtbaren Recherche-Team auszustatten.

\subsection*{Für einen Reporter: Recherche \& Schreiben}

\small
\begin{tabularx}{\linewidth}{@{}l X X@{}}
\toprule
\textbf{Schritt} & \textbf{HEUTE} & \textbf{MORGEN (mit KI)} \\
\midrule
Thema finden & Bauchgefühl, Tipps, Redaktionskonferenz & KI-Radar scannt 100+ Quellen täglich (Lokal, Regional, Soziale Medien), priorisiert Story-Leads nach Relevanz \\[6pt]
Recherche & 2-3h manuell googlen, Archiv durchsuchen, Quellen anrufen & \textbf{10 Min:} KI liefert Briefing mit Quellen, Fakten, Zitaten, möglichen Story-Angles \\[6pt]
Schreiben & 3-4h, von Null anfangen & \textbf{2h:} Reporter schreibt (Qualität bleibt beim Menschen), KI unterstützt bei Struktur, Faktencheck, Stil \\[6pt]
Headline & 1 Versuch, Bauchgefühl & 5 datengetriebene Varianten mit SEO-Score, A/B-Test-Empfehlung \\[6pt]
Kontrolle & Redakteur liest, korrigiert & Redakteur liest (finales Wort beim Menschen) + KI-Faktencheck als Safety Net \\
\bottomrule
\end{tabularx}
\normalsize

\vspace{12pt}
\textbf{Zeitersparnis pro Artikel: 2-3 Stunden.} Bei 5 Artikeln pro Woche = 10-15h/Reporter. Bei 100 Reportern = 1.000-1.500h/Woche frei für investigative Recherche, Interviews, Deep Dives.

\subsection*{Für den GF: Strategische Entscheidungen}

\small
\begin{tabularx}{\linewidth}{@{}l X X@{}}
\toprule
\textbf{Aufgabe} & \textbf{HEUTE} & \textbf{MORGEN} \\
\midrule
Redaktionsplanung & Wöchentliche Konferenz, Bauchgefühl & Datengetriebene Trend-Analyse: welche Themen performen regional, wo sind Content-Gaps, was suchen Leser \\[6pt]
Meeting-Prep & Assistenz recherchiert 30-60 Min, Notizen & KI-Briefing in 5 Min: Kontext, Historie, Gesprächsleitfaden \\[6pt]
Leser-Feedback & Manuell lesen, aufwendig & KI kategorisiert, fasst zusammen, identifiziert Trends und Problemfelder \\[6pt]
Revenue Streams & Brainstorming, externe Berater & \textbf{Konkrete Use Cases:} Personalisierte Newsletter (€396K/Jahr), Archiv-Monetarisierung (€327K/Jahr) \\
\bottomrule
\end{tabularx}
\normalsize

\vspace{8pt}
\textbf{Das Ergebnis:} Bessere Entscheidungen, schneller getroffen, mit mehr Datenbasis.

\clearpage

% SEITE 3: Strategischer Wert & ROI
\section*{Strategischer Wert: Warum das mehr ist als \grqq{}Effizienz\grqq{}}

\subsection*{1. Output-Multiplikator}

\begin{itemize}[itemsep=4pt,parsep=0pt,leftmargin=18pt]
    \item \textbf{Gleiche Mannschaft, 2-3x Content-Output} -- ohne neue Stellen, ohne Überstunden
    \item \textbf{19 Redaktionen x Effizienzgewinn} -- massiver Compound-Effekt durch die gesamte Organisation
    \item \textbf{Jeder Reporter wird zum \grqq{}Reporter + Recherche-Team\grqq{}} -- mehr Zeit für das, was Menschen besser können (Interviews, Vor-Ort-Recherche, investigative Deep Dives)
\end{itemize}

\subsection*{2. Neue Revenue Streams (konservativ gerechnet)}

\begin{itemize}[itemsep=4pt,parsep=0pt,leftmargin=18pt]
    \item \textbf{Personalisierte Newsletter} -- Neue Abo-Modelle (Regional-Themen, Branchen-News) -- \textbf{€396.000/Jahr}
    \item \textbf{Archiv-Monetarisierung} -- Premium-Dossiers, KI-kuratierte Themen-Pakete -- \textbf{€327.000/Jahr}
    \item \textbf{SEO-optimierte Headlines} -- Mehr organischer Traffic -- Mehr Digital-Abos -- \textbf{€200-500K/Jahr}
\end{itemize}

\textbf{Gesamt-Potenzial: €1-3 Millionen zusätzlicher Jahresumsatz} (ohne bestehende Abos zu kannibalisieren).

\subsection*{3. Return on Investment}

\keypoint{%
\textbf{Return:} €1-3M zusätzlicher Jahresumsatz (konservativ gerechnet)\\[4pt]
\textbf{Payback:} < 6 Monate nach Go-Live\\[4pt]
\textbf{Effizienzgewinn:} 2-3x Content-Output bei gleicher Mannschaft
}

\subsection*{Warum JETZT und nicht in 2 Jahren?}

\begin{enumerate}[itemsep=4pt,parsep=0pt,leftmargin=18pt]
    \item \textbf{Die Technologie ist reif.} GPT-4, Claude, lokale DSGVO-konforme Modelle --- funktionieren heute produktionsreif.
    \item \textbf{First Mover Advantage.} Die erste KI-native Regionalzeitung wird zum Benchmark für die Branche.
    \item \textbf{Wettbewerber werden folgen.} Sächsische Zeitung, MDR, andere Regionalverlage --- es ist keine Frage OB, sondern WER ZUERST.
    \item \textbf{Die Kosten sinken nicht von allein.} Ohne Effizienzgewinn wird der Kostendruck die Qualität killen. Mit KI bleibt Raum für investigativen Journalismus.
\end{enumerate}

\subsection*{Warum Florian Ziesche?}

\begin{itemize}[itemsep=4pt,parsep=0pt,leftmargin=18pt]
    \item \textbf{KI-System im täglichen Produktiveinsatz:} Arbeitet jeden Tag mit einem selbst aufgebauten Multi-Agenten-System -- Langzeitgedächtnis über Wochen, automatische Recherche, Kalender-/Email-Integration, spezialisierte Sub-Agenten. Dieses Dokument, 9 funktionsfähige Demos und eine 27-seitige Strategieanalyse entstanden damit in unter 45 Minuten. Kein Prototyp -- Produktivsystem.
    \item \textbf{5 Jahre CEO/CTO} -- 36ZERO Vision (Cloud Computer Vision SaaS), Team auf 15 FTE skaliert, €5.5M+ Fundraising
    \item \textbf{Kennt die Region} -- Schlottwitz/Glashütte, lokale Verwurzelung
    \item \textbf{Kein Berater-Overhead} -- direkte Implementierung, kein PowerPoint, sondern Systeme die laufen
\end{itemize}

\subsection*{Was noch vorliegt}

Parallel zu diesem Executive Brief wurden erstellt:

\begin{itemize}[itemsep=4pt,parsep=0pt,leftmargin=18pt]
    \item \textbf{27-seitige Strategische Analyse} -- Marktdaten, SWOT, 12 KI-Einsatzfelder, Technologie-Assessment, Implementierungs-Roadmap, Revenue-Potenzial
    \item \textbf{Business-Potential Deep Dive} -- 10 spezialisierte KI-Agenten (Market Analyst, Revenue Strategist, Cost Analyst, Technology Architect, Risk Analyst u.a.) haben in 6 Minuten eine Analyse mit 6.800 Wörtern erstellt: 6 quantifizierte Revenue Streams, \textbf{NPV von EUR 18--24M über 5 Jahre}, hierarchische Wissensarchitektur, 3-Jahres-Modell
    \item \textbf{9 getestete, vorzeigbare Demos} -- Recherche-Agent, Leserbrief-Moderation, Headline-Generator, Lokalnachrichten-Radar, Personalisierte Newsletter, Archiv-Dossier, Meeting-Briefing, Trend-Erkennung. Alle mit echten Daten aus der Region Chemnitz/Erzgebirge getestet -- keine Mockups, sondern funktionsfähige Prototypen (15.500+ Wörter Content)
\end{itemize}

\vspace{20pt}

\textbf{Alles in unter 45 Minuten. Alles getestet. Alles im Gespr\"ach live vorzeigbar.}

\vspace{12pt}

30 Minuten Ihrer Zeit. Ich zeige Ihnen live, was heute m\"oglich ist.

\vspace{16pt}

\textbf{Florian Ziesche}\\
+49 151 230 39 208\\
florian.ziesche@hotmail.com

\end{document}
