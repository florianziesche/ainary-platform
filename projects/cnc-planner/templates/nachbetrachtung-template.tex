% ================================================================
% CNC Planer Pro — Nachbetrachtung Template
% VARIABLEN: Suche nach {{VARIABLE}} und ersetze
% ================================================================
\documentclass[10pt,a4paper]{article}
\usepackage[left=20mm,right=20mm,top=20mm,bottom=22mm]{geometry}
\usepackage{fontspec}
\usepackage{xcolor}
\usepackage{booktabs}
\usepackage{tabularx}
\usepackage{colortbl}
\usepackage{enumitem}
\usepackage{fancyhdr}
\usepackage{titlesec}
\usepackage{amssymb}
\usepackage{multirow}
\usepackage{tikz}
\usepackage{tcolorbox}
\tcbuselibrary{skins,breakable}
\usepackage{polyglossia}
\setdefaultlanguage{german}

\setmainfont{Helvetica}
\setmonofont{Menlo}[Scale=0.85]

% ── Ainary CI Farben ──
\definecolor{AinaryGold}{HTML}{C8AA50}
\definecolor{AinaryDark}{HTML}{1F2937}
\definecolor{AinaryRed}{HTML}{DC2626}
\definecolor{AinaryGreen}{HTML}{059669}
\definecolor{AinaryBlue}{HTML}{2563EB}
\definecolor{AinaryOrange}{HTML}{D97706}
\definecolor{SurfaceAlt}{HTML}{F9FAFB}
\definecolor{SurfaceWarn}{HTML}{FFFBEB}
\definecolor{SurfaceGreen}{HTML}{E8F5E8}
\definecolor{SurfaceRed}{HTML}{FDE8EA}
\definecolor{SurfaceBlue}{HTML}{E7F1FF}
\definecolor{BorderLight}{HTML}{E5E7EB}

% ── Typografie ──
\tolerance=2000
\emergencystretch=15pt
\widowpenalty=10000
\clubpenalty=10000

\titleformat{\section}
  {\large\bfseries\color{AinaryDark}}
  {\thesection}{1em}{}[\titlerule]
\titleformat{\subsection}
  {\normalsize\bfseries\color{AinaryDark}}
  {\thesubsection}{1em}{}
\titlespacing{\section}{0pt}{14pt}{6pt}
\titlespacing{\subsection}{0pt}{10pt}{4pt}

% ── Header/Footer ──
\pagestyle{fancy}
\fancyhf{}
\renewcommand{\headrulewidth}{0pt}
\fancyhead[L]{\small\color{AinaryGold}\textbf{CNC Planer Pro} \color{gray}| Nachbetrachtung}
\fancyhead[R]{\small\color{gray}Zeichnung {{ZEICHNUNGSNR}}}
\renewcommand{\footrulewidth}{0.4pt}
\fancyfoot[L]{\scriptsize\color{gray} INTERN — Nachbetrachtung | CNC Planer Pro {{VERSION}}}
\fancyfoot[C]{\scriptsize\color{gray} Erstellt: {{DATUM}}}
\fancyfoot[R]{\scriptsize\color{gray} Seite \thepage}

% ── Boxen ──
\newtcolorbox{insightbox}[2][]{%
  enhanced,breakable,
  colback=#2,colframe=#2,
  boxrule=0pt,left=8pt,right=8pt,top=6pt,bottom=6pt,
  borderline west={3pt}{0pt}{#1},
  fontupper=\small
}

\newtcolorbox{protocolbox}{%
  enhanced,
  colback=SurfaceAlt,colframe=BorderLight,
  boxrule=0.5pt,left=8pt,right=8pt,top=6pt,bottom=6pt,
  fontupper=\small
}

\setlength{\parindent}{0pt}
\setlength{\parskip}{4pt}

\begin{document}

% ================================================================
% TITEL
% ================================================================
\begin{center}
{\color{AinaryGold}\rule{\linewidth}{2pt}}\\[8pt]
{\LARGE\bfseries\color{AinaryDark} Nachbetrachtung}\\[3pt]
{\large\color{gray} Kalkulationsvergleich: CNC Planer Pro vs. {{BETRIEB}} Vorkalkulation}\\[3pt]
{\small\color{AinaryGold} {{BAUTEIL}} — Zeichnung {{ZEICHNUNGSNR}} | {{KUNDE}}}\\[6pt]
{\color{AinaryGold}\rule{\linewidth}{0.5pt}}
\end{center}

\vspace{-2pt}

% ================================================================
% 1. KOPFDATEN
% ================================================================
\section*{1\quad Kopfdaten}

\begin{tabularx}{\textwidth}{@{}lXX@{}}
\toprule
& \textbf{{{BETRIEB}} Vorkalkulation} & \textbf{CNC Planer Pro} \\
\midrule
\textbf{Dokument} & {{BETRIEB_DOK}} & Kalkulationsbericht {{VERSION}} \\
\textbf{System} & {{BETRIEB_SYSTEM}} & CNC Planer Pro (KI) \\
\textbf{Datum} & {{BETRIEB_DATUM}} & {{DATUM}} \\
\textbf{Bearbeiter} & {{BETRIEB_BEARBEITER}} & KI-gestützt (REFA/VDI) \\
\textbf{Kunde} & {{KUNDE}} & {{KUNDE}} \\
\textbf{Stückzahl} & {{STUECKZAHL}} Stück & {{STUECKZAHL}} Stück \\
\textbf{Methodik} & Betriebliche Systemzeiten (AV) & REFA-Richtwerte, VDI 3321 \\
\textbf{Scope} & Gesamtfertigung & Nur CNC-Fräsvorgänge \\
\bottomrule
\end{tabularx}

\smallskip
\begin{insightbox}[AinaryBlue]{SurfaceBlue}
\textbf{Abgrenzung:} Dieser Vergleich betrachtet \textbf{ausschließlich die CNC-Fräsvorgänge}. Vor- und nachgelagerte Prozesse sind nicht im Scope des CNC Planer Pro und werden nicht verglichen. Beide Kalkulationen sind Planwerte — \textbf{Ist-Daten liegen noch nicht vor.}
\end{insightbox}

% ================================================================
% 2. ARBEITSGÄNGE — NUR FRÄSEN
% ================================================================
\section*{2\quad Arbeitsgänge — Nur Fräsvorgänge}

\subsection*{{{BETRIEB}} Vorkalkulation}

{\small Die Vorkalkulation enthält neben den Fräsvorgängen weitere Positionen, die hier \textit{nicht} betrachtet werden.}

\vspace{4pt}

% ── BETRIEB Arbeitsgänge: Zeilen anpassen ──
{\small\renewcommand{\arraystretch}{1.25}
\begin{tabularx}{\textwidth}{@{}r p{3cm} r r r r r@{}}
\toprule
\textbf{Pos} & \textbf{Bezeichnung} & \textbf{Tr [min]} & \textbf{Te [min]} & \textbf{HK Lohn} & \textbf{HK Masch} & \textbf{HK Gesamt} \\
\midrule
\rowcolor{SurfaceBlue}
{{POS1_NR}} & {{POS1_BEZ}} & {{POS1_TR}} & {{POS1_TE}} & {{POS1_LOHN}} & {{POS1_MASCH}} & {{POS1_GES}} \\
\rowcolor{SurfaceBlue}
{{POS2_NR}} & {{POS2_BEZ}} & {{POS2_TR}} & {{POS2_TE}} & {{POS2_LOHN}} & {{POS2_MASCH}} & {{POS2_GES}} \\
% Weitere Positionen nach Bedarf ergänzen
\midrule
& \textbf{Summe Fräsen} & \textbf{{{BETRIEB_TR_SUMME}}} & \textbf{{{BETRIEB_TE_SUMME}}} & \textbf{{{BETRIEB_LOHN_SUMME}}} & \textbf{{{BETRIEB_MASCH_SUMME}}} & \textbf{{{BETRIEB_GES_SUMME}}} \\
\bottomrule
\end{tabularx}}

{\scriptsize\color{gray} Annahme: Werte beziehen sich auf den Gesamtauftrag ({{STUECKZAHL}} Stück).}

\vspace{10pt}

\subsection*{CNC Planer Pro (gleiche Darstellung, $\times${{STUECKZAHL}} Stück)}

% ── CNC Planer Arbeitsgänge: Zeilen anpassen ──
{\small\renewcommand{\arraystretch}{1.25}
\begin{tabularx}{\textwidth}{@{}r p{3cm} r r r r r@{}}
\toprule
\textbf{AG} & \textbf{Bezeichnung} & \textbf{Rüst [min]} & \textbf{Bear. [min]} & \textbf{HK Lohn} & \textbf{HK Masch} & \textbf{HK Gesamt} \\
\midrule
\rowcolor{SurfaceBlue}
{{CNC_AG_NR}} & {{CNC_AG_BEZ}} & {{CNC_RUEST}} & {{CNC_BEAR}} & {{CNC_LOHN}} & {{CNC_MASCH}} & {{CNC_GES}} \\
\midrule
& \textbf{Summe Fräsen ($\times${{STUECKZAHL}})} & \textbf{{{CNC_RUEST}}} & \textbf{{{CNC_BEAR}}} & \textbf{{{CNC_LOHN}}} & \textbf{{{CNC_MASCH}}} & \textbf{{{CNC_GES}}} \\
\bottomrule
\end{tabularx}}

{\scriptsize\color{gray} Bearbeitungszeit: {{CNC_BEAR_STK}}\,min/Stk $\times$ {{STUECKZAHL}} = {{CNC_BEAR}}\,min. Rüstzeit: {{CNC_RUEST}}\,min einmalig. Stundensatz: {{CNC_STUNDENSATZ}}\,€/h.}

% ================================================================
% 3. PREISVERGLEICH
% ================================================================
\section*{3\quad Preisvergleich — Nur Fräsen, pro Stück}

\begin{insightbox}[AinaryOrange]{SurfaceWarn}
\textbf{Annahme:} Die {{BETRIEB}}-Werte beziehen sich auf den \textbf{Gesamtauftrag ({{STUECKZAHL}} Stück)}. Pro-Stück-Werte ergeben sich durch Division. \textit{Noch zu klären: Bestätigung durch {{BETRIEB}}.}
\end{insightbox}

\vspace{4pt}

\renewcommand{\arraystretch}{1.35}
\begin{tabularx}{\textwidth}{@{}Xrrrr@{}}
\toprule
& \multicolumn{2}{c}{\textbf{{{BETRIEB}}}} & \multicolumn{2}{c}{\textbf{CNC Planer Pro}} \\
\cmidrule(lr){2-3}\cmidrule(lr){4-5}
\textbf{Position} & \textbf{Gesamt ({{STUECKZAHL}}\,Stk)} & \textbf{pro Stück} & \textbf{Gesamt ({{STUECKZAHL}}\,Stk)} & \textbf{pro Stück} \\
\midrule
{{VERGLEICH_ZEILE_1}} \\
{{VERGLEICH_ZEILE_2}} \\
\midrule
\textbf{Summe Fräsen} & \textbf{{{BETRIEB_GES_SUMME}}} & \textbf{{{BETRIEB_GES_STK}}} & \textbf{{{CNC_GES}}} & \textbf{{{CNC_GES_STK}}} \\
\midrule
\textbf{Delta} & \multicolumn{4}{c}{CNC Planer = \textbf{{{DELTA_PROZENT}}\% von {{BETRIEB}}} (Faktor {{DELTA_FAKTOR}}$\times$)} \\
\bottomrule
\end{tabularx}

\vspace{8pt}

\subsection*{Fertigungskosten-Vergleich pro Stück}

\renewcommand{\arraystretch}{1.35}
\begin{tabularx}{\textwidth}{@{}Xrrr@{}}
\toprule
\textbf{Position} & \textbf{{{BETRIEB}} (÷{{STUECKZAHL}} Stk)} & \textbf{CNC Planer} & \textbf{Delta} \\
\midrule
Summe Fertigung (Bearbeitung) & — & {{CNC_FERTIGUNG_MIN}}\,min / {{CNC_FERTIGUNG_EUR}}\,€ & — \\
Rüstkosten ({{CNC_RUEST}}\,min ÷ {{STUECKZAHL}}\,Stk) & — & {{CNC_RUEST_STK}}\,min / {{CNC_RUEST_STK_EUR}}\,€ & — \\
\textbf{Fertigungskosten Basis} & — & \textbf{{{CNC_FK_MIN}}\,min / {{CNC_FK_EUR}}\,€} & — \\
\midrule
{{BETRIEB}} Stückpreis (HK ÷ {{STUECKZAHL}}) & \textbf{{{BETRIEB_HK_STK}}\,€} & — & — \\
CNC Planer Basis (REFA) & — & {{CNC_BASIS}}\,€ & — \\
\rowcolor{SurfaceGreen}
\textbf{CNC Planer Empfehlung (ohne Risiko)} & — & \textbf{{{CNC_EMPFEHLUNG}}\,€} & — \\
\rowcolor{SurfaceGreen}
\textbf{CNC Planer mit allen Zuschlägen} & — & \textbf{{{CNC_ZUSCHLAEGE}}\,€} & — \\
\rowcolor{SurfaceBlue}
\textbf{{{BETRIEB}} Fräskosten gesamt ({{STUECKZAHL}}\,Stk)} & \multicolumn{3}{c}{\textbf{{{BETRIEB_GES_SUMME}}}} \\
\bottomrule
\end{tabularx}

\vspace{8pt}

\begin{insightbox}[AinaryGreen]{SurfaceGreen}
\textbf{Ohne Risikobetrachtung} liegt der CNC Planer Pro bei \textbf{{{CNC_EMPFEHLUNG}}\,€/Stk}. \textbf{Mit allen Zuschlägen} bei \textbf{{{CNC_ZUSCHLAEGE}}\,€/Stk}.\\[6pt]
% ── Hier den spezifischen Vergleichskommentar einfügen ──
{{VERGLEICHSKOMMENTAR}}
\end{insightbox}

% ================================================================
% 4. PREISEMPFEHLUNG MIT ZUSCHLÄGEN
% ================================================================
\section*{4\quad Preisempfehlung mit allen Hinweisen}

\subsection*{Angebotsoptionen}

% ── Optionen-Tabelle: Zeilen anpassen ──
\renewcommand{\arraystretch}{1.35}
\begin{tabularx}{\textwidth}{@{}p{4.5cm}rrX@{}}
\toprule
\textbf{Variante} & \textbf{Stückpreis} & \textbf{{{STUECKZAHL}} Stück} & \textbf{Anmerkung} \\
\midrule
Basispreis & {{CNC_BASIS}}\,€ & {{CNC_BASIS_GESAMT}}\,€ & REFA-Richtwerte, ohne Korrekturen \\
\rowcolor{SurfaceGreen}
\textbf{Empfehlung} & \textbf{{{CNC_EMPFEHLUNG}}\,€} & \textbf{{{CNC_EMPFEHLUNG_GESAMT}}\,€} & \textbf{Korrigierte Zeiten + Risikopuffer} \\
Sicherheitsmarge & {{CNC_SICHERHEIT}}\,€ & {{CNC_SICHERHEIT_GESAMT}}\,€ & Erstauftrag + Risikopuffer \\
\bottomrule
\end{tabularx}

\vspace{8pt}

\subsection*{Empfehlung mit allen Zuschlägen eingepreist}

% ── Zuschlagstabelle: Zeilen anpassen ──
\renewcommand{\arraystretch}{1.3}
\begin{tabular}{@{}p{6cm}rr@{}}
\toprule
\textbf{Position} & \textbf{pro Stück} & \textbf{Quelle} \\
\midrule
CNC Planer Empfehlung (Basis) & {{CNC_EMPFEHLUNG}}\,€ & Korrigierte REFA-Zeiten \\
% ── Zuschläge hier einfügen ──
\quad + {{ZUSCHLAG_1_BEZ}} & {{ZUSCHLAG_1_EUR}}\,€ & {{ZUSCHLAG_1_QUELLE}} \\
\quad + {{ZUSCHLAG_2_BEZ}} & {{ZUSCHLAG_2_EUR}}\,€ & {{ZUSCHLAG_2_QUELLE}} \\
\quad + {{ZUSCHLAG_3_BEZ}} & {{ZUSCHLAG_3_EUR}}\,€ & {{ZUSCHLAG_3_QUELLE}} \\
\quad + {{ZUSCHLAG_4_BEZ}} & {{ZUSCHLAG_4_EUR}}\,€ & {{ZUSCHLAG_4_QUELLE}} \\
\midrule
\rowcolor{SurfaceGreen}
\textbf{Empfohlener Angebotspreis} & \textbf{{{CNC_ZUSCHLAEGE}}\,€} & \\
\rowcolor{SurfaceGreen}
\textbf{Auftragswert ({{STUECKZAHL}} Stk, netto)} & \textbf{{{AUFTRAGSWERT_NETTO}}\,€} & \\
\bottomrule
\end{tabular}

% ================================================================
% 5. PREISRELEVANTE HINWEISE
% ================================================================
\section*{5\quad Preisrelevante Hinweise}

% ── Hinweise als insightbox: nach Bedarf anpassen/ergänzen/löschen ──

\begin{insightbox}[AinaryRed]{SurfaceRed}
\textbf{$\blacktriangle$ {{HINWEIS_1_TITEL}}}\\[2pt]
{{HINWEIS_1_TEXT}}\\[2pt]
{\footnotesize {{HINWEIS_1_QUELLE}}}
\end{insightbox}

\begin{insightbox}[AinaryOrange]{SurfaceWarn}
\textbf{$\blacktriangle$ {{HINWEIS_2_TITEL}}}\\[2pt]
{{HINWEIS_2_TEXT}}\\[2pt]
{\footnotesize {{HINWEIS_2_QUELLE}}}
\end{insightbox}

\begin{insightbox}[AinaryGreen]{SurfaceGreen}
\textbf{$\blacktriangle$ {{HINWEIS_3_TITEL}}}\\[2pt]
{{HINWEIS_3_TEXT}}\\[2pt]
{\footnotesize {{HINWEIS_3_QUELLE}}}
\end{insightbox}

\begin{insightbox}[AinaryBlue]{SurfaceBlue}
\textbf{$\circlearrowleft$ {{HINWEIS_4_TITEL}}}\\[2pt]
{{HINWEIS_4_TEXT}}\\[2pt]
{\footnotesize {{HINWEIS_4_QUELLE}}}
\end{insightbox}

% ================================================================
% 6. IST-ZEITEN
% ================================================================
\section*{6\quad Ist-Zeiten (nach Fertigung eintragen)}

% ── Arbeitsgänge: Zeilen anpassen ──
{\small\renewcommand{\arraystretch}{1.25}
\begin{tabularx}{\textwidth}{@{}Xrrrr@{}}
\toprule
\textbf{Arbeitsgang} & \textbf{CNC Planer} & \textbf{{{BETRIEB}} (AV)} & \textbf{IST Teil 1} & \textbf{IST Teil 2} \\
\midrule
{{IST_ZEILE_1}} \\
{{IST_ZEILE_2}} \\
{{IST_ZEILE_3}} \\
% Weitere Zeilen nach Bedarf
\midrule
\textbf{Summe} & \textbf{{{IST_SUMME}}\,min} & — & \textcolor{gray}{\textit{\_\_\_\_\_\_}} & \textcolor{gray}{\textit{\_\_\_\_\_\_}} \\
\bottomrule
\end{tabularx}}

% ================================================================
% 7. OFFENE FRAGEN
% ================================================================
\section*{7\quad Offene Fragen}

\begin{enumerate}[leftmargin=*,itemsep=4pt]
  \item \textbf{Mengeneinheit:} Sind die HK-Werte für 1 Stück oder den Gesamtauftrag?
  \item \textbf{Maschinenaufteilung:} Welche Bearbeitungen laufen auf welcher Maschine?
  % ── Weitere Fragen nach Bedarf ──
  \item {{FRAGE_3}}
  \item {{FRAGE_4}}
  \item \textbf{Ist-Zeiten:} Protokollierung der ersten 2 Teile AG für AG.
\end{enumerate}

% ================================================================
% 8. GESAMTVERGLEICH
% ================================================================
\clearpage
\section*{8\quad Gesamtvergleich — Gegenüberstellung}

\subsection*{8.1\quad Zeiten und Preise}

\renewcommand{\arraystretch}{1.4}
\begin{tabularx}{\textwidth}{@{}Xrr@{}}
\toprule
\textbf{Position} & \textbf{{{BETRIEB}} (÷{{STUECKZAHL}} Stk)} & \textbf{CNC Planer Pro} \\
\midrule
\multicolumn{3}{@{}l}{\textbf{Kalkulationszeiten (nur Fräsen)}} \\
\midrule
Rüstzeit CNC & {{BETRIEB_TR_STK}}\,min & {{CNC_RUEST_GES}}\,min \\
Bearbeitungszeit CNC & {{BETRIEB_TE_STK}}\,min & {{CNC_BEAR_STK}}\,min \\
\textbf{Gesamtzeit CNC pro Stück} & \textbf{{{BETRIEB_ZEIT_STK}}\,min} & \textbf{{{CNC_ZEIT_STK}}\,min} \\
\midrule
\multicolumn{3}{@{}l}{\textbf{Stückpreise}} \\
\midrule
{{BETRIEB}} Herstellkosten & \textbf{{{BETRIEB_HK_STK}}\,€} & — \\
CNC Planer Basis (REFA) & — & {{CNC_BASIS}}\,€ \\
\rowcolor{SurfaceGreen}
\textbf{CNC Planer Empfehlung (ohne Risiko)} & — & \textbf{{{CNC_EMPFEHLUNG}}\,€} \\
\rowcolor{SurfaceGreen}
\textbf{CNC Planer mit allen Zuschlägen} & — & \textbf{{{CNC_ZUSCHLAEGE}}\,€} \\
\rowcolor{SurfaceBlue}
\textbf{{{BETRIEB}} Fräskosten gesamt ({{STUECKZAHL}}\,Stk)} & \multicolumn{2}{c}{\textbf{{{BETRIEB_GES_SUMME}}}} \\
\bottomrule
\end{tabularx}

\vspace{8pt}

\begin{insightbox}[AinaryGreen]{SurfaceGreen}
\textbf{Ohne Risikobetrachtung} liegt der CNC Planer Pro bei \textbf{{{CNC_EMPFEHLUNG}}\,€/Stk}. \textbf{Mit allen Zuschlägen} bei \textbf{{{CNC_ZUSCHLAEGE}}\,€/Stk}.\\[6pt]
{{VERGLEICHSKOMMENTAR}}
\end{insightbox}

\subsection*{8.2\quad Limitationen und Mitigations}

\renewcommand{\arraystretch}{1.3}
\begin{tabularx}{\textwidth}{@{}clXX@{}}
\toprule
& \textbf{Limitation} & \textbf{Auswirkung} & \textbf{Mitigation} \\
\midrule
L1 & \textbf{1-Maschinen-Scope} & Betrieb nutzt ggf. mehrere Maschinen. CNC Planer kalkuliert auf einer. & Bewusste Abgrenzung v1.0. Verfahrweg-Warnung bei Überschreitung. \\
\addlinespace
L2 & \textbf{Vorbearbeitung nicht im Scope} & Schweißen, Glühen etc. fehlen. & Abfrage: ,,Vorbearbeitung nötig?'' mit Hinweis auf manuelle Ergänzung. \\
\addlinespace
L3 & \textbf{REFA vs. Betriebszeiten} & REFA-Richtwerte sind Idealwerte. & Konfigurierbarer Betriebsfaktor (1,4--1,6$\times$). Kalibrierung über Ist-Daten. \\
\addlinespace
L4 & \textbf{Keine Ist-Daten} & Alle Zeiten sind Planwerte. & Ist-Zeiten der ersten 2 Teile erfassen. Automatische Korrekturfaktoren. \\
\bottomrule
\end{tabularx}

% ================================================================
% 9. EHRLICHE BETRACHTUNG
% ================================================================
\section*{9\quad Ehrliche Betrachtung}

{\small Transparenz über Fehler und Grenzen der Kalkulation.}

% ── Fehler offen benennen: Abschnitte anpassen ──

\subsection*{Fehler 1: {{FEHLER_1_TITEL}}}
{{FEHLER_1_TEXT}}

\textbf{Lösung:} {{FEHLER_1_LOESUNG}}

\subsection*{Fehler 2: {{FEHLER_2_TITEL}}}
{{FEHLER_2_TEXT}}

\textbf{Lösung:} {{FEHLER_2_LOESUNG}}

% Weitere Fehler nach Bedarf ergänzen

\subsection*{Selbstkritik an dieser Nachbetrachtung}

\begin{itemize}[leftmargin=*,itemsep=2pt]
  \item {{SELBSTKRITIK_1}}
  \item {{SELBSTKRITIK_2}}
  \item {{SELBSTKRITIK_3}}
\end{itemize}

% ================================================================
% 10. EINSCHÄTZUNG UND AUSBLICK
% ================================================================
\section*{10\quad Einschätzung und Ausblick}

\begin{protocolbox}
{{EINSCHAETZUNG_TEXT}}
\end{protocolbox}

\subsection*{Nächste Meilensteine}

\renewcommand{\arraystretch}{1.3}
\begin{tabularx}{\textwidth}{@{}clX@{}}
\toprule
\textbf{Nr.} & \textbf{Meilenstein} & \textbf{Erwarteter Effekt} \\
\midrule
M1 & Ist-Zeiten erfassen & Erstmals reale Korrekturfaktoren pro Operationstyp \\
M2 & {{MEILENSTEIN_2}} & {{MEILENSTEIN_2_EFFEKT}} \\
M3 & {{MEILENSTEIN_3}} & {{MEILENSTEIN_3_EFFEKT}} \\
M4 & Zweites Bauteil vergleichen & Validierung ob Kalibrierung wirkt \\
\bottomrule
\end{tabularx}

\vspace{8pt}

\begin{insightbox}[AinaryGold]{SurfaceAlt}
\textbf{Fazit:} Der CNC Planer Pro ist ein lernendes System. Diese Nachbetrachtung ist ein Datenpunkt einer Feedback-Schleife: \textit{Kalkulieren → Vergleichen → Fehler erkennen → Kalibrieren → Besser kalkulieren.} Je mehr reale Vergleiche, desto genauer wird das System. Das Ziel ist nicht Perfektion ab Tag 1, sondern \textbf{nachweisbare Verbesserung mit jedem Auftrag}.
\end{insightbox}

\vspace{12pt}
\begin{center}
{\color{AinaryGold}\rule{0.6\linewidth}{0.5pt}}\\[4pt]
{\small\color{gray} Nachbetrachtung auf Basis von CNC Planer Pro {{VERSION}}\\
und {{BETRIEB}} Vorkalkulation ({{BETRIEB_DATUM}}).\\[2pt]
Ist-Daten ausstehend — Dokument wird nach Fertigung aktualisiert.}\\[4pt]
{\small\color{AinaryGold}\textbf{CNC Planer Pro}}
\end{center}

\end{document}
