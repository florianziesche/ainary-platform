% ================================================================
% CNC Planer Pro — Nachbetrachtung / Kalkulationsvergleich
% Lagerungstraverse KBA | 10028104.79
% ================================================================
\documentclass[10pt,a4paper]{article}
\usepackage[left=20mm,right=20mm,top=20mm,bottom=22mm]{geometry}
\usepackage{fontspec}
\usepackage{xcolor}
\usepackage{booktabs}
\usepackage{tabularx}
\usepackage{colortbl}
\usepackage{enumitem}
\usepackage{fancyhdr}
\usepackage{titlesec}
\usepackage{amssymb}
\usepackage{multirow}
\usepackage{tikz}
\usepackage{tcolorbox}
\tcbuselibrary{skins,breakable}
\usepackage{polyglossia}
\setdefaultlanguage{german}

\setmainfont{Helvetica}
\setmonofont{Menlo}[Scale=0.85]

\definecolor{AinaryGold}{HTML}{C8AA50}
\definecolor{AinaryDark}{HTML}{1F2937}
\definecolor{AinaryRed}{HTML}{DC2626}
\definecolor{AinaryGreen}{HTML}{059669}
\definecolor{AinaryBlue}{HTML}{2563EB}
\definecolor{AinaryOrange}{HTML}{D97706}
\definecolor{SurfaceAlt}{HTML}{F9FAFB}
\definecolor{SurfaceWarn}{HTML}{FFFBEB}
\definecolor{SurfaceGreen}{HTML}{E8F5E8}
\definecolor{SurfaceRed}{HTML}{FDE8EA}
\definecolor{SurfaceBlue}{HTML}{E7F1FF}
\definecolor{BorderLight}{HTML}{E5E7EB}
\definecolor{MBSblue}{HTML}{1E3A5F}

\tolerance=2000
\emergencystretch=15pt
\widowpenalty=10000
\clubpenalty=10000

\titleformat{\section}
  {\large\bfseries\color{AinaryDark}}
  {\thesection}{1em}{}[\titlerule]
\titleformat{\subsection}
  {\normalsize\bfseries\color{AinaryDark}}
  {\thesubsection}{1em}{}
\titlespacing{\section}{0pt}{14pt}{6pt}
\titlespacing{\subsection}{0pt}{10pt}{4pt}

\pagestyle{fancy}
\fancyhf{}
\renewcommand{\headrulewidth}{0pt}
\fancyhead[L]{\small\color{AinaryGold}\textbf{CNC Planer Pro} \color{gray}| Nachbetrachtung}
\fancyhead[R]{\small\color{gray}Zeichnung 10028104.79}
\renewcommand{\footrulewidth}{0.4pt}
\fancyfoot[L]{\scriptsize\color{gray} INTERN — Nachbetrachtung | CNC Planer Pro v0.20}
\fancyfoot[C]{\scriptsize\color{gray} Erstellt: 11.02.2026}
\fancyfoot[R]{\scriptsize\color{gray} Seite \thepage}

\newtcolorbox{insightbox}[2][]{%
  enhanced,breakable,
  colback=#2,colframe=#2,
  boxrule=0pt,left=8pt,right=8pt,top=6pt,bottom=6pt,
  borderline west={3pt}{0pt}{#1},
  fontupper=\small
}

\newtcolorbox{protocolbox}{%
  enhanced,
  colback=SurfaceAlt,colframe=BorderLight,
  boxrule=0.5pt,left=8pt,right=8pt,top=6pt,bottom=6pt,
  fontupper=\small
}

\setlength{\parindent}{0pt}
\setlength{\parskip}{4pt}

\begin{document}

% ================================================================
% TITLE
% ================================================================
\begin{center}
{\color{AinaryGold}\rule{\linewidth}{2pt}}\\[8pt]
{\LARGE\bfseries\color{AinaryDark} Nachbetrachtung}\\[3pt]
{\large\color{gray} Kalkulationsvergleich: CNC Planer Pro vs. MBS Vorkalkulation}\\[3pt]
{\small\color{AinaryGold} Lagerungstraverse — Zeichnung 10028104.79 | KBA Koenig \& Bauer}\\[6pt]
{\color{AinaryGold}\rule{\linewidth}{0.5pt}}
\end{center}

\vspace{-2pt}

% ================================================================
% 1. KOPFDATEN
% ================================================================
\section*{1\quad Kopfdaten}

\begin{tabularx}{\textwidth}{@{}lXX@{}}
\toprule
& \textbf{MBS Vorkalkulation} & \textbf{CNC Planer Pro} \\
\midrule
\textbf{Dokument} & Vorkalkulation Nr. 74374 & Kalkulationsbericht v0.20 \\
\textbf{System} & b-logic (ERP) & CNC Planer Pro (KI) \\
\textbf{Datum} & 09.02.2026 & 11.02.2026 \\
\textbf{Bearbeiter} & Björn Krügel (AV) & KI-gestützt (REFA/VDI) \\
\textbf{Kunde} & KBA Koenig \& Bauer AG & KBA Koenig \& Bauer \\
\textbf{Stückzahl Auftrag} & 4 Stück & 4 Stück \\
\textbf{Kalk.-Bezug} & 1 Stück (Menge: 1) & 1 Stück \\
\textbf{Methodik} & Betriebliche Systemzeiten (AV) & REFA-Richtwerte, VDI 3321 \\
\textbf{Scope} & Gesamtfertigung & Nur CNC-Fräsvorgänge \\
\bottomrule
\end{tabularx}

\smallskip
\begin{insightbox}[AinaryBlue]{SurfaceBlue}
\textbf{Abgrenzung:} Dieser Vergleich betrachtet \textbf{ausschließlich die CNC-Fräsvorgänge}. Vor- und nachgelagerte Prozesse (Kontrolle, Sägen, Schweißen, Lackierung) sind nicht im Scope des CNC Planer Pro und werden nicht verglichen. Beide Kalkulationen sind Planwerte — \textbf{Ist-Daten liegen noch nicht vor.}
\end{insightbox}

% ================================================================
% 2. ARBEITSGÄNGE — NUR FRÄSEN
% ================================================================
\section*{2\quad Arbeitsgänge — Nur Fräsvorgänge}

\subsection*{MBS Vorkalkulation (b-logic)}

{\small Die MBS-Kalkulation enthält neben den Fräsvorgängen weitere Positionen (Kontrolle, Sägen, Schweißen, Lackieren), die hier \textit{nicht} betrachtet werden. Nur Pos\,20 und Pos\,25 sind Fräsoperationen.}

\vspace{4pt}

{\small\renewcommand{\arraystretch}{1.25}
\begin{tabularx}{\textwidth}{@{}r p{3cm} r r r r r@{}}
\toprule
\textbf{Pos} & \textbf{Bezeichnung} & \textbf{Tr [min]} & \textbf{Te [min]} & \textbf{HK Lohn} & \textbf{HK Masch} & \textbf{HK Gesamt} \\
\midrule
\rowcolor{SurfaceBlue}
20 & AXA (Fräszentrum) & 450 & 1.800 & 1.417,50\,€ & 1.050,00\,€ & 2.467,50\,€ \\
\rowcolor{SurfaceBlue}
25 & FLP 8000 (Portalfräse) & 240 & 435 & 410,06\,€ & 900,00\,€ & 1.310,06\,€ \\
\midrule
& \textbf{Summe Fräsen} & \textbf{690} & \textbf{2.235} & \textbf{1.827,56\,€} & \textbf{1.950,00\,€} & \textbf{3.777,56\,€} \\
\bottomrule
\end{tabularx}}

{\scriptsize\color{gray} MBS-Werte sind pro Stück (Menge: 1 Stk. lt. Vorkalkulation). Stundensätze rückgerechnet: AXA ca. 66\,€/h, FLP 8000 ca. 82\,€/h.}

\vspace{10pt}

\subsection*{CNC Planer Pro (gleiche Darstellung, pro Stück)}

{\small\renewcommand{\arraystretch}{1.25}
\begin{tabularx}{\textwidth}{@{}r p{3cm} r r r r r@{}}
\toprule
\textbf{AG} & \textbf{Bezeichnung} & \textbf{Rüst [min]} & \textbf{Bear. [min]} & \textbf{HK Lohn} & \textbf{HK Masch} & \textbf{HK Gesamt} \\
\midrule
\rowcolor{SurfaceBlue}
20--70 & CNC-Maschine (Aufsp. 1--4) & 41 & 346 & 195,17\,€ & 258,50\,€ & 453,67\,€ \\
\midrule
& \textbf{Summe Fräsen (1 Stk)} & \textbf{41} & \textbf{346} & \textbf{195,17\,€} & \textbf{258,50\,€} & \textbf{453,67\,€} \\
\bottomrule
\end{tabularx}}

{\scriptsize\color{gray} Werte pro Stück (analog MBS). Rüstzeit: 164\,min ÷ 4\,Stk = 41\,min/Stk. Maschine: CNC-Maschine (3-Achs BAZ). Stundensatz: 70\,€/h (38 Lohn + 32 Maschine).}

% ================================================================
% 3. FERTIGUNGSZEITEN — GEGENÜBERSTELLUNG
% ================================================================
\section*{3\quad Fertigungszeiten — Gegenüberstellung}

\begin{insightbox}[AinaryBlue]{SurfaceBlue}
\textbf{Bezugsgröße:} MBS-Vorkalkulation ist auf \textbf{Menge: 1 Stück} ausgelegt. CNC Planer Pro kalkuliert ebenfalls \textbf{pro Stück}. Alle folgenden Werte sind direkt vergleichbar.
\end{insightbox}

\vspace{6pt}

\subsection*{3.1\quad Zeiten pro Stück (direkt aus Dokumenten)}

\renewcommand{\arraystretch}{1.35}
{\small
\begin{tabularx}{\textwidth}{@{}Xrrrr@{}}
\toprule
\textbf{Position} & \textbf{MBS Tr [min]} & \textbf{MBS Te [min]} & \textbf{CNC Planer [min]} & \textbf{Faktor} \\
\midrule
AXA (Fräszentrum) & 450 & 1.800 & \multirow{2}{*}{346\textsuperscript{a}} & \multirow{2}{*}{—} \\
FLP 8000 (Portalfräse) & 240 & 435 & & \\
\midrule
\textbf{Summe Rüstzeit (Tr)} & \multicolumn{2}{r}{\textbf{690\,min (11,5\,h)}} & \textbf{41\,min\textsuperscript{b}} & \textbf{16,8$\times$} \\
\textbf{Summe Bearbeitungszeit (Te)} & \multicolumn{2}{r}{\textbf{2.235\,min (37,3\,h)}} & \textbf{346\,min (5,8\,h)} & \textbf{6,5$\times$} \\
\midrule
\rowcolor{SurfaceWarn}
\textbf{Summe Fräsen pro Stück} & \multicolumn{2}{r}{\textbf{2.925\,min (48,8\,h)}} & \textbf{387\,min (6,5\,h)} & \textbf{7,6$\times$} \\
\midrule
\multicolumn{5}{@{}l}{\textbf{Gesamtfertigung pro Stück (alle Arbeitsgänge)}} \\
\midrule
Fertigung (alle AG) & \multicolumn{2}{r}{—\textsuperscript{c}} & 497\,min (8,3\,h) & — \\
Rüst anteilig (÷ 4\,Stk) & \multicolumn{2}{r}{—} & 41\,min & — \\
\rowcolor{SurfaceWarn}
\textbf{Gesamtzeit pro Stück} & \multicolumn{2}{r}{\textbf{2.925\,min (48,8\,h)}} & \textbf{538\,min (9,0\,h)} & \textbf{5,4$\times$} \\
\bottomrule
\end{tabularx}}

{\scriptsize\textsuperscript{a} CNC Planer: 8 Fräs-AG (346\,min) auf einer Maschine, 4 Aufspannungen.\\
\textsuperscript{b} CNC Planer Rüstzeit: 164\,min ÷ 4\,Stk = 41\,min/Stk (Losgrößen-Umlage).\\
\textsuperscript{c} MBS Gesamtfertigung enthält weitere Positionen (Kontrolle, Sägen, Schweißen, Lackieren) — nicht 1:1 vergleichbar.}

\vspace{6pt}

\subsection*{3.2\quad Hochrechnung auf 4 Stück (Gesamtauftrag)}

\renewcommand{\arraystretch}{1.35}
{\small
\begin{tabularx}{\textwidth}{@{}Xrrr@{}}
\toprule
\textbf{Position} & \textbf{MBS} & \textbf{CNC Planer} & \textbf{Herleitung} \\
\midrule
Rüstzeit (einmalig) & 690\,min & 164\,min & MBS: 450+240. CNC: lt. Kalkulation \\
Bearbeitungszeit Fräsen ($\times$4) & 8.940\,min & 1.384\,min & MBS: 2.235$\times$4. CNC: 346$\times$4 \\
Bearbeitungszeit gesamt ($\times$4) & — & 1.988\,min & CNC: 497$\times$4 (alle 11 AG) \\
\midrule
\rowcolor{SurfaceWarn}
\textbf{Gesamtzeit (4 Stk, nur Fräsen)} & \textbf{9.630\,min} & \textbf{1.548\,min} & \\
\textbf{Gesamtzeit (4 Stk, alle AG)} & — & \textbf{2.152\,min (35,9\,h)} & 497$\times$4 + 164 Rüst \\
\bottomrule
\end{tabularx}}

{\scriptsize Arbeitstag = 8\,h. MBS Rüstzeiten: Unklar ob einmalig oder pro Stück — hier als einmalig angenommen. Bei Rüstzeit pro Stück: MBS gesamt = 11.700\,min (195\,h).}

\vspace{6pt}

\begin{insightbox}[AinaryOrange]{SurfaceWarn}
\textbf{Erklärung der Abweichung (Faktor 7,6$\times$):}\\[4pt]
Die MBS-Kalkulation enthält \textbf{zwei Maschinen} mit separaten Rüst- und Bearbeitungszeiten. Der CNC Planer kennt nur \textbf{eine Maschine}. Darüber hinaus:
\begin{itemize}[leftmargin=*,itemsep=2pt]
  \item \textbf{MBS Systemzeiten} enthalten AV-Nebenzeiten, Wartezeiten, Sicherheitspuffer — höher als REFA-Idealwerte
  \item \textbf{FLP 8000 nicht erfasst:} 675\,min (Tr\,240 + Te\,435) fehlen komplett im CNC Planer
  \item \textbf{Rüstzeiten:} MBS 690\,min vs. CNC Planer 164\,min — MBS rüstet vermutlich für 2 Maschinen separat
  \item \textbf{REFA vs. Praxis:} REFA-Richtwerte sind typischerweise Faktor 1,4--1,6 unter betrieblichen Systemzeiten
\end{itemize}
\textbf{Die Zeitabweichung ist die größte offene Frage dieses Vergleichs.} Erst Ist-Zeiten werden klären, welche Kalkulation näher an der Realität liegt.
\end{insightbox}

% ================================================================
% 4. PREISVERGLEICH — NUR FRÄSEN
% ================================================================
\section*{4\quad Preisvergleich — Nur Fräsen, pro Stück}

\subsection*{Preisvergleich pro Stück (nur Fräsen)}

\renewcommand{\arraystretch}{1.35}
\begin{tabularx}{\textwidth}{@{}Xrrr@{}}
\toprule
\textbf{Position} & \textbf{MBS} & \textbf{CNC Planer} & \textbf{Delta} \\
\midrule
\multicolumn{4}{@{}l}{\textbf{Fräskosten pro Stück (vergleichbar)}} \\
\midrule
AXA / CNC-Maschine & 2.467,50\,€ & 453,67\,€ & — \\
FLP 8000 (Portalfräse) & 1.310,06\,€ & — & — \\
\textbf{Summe Fräsen pro Stück} & \textbf{3.777,56\,€} & \textbf{453,67\,€} & — \\
\midrule
CNC Planer Basispreis & — & 2.383,58\,€ & — \\
\rowcolor{SurfaceGreen}
\textbf{CNC Planer Empfehlung (ohne Risiko)} & — & \textbf{2.750,00\,€} & — \\
\rowcolor{SurfaceGreen}
\textbf{CNC Planer mit allen Zuschlägen} & — & \textbf{3.112,00\,€} & $-$17,6\% \\
\midrule
\rowcolor{SurfaceAlt}
\textit{MBS Gesamtherstellkosten pro Stück} & \textit{5.377,11\,€} & — & \textit{nicht vergleichbar\textsuperscript{*}} \\
\bottomrule
\end{tabularx}

{\scriptsize\textsuperscript{*} MBS Gesamtherstellkosten enthalten neben Fräsen auch Kontrolle, Sägen, Schweißen, Lackierung und Material — nicht im Scope des CNC Planer Pro.\\
Alle Werte pro Stück. MBS lt. Vorkalkulation (Menge: 1 Stk.). CNC Planer: Empfehlung enthält Zeitkorrekturen + Risikopuffer. ,,Mit allen Zuschlägen'' enthält Kran, NC-Prog, H7-QS, Großkunde (siehe Abschnitt 4+5).}

% ================================================================
% 5. HERSTELLKOSTEN & DECKUNGSBEITRAGRECHNUNG
% ================================================================
\section*{5\quad Herstellkosten \& Deckungsbeitragrechnung}

{\small Aufbau analog MBS-BAB: Materialkosten → Fertigungskosten → Herstellkosten → DB-Rechnung. Alle Werte pro Stück.}

\vspace{6pt}

\subsection*{5.1\quad Herstellkosten pro Stück}

\renewcommand{\arraystretch}{1.3}
\begin{tabularx}{\textwidth}{@{}Xrrr@{}}
\toprule
\textbf{Position} & \textbf{MBS} & \textbf{CNC Planer} & \textbf{Anmerkung} \\
\midrule
\multicolumn{4}{@{}l}{\textbf{Materialkosten}} \\
\midrule
Material (Rohteil GJS-700) & 1.228,60\,€\textsuperscript{a} & 1.200,00\,€ & Beistellung KBA \\
MGK (5\%) & 61,43\,€ & 60,00\,€ & Materialgemeinkostenzuschlag \\
\textbf{Summe Material} & \textbf{1.290,03\,€} & \textbf{1.260,00\,€} & \\
\midrule
\multicolumn{4}{@{}l}{\textbf{Fertigungskosten (nur Fräsen, vergleichbar)}} \\
\midrule
Fertigungslohn & 1.827,56\,€ & 195,17\,€ & MBS: AXA+FLP. CNC: 346\,min $\times$ 38\,€/h \\
Maschinenkosten & 1.950,00\,€ & 258,50\,€ & MBS: AXA+FLP. CNC: 346\,min $\times$ 32\,€/h \\
Rüstkosten (anteilig ÷ 4\,Stk) & —\textsuperscript{b} & 47,83\,€ & CNC: 164\,min ÷ 4 $\times$ 70\,€/h \\
\textbf{Summe Fräsen} & \textbf{3.777,56\,€} & \textbf{501,50\,€} & \\
\midrule
\multicolumn{4}{@{}l}{\textbf{Weitere Fertigungskosten (nur MBS, nicht im CNC Planer Scope)}} \\
\midrule
\textit{Kontrolle (Pos 10)} & \textit{28,67\,€} & — & Nicht im Scope \\
\textit{Sägen (Pos 15)} & \textit{128,00\,€} & — & Nicht im Scope \\
\textit{Schweißen (Pos 17)} & \textit{36,00\,€} & — & Nicht im Scope \\
\textit{Lackieren (Pos 30)} & \textit{116,85\,€} & — & Nicht im Scope \\
\midrule
\multicolumn{4}{@{}l}{\textbf{Herstellkosten}} \\
\midrule
\rowcolor{SurfaceBlue}
\textbf{HK gesamt (Fräsen + Material)} & \textbf{5.067,59\,€} & \textbf{1.761,50\,€} & Vergleichbar \\
\rowcolor{SurfaceAlt}
\textit{MBS HK inkl. aller Positionen} & \textit{5.377,11\,€} & — & Nicht vergleichbar \\
\bottomrule
\end{tabularx}

{\scriptsize\textsuperscript{a} MBS Material aus Preisanteile Grenzkosten (Scan). \textsuperscript{b} MBS Rüstkosten sind in Tr enthalten (450+240\,min).\\
CNC Planer: 11 AG gesamt (497\,min), davon 8 AG Fräsen (346\,min). AG10 Sägen, AG100 Entgraten, AG110 QS hier nicht gezeigt.}

\vspace{8pt}

\subsection*{5.2\quad MBS Deckungsbeitragrechnung (aus Vorkalkulation)}

{\small Vollständige DB-Rechnung lt. MBS-Vorkalkulation Nr. 74374, pro Stück:}

\vspace{4pt}
\renewcommand{\arraystretch}{1.35}
\begin{tabularx}{\textwidth}{@{}Xlrr@{}}
\toprule
\textbf{Position} & \textbf{Zuschlag} & \textbf{MBS} & \textbf{Anmerkung} \\
\midrule
Material & & 1.228,60\,€ & \\
Maschinen & & 2.056,67\,€ & \\
Lohn & & 1.504,75\,€ & \\
\textbf{(1) Grenzkosten} & & \textbf{4.790,02\,€} & Variable Kosten \\
\midrule
+ Gemeinkostenzuschlag (GKZ) & 12,28\% & 588,09\,€ & \\
\textbf{(2) Herstellkosten} & & \textbf{5.378,11\,€} & \\
\midrule
+ Verwaltung \& Vertrieb (VuV) & 20,00\% & 1.075,62\,€ & \\
\textbf{(3) Selbstkosten} & & \textbf{6.453,73\,€} & \\
\midrule
+ Gewinnzuschlag & 10,00\% & 645,37\,€ & \\
\rowcolor{SurfaceBlue}
\textbf{Kalkulierter Verkaufspreis} & & \textbf{7.099,10\,€} & \\
\midrule
\multicolumn{4}{@{}l}{\textbf{Deckungsbeiträge (MBS)}} \\
\midrule
\textbf{DB\,I} (VK $-$ Grenzkosten) & & \textbf{2.309,09\,€} & 32,5\% \\
\textbf{DB\,II} (VK $-$ Herstellkosten) & & \textbf{1.720,99\,€} & 24,2\% \\
\textbf{Gewinn} (VK $-$ Selbstkosten) & & \textbf{645,37\,€} & 9,1\% \\
\bottomrule
\end{tabularx}

\vspace{8pt}

\subsection*{5.3\quad CNC Planer Pro — gleiche Struktur}

{\small CNC Planer Kalkulation in MBS-Aufbau überführt. Alle Werte pro Stück.}

\vspace{4pt}
\renewcommand{\arraystretch}{1.35}
\begin{tabularx}{\textwidth}{@{}Xlrr@{}}
\toprule
\textbf{Position} & \textbf{Zuschlag} & \textbf{CNC Planer} & \textbf{Anmerkung} \\
\midrule
Material & & 1.260,00\,€ & Beistellung KBA \\
Fertigung (11 AG, 497\,min) & & 523,97\,€ & div. Sätze (31--70\,€/h) \\
Rüstkosten (164\,min ÷ 4\,Stk) & & 47,83\,€ & 41\,min $\times$ 70\,€/h \\
\textbf{(1) Grenzkosten} & & \textbf{1.831,80\,€} & 538\,min = 9,0\,h/Stk \\
\midrule
+ GKZ (analog MBS) & 12,28\% & 224,95\,€ & \\
\textbf{(2) Herstellkosten} & & \textbf{2.056,75\,€} & \\
\midrule
+ VuV (analog MBS) & 20,00\% & 411,35\,€ & \\
\textbf{(3) Selbstkosten} & & \textbf{2.468,10\,€} & \\
\midrule
+ Gewinn (analog MBS) & 10,00\% & 246,81\,€ & \\
\rowcolor{SurfaceAlt}
\textit{Kalk. VK (MBS-Methodik)} & & \textit{2.714,91\,€} & Rein rechnerisch \\
\midrule
\multicolumn{4}{@{}l}{\textbf{CNC Planer Angebotspreis (eigene Zuschlagsmethodik)}} \\
\midrule
CNC Planer Empfehlung & & 2.750,00\,€ & Korrigierte REFA-Zeiten \\
\quad + Kran-Handling & & 178,00\,€ & 1.415\,kg \\
\quad + NC-Programmierung (÷ 4) & & 70,00\,€ & 280\,€ einmalig \\
\quad + QS H7 (30\,min) & & 35,00\,€ & 12$\times$ H7-Passungen \\
\quad + Großkunde 15\% & & 79,00\,€ & KBA \\
\rowcolor{SurfaceGreen}
\textbf{Angebotspreis CNC Planer} & & \textbf{3.112,00\,€} & \\
\midrule
\multicolumn{4}{@{}l}{\textbf{Deckungsbeiträge bei 3.112\,€}} \\
\midrule
\textbf{DB\,I} (AP $-$ Grenzkosten) & & \textbf{1.280,20\,€} & 41,1\% \\
\textbf{DB\,II} (AP $-$ Herstellkosten) & & \textbf{1.055,25\,€} & 33,9\% \\
\textbf{Gewinn} (AP $-$ Selbstkosten) & & \textbf{643,90\,€} & 20,7\% \\
\bottomrule
\end{tabularx}

{\scriptsize Fertigung: 497\,min (8,3\,h) lt. CNC Planer Kalkulationsbericht. Rüst: 164\,min ÷ 4\,Stk = 41\,min. Gesamt: 538\,min (9,0\,h) pro Stück.\\
GKZ, VuV und Gewinn-Zuschlag: MBS-Sätze übernommen (12,28\% / 20\% / 10\%) für Vergleichbarkeit.}

\vspace{8pt}

\subsection*{5.4\quad Gegenüberstellung}

\renewcommand{\arraystretch}{1.35}
\begin{tabularx}{\textwidth}{@{}Xrrr@{}}
\toprule
& \textbf{MBS} & \textbf{CNC Planer} & \textbf{Delta} \\
\midrule
Verkaufspreis & 7.099,10\,€ & 3.112,00\,€ & $-$56\% \\
Grenzkosten & 4.790,02\,€ & 1.831,80\,€ & $-$62\% \\
Herstellkosten & 5.378,11\,€ & 2.056,75\,€ & $-$62\% \\
Selbstkosten & 6.453,73\,€ & 2.468,10\,€ & $-$62\% \\
\midrule
\textbf{DB\,I} & \textbf{2.309,09\,€} (32,5\%) & \textbf{1.280,20\,€} (41,1\%) & — \\
\textbf{DB\,II} & \textbf{1.720,99\,€} (24,2\%) & \textbf{1.055,25\,€} (33,9\%) & — \\
\textbf{Gewinn} & \textbf{645,37\,€} (9,1\%) & \textbf{643,90\,€} (20,7\%) & — \\
\bottomrule
\end{tabularx}

{\scriptsize CNC Planer: niedrigere Absolutwerte aber höhere prozentuale Margen weil Grenzkosten deutlich niedriger (1.832\,€ vs. 4.790\,€).\\
Gewinn absolut nahezu identisch (645\,€ vs. 644\,€) — bei komplett unterschiedlichen Preispunkten.}

\vspace{6pt}

\begin{insightbox}[AinaryOrange]{SurfaceWarn}
\textbf{Betrachtung:} MBS kalkuliert einen VK von \textbf{7.099\,€/Stk}, CNC Planer empfiehlt \textbf{3.112\,€/Stk} — Faktor 2,3$\times$. Die Differenz liegt fast ausschließlich in den \textbf{Fertigungszeiten}: MBS 2.925\,min vs. CNC Planer 538\,min (Faktor 5,4$\times$). Ursachen:
\begin{itemize}[leftmargin=*,itemsep=2pt]
  \item \textbf{Zwei Maschinen} bei MBS (AXA + FLP 8000) vs. eine beim CNC Planer
  \item \textbf{Systemzeiten vs. REFA:} Betriebliche Zeiten enthalten Nebenzeiten, Sicherheitspuffer
  \item \textbf{Scope:} MBS enthält Schweißen (36\,€), Lackieren (117\,€), Kontrolle (29\,€)
\end{itemize}
\textbf{Bemerkenswert:} Der absolute Gewinn ist bei beiden fast gleich (~645\,€) — aber bei völlig unterschiedlichen Preisen und Kostenstrukturen. \textbf{Ist-Zeiten werden klären, welche Grenzkosten realistisch sind.}
\end{insightbox}

% ================================================================
% 5. PREISEMPFEHLUNG
% ================================================================
\section*{6\quad Preisempfehlung mit allen Hinweisen}

\subsection*{Angebotsoptionen}

\renewcommand{\arraystretch}{1.35}
\begin{tabularx}{\textwidth}{@{}p{4.5cm}rrX@{}}
\toprule
\textbf{Variante} & \textbf{Stückpreis} & \textbf{4 Stück} & \textbf{Anmerkung} \\
\midrule
Basispreis & 2.384\,€ & 9.534\,€ & REFA-Richtwerte, ohne Korrekturen \\
Betriebsleiter-Korrektur & 2.680\,€ & 10.720\,€ & + fehlende Positionen \\
\rowcolor{SurfaceGreen}
\textbf{Empfehlung} & \textbf{2.750\,€} & \textbf{11.000\,€} & \textbf{Korrigierte Zeiten + Risikopuffer} \\
REFA-Korrektur & 2.900\,€ & 11.600\,€ & Konservative Zeitkorrektur \\
Premiumkunde (KBA) & 2.950\,€ & 11.800\,€ & Mit 85\,€/h CNC-Satz \\
Sicherheitsmarge & 3.100\,€ & 12.400\,€ & Erstauftrag + GJS-700 Risiko \\
\bottomrule
\end{tabularx}

\vspace{8pt}

\subsection*{Empfehlung mit allen Zuschlägen eingepreist}

\renewcommand{\arraystretch}{1.3}
\begin{tabular}{@{}p{6cm}rr@{}}
\toprule
\textbf{Position} & \textbf{pro Stück} & \textbf{Quelle} \\
\midrule
CNC Planer Empfehlung (Basis) & 2.750,00\,€ & Korrigierte REFA-Zeiten \\
\quad + Schwerlast / Kran-Handling & 178,00\,€ & Erfahrungswert \\
\quad + NC-Programmierung (einmalig ÷ 4) & 70,00\,€ & 280\,€ / 4 Stück \\
\quad + QS-Aufwand H7 (30\,min × 70\,€/h) & 35,00\,€ & 12× H7-Passungen \\
\quad + Großkunde-Zuschlag 15\% auf Fertigung & 79,00\,€ & Marktdaten KBA \\
\midrule
\rowcolor{SurfaceGreen}
\textbf{Empfohlener Angebotspreis} & \textbf{3.112,00\,€} & \\
\rowcolor{SurfaceGreen}
\textbf{Auftragswert (4 Stück, netto)} & \textbf{12.448,00\,€} & \\
\quad + MwSt 19\% & 2.365,12\,€ & \\
\rowcolor{SurfaceGreen}
\textbf{Auftragswert (4 Stück, brutto)} & \textbf{14.813,12\,€} & \\
\bottomrule
\end{tabular}

\vspace{6pt}
{\scriptsize\color{gray} Optional: Erstfertigung-Risikozuschlag (Ausschuss 1 von 4 Stk = +25\% → Stückpreis 3.890\,€). Empfehlung: Risiko über Klausel absichern statt einpreisen.}

% ================================================================
% 5. PREISRELEVANTE HINWEISE
% ================================================================
\section*{7\quad Preisrelevante Hinweise}

\begin{insightbox}[AinaryRed]{SurfaceRed}
\textbf{$\blacktriangle$ Schwerlast — Handling-Zuschlag (+178\,€/Stk)}\\[2pt]
Bauteil ca. 1.415\,kg. Kran- oder Staplernutzung für jede Aufspannung nötig (4× Kran-Einsatz pro Stück). Transport, Verpackung und Versicherung als separate Positionen anbieten. Ggf. Sondertransport für 2\,m+ Bauteile.\\[2pt]
{\footnotesize Quelle: Erfahrungswerte Lohnfertigung. In Empfehlung eingepreist.}
\end{insightbox}

\begin{insightbox}[AinaryOrange]{SurfaceWarn}
\textbf{$\blacktriangle$ Erstauftrag — NC-Programmierung + Risiko (+70\,€/Stk)}\\[2pt]
Kein Wiederholauftrag. NC-Programmierung: 3--4\,h = 210--280\,€ \textbf{einmalig}, auf 4 Stück umgelegt = 70\,€/Stk. Bei Folgeaufträgen entfällt dieser Posten.\\[4pt]
\textbf{Ausschussrisiko:} Bei 4\,Stk kein Ersatzteil vorgesehen. 1 Ausschussteil = +25\% Kosten auf den gesamten Auftrag. Empfehlung: Klausel im Angebot: \textit{,,Preis gilt nach Erstteileprüfung bei Gussqualität wie Muster. Ausschuss durch Materialfehler trägt der Auftraggeber.''}\\[2pt]
{\footnotesize NC-Programmierung in Empfehlung eingepreist. Ausschussrisiko über Klausel abgesichert.}
\end{insightbox}

\begin{insightbox}[AinaryGreen]{SurfaceGreen}
\textbf{$\blacktriangle$ Großkunde — höherer Stundensatz (+79\,€/Stk)}\\[2pt]
KBA (Koenig \& Bauer) ist börsennotiert mit {>}\,1\,Mrd.\,€ Umsatz. Solche Kunden sind Stundensätze von 85--95\,€/h gewohnt (vs. 70\,€/h kalkuliert). Empfehlung: +15\% auf Fertigungskosten.\\[2pt]
{\footnotesize Quelle: Marktdaten Sachsen Q4/2025. In Empfehlung eingepreist.}
\end{insightbox}

\begin{insightbox}[AinaryBlue]{SurfaceBlue}
\textbf{$\circlearrowleft$ QS-Aufwand H7-Passungen (+35\,€/Stk)}\\[2pt]
12$\times$ H7-Bohrungen erfordern Einzelprüfung mit Lehrring/Innenmessschraube. Mind. 30\,min/Stück zusätzlicher QS-Aufwand über die Standard-QS (AG110) hinaus.\\[4pt]
\textbf{Maschinenbelegung:} 8,3\,h Bearbeitung pro Stück = mehr als eine Schicht. Bei 4 Stück: ca. 5 Maschinentage (inkl. Rüstung). Prüfen: Maschinenauslastung, ggf. Nachtschicht-Zuschlag.\\[2pt]
{\footnotesize H7-Zuschlag in Empfehlung eingepreist. Nachtschicht-Zuschlag nicht enthalten.}
\end{insightbox}

% ================================================================
% 6. IST-ZEITEN (zum Ausfüllen)
% ================================================================
\section*{8\quad Ist-Zeiten (nach Fertigung eintragen)}

{\small\renewcommand{\arraystretch}{1.25}
\begin{tabularx}{\textwidth}{@{}Xrrrr@{}}
\toprule
\textbf{Arbeitsgang} & \textbf{CNC Planer} & \textbf{MBS (AV)} & \textbf{IST Teil 1} & \textbf{IST Teil 2} \\
\midrule
Planfräsen Unterseite & 55\,min & — & \textcolor{gray}{\textit{\_\_\_\_\_\_}} & \textcolor{gray}{\textit{\_\_\_\_\_\_}} \\
Bohrungen Unterseite & 44\,min & — & \textcolor{gray}{\textit{\_\_\_\_\_\_}} & \textcolor{gray}{\textit{\_\_\_\_\_\_}} \\
Planfräsen Oberseite & 52\,min & — & \textcolor{gray}{\textit{\_\_\_\_\_\_}} & \textcolor{gray}{\textit{\_\_\_\_\_\_}} \\
Taschen fräsen (4$\times$) & 46\,min & — & \textcolor{gray}{\textit{\_\_\_\_\_\_}} & \textcolor{gray}{\textit{\_\_\_\_\_\_}} \\
Langlöcher (3$\times$) & 24\,min & — & \textcolor{gray}{\textit{\_\_\_\_\_\_}} & \textcolor{gray}{\textit{\_\_\_\_\_\_}} \\
Konturfräsen & 28\,min & — & \textcolor{gray}{\textit{\_\_\_\_\_\_}} & \textcolor{gray}{\textit{\_\_\_\_\_\_}} \\
Stirnseite 1 & 40\,min & — & \textcolor{gray}{\textit{\_\_\_\_\_\_}} & \textcolor{gray}{\textit{\_\_\_\_\_\_}} \\
Stirnseite 2 & 57\,min & — & \textcolor{gray}{\textit{\_\_\_\_\_\_}} & \textcolor{gray}{\textit{\_\_\_\_\_\_}} \\
Entgraten & 68\,min & — & \textcolor{gray}{\textit{\_\_\_\_\_\_}} & \textcolor{gray}{\textit{\_\_\_\_\_\_}} \\
QS / Messprotokoll & 55\,min & — & \textcolor{gray}{\textit{\_\_\_\_\_\_}} & \textcolor{gray}{\textit{\_\_\_\_\_\_}} \\
\midrule
\textbf{Summe} & \textbf{469\,min} & — & \textcolor{gray}{\textit{\_\_\_\_\_\_}} & \textcolor{gray}{\textit{\_\_\_\_\_\_}} \\
\bottomrule
\end{tabularx}}

% ================================================================
% 7. OFFENE FRAGEN
% ================================================================
\section*{9\quad Offene Fragen}

\begin{enumerate}[leftmargin=*,itemsep=4pt]
  \item \textbf{MBS-Mengeneinheit:} Sind die HK-Werte in der Vorkalkulation für 1 Stück oder den Gesamtauftrag (4\,Stk)?
  \item \textbf{FLP 8000:} Welche Bearbeitungen laufen auf der Portalfräse? Deckt der CNC Planer diese auf einer Maschine mit ab, oder ist es zusätzliche Arbeit?
  \item \textbf{Schweißen (MBS Pos\,17):} Wird die Traverse aus Segmenten geschweißt? Falls ja: Schweißnaht-Nacharbeit und Verzugskompensation fehlen im CNC Planer.
  \item \textbf{Verfahrwege:} Hat die geplante Maschine ausreichend Verfahrweg für 2.095\,mm?
  \item \textbf{Ist-Zeiten:} Protokollierung der ersten 2 Teile AG für AG — Grundlage für Kalibrierung.
\end{enumerate}

% ================================================================
% 8. GESAMTVERGLEICH
% ================================================================
\clearpage
\section*{10\quad Gesamtvergleich — Gegenüberstellung}

\subsection*{10.1\quad Zeiten und Preise}

\renewcommand{\arraystretch}{1.4}
\begin{tabularx}{\textwidth}{@{}Xrr@{}}
\toprule
\textbf{Position} & \textbf{MBS (pro Stk)} & \textbf{CNC Planer (pro Stk)} \\
\midrule
\multicolumn{3}{@{}l}{\textbf{Kalkulationszeiten pro Stück (nur Fräsen)}} \\
\midrule
Rüstzeit CNC & 690\,min & 41\,min \\
Bearbeitungszeit CNC & 2.235\,min & 346\,min \\
\textbf{Gesamtzeit Fräsen pro Stück} & \textbf{2.925\,min (48,8\,h)} & \textbf{387\,min (6,5\,h)} \\
Gesamtzeit alle AG pro Stück & — & 538\,min (9,0\,h) \\
\midrule
\multicolumn{3}{@{}l}{\textbf{Fräskosten pro Stück (vergleichbar)}} \\
\midrule
\textbf{Summe Fräsen} & \textbf{3.777,56\,€} & \textbf{453,67\,€} \\
\midrule
\multicolumn{3}{@{}l}{\textbf{Stückpreise}} \\
\midrule
CNC Planer Basis (REFA) & — & 2.383,58\,€ \\
\rowcolor{SurfaceGreen}
\textbf{CNC Planer Empfehlung (ohne Risiko)} & — & \textbf{2.750,00\,€} \\
\rowcolor{SurfaceGreen}
\textbf{CNC Planer mit allen Zuschlägen} & — & \textbf{3.112,00\,€} \\
\midrule
\rowcolor{SurfaceAlt}
\textit{MBS Gesamtherstellkosten} & \textit{5.377,11\,€} & \textit{nicht vergleichbar\textsuperscript{*}} \\
\bottomrule
\end{tabularx}

{\scriptsize\textsuperscript{*} MBS Gesamtherstellkosten enthalten neben Fräsen auch Kontrolle, Sägen, Schweißen, Lackierung und Material.\\
Alle Werte pro Stück. MBS lt. Vorkalkulation (Menge: 1 Stk.). CNC Planer Rüstzeit: 164\,min ÷ 4\,Stk = 41\,min/Stk.}

\vspace{8pt}

\begin{insightbox}[AinaryGreen]{SurfaceGreen}
\textbf{Ohne Risikobetrachtung} liegt der CNC Planer Pro bei \textbf{2.750\,€/Stk}. \textbf{Mit allen Zuschlägen} bei \textbf{3.112\,€/Stk} — das sind \textbf{17,6\% unter den MBS-Fräskosten} (3.777,56\,€/Stk).\\[6pt]
Dabei war dem CNC Planer \textbf{nicht bekannt}, dass MBS zwei Fräsmaschinen einsetzt (AXA + FLP 8000). Die gesamte Kalkulation basiert auf einer einzelnen CNC-Maschine.
\end{insightbox}

\subsection*{10.2\quad Limitationen des CNC Planer Pro bei diesem Bauteil}

\renewcommand{\arraystretch}{1.3}
\begin{tabularx}{\textwidth}{@{}clXX@{}}
\toprule
& \textbf{Limitation} & \textbf{Auswirkung} & \textbf{Mitigation} \\
\midrule
L1 & \textbf{Zweite Maschine nicht erkannt} & FLP 8000 (Portalfräse) fehlt komplett. MBS teilt Arbeit auf 2 Maschinen auf — CNC Planer plant nur 1. & Multi-Maschinen-Planung implementieren. Großteile ({>}1.500\,mm) automatisch prüfen ob Verfahrwege ausreichen. \\
\addlinespace
L2 & \textbf{Schweißen nicht im Scope} & MBS schweißt vor dem Fräsen (Pos\,17). Falls die Traverse aus Segmenten besteht, fehlen Schweißnaht-Nacharbeit und Verzugskompensation. & Schweißbaugruppen-Erkennung einführen. Abfrage: ,,Rohteil geschweißt? Ja/Nein'' mit Zeitzuschlag. \\
\addlinespace
L3 & \textbf{Verfahrwege nicht geprüft} & CNC-Maschine hat ca. 850\,mm X-Verfahrweg. Bauteil ist 2.095\,mm. Maschine ist möglicherweise zu klein. & Automatische Verfahrweg-Prüfung gegen Bauteilabmessungen. Warnung wenn Bauteil {>} Maschinenraum. \\
\addlinespace
L4 & \textbf{REFA vs. Betriebszeiten} & REFA-Richtwerte sind Idealwerte. Betriebliche Systemzeiten enthalten AV-Nebenzeiten, Wartezeiten, Sicherheitspuffer. & Konfigurierbarer Betriebsfaktor (z.\,B. 1,4--1,6$\times$ auf REFA-Zeiten). Kalibrierung über Ist-Daten. \\
\addlinespace
L5 & \textbf{Keine Ist-Daten vorhanden} & Alle Zeiten und Preise sind Planwerte. Kalibrierung ist erst nach Fertigung möglich. & Ist-Zeiten der ersten 2 Teile AG für AG erfassen. Automatische Korrekturfaktor-Berechnung im Planer. \\
\bottomrule
\end{tabularx}

\vspace{8pt}

\subsection*{10.3\quad Bewertung}

\begin{protocolbox}
\textbf{Was funktioniert:}
\begin{itemize}[leftmargin=*,itemsep=2pt,label={\color{AinaryGreen}$\checkmark$}]
  \item Materialkosten in plausibler Größenordnung
  \item Risikoanalyse und Peer Reviews warnen korrekt vor zu engen Zeiten
  \item Preiszuschläge (Kran, Großkunde etc.) bringen Kalkulation in Richtung MBS-Niveau
\end{itemize}

\textbf{Was fehlt:}
\begin{itemize}[leftmargin=*,itemsep=2pt,label={\color{AinaryRed}$\times$}]
  \item Zweite Maschine (Portalfräse) nicht eingeplant
  \item Schweißvorgänge nicht berücksichtigt
  \item Verfahrweg-Check gegen Bauteilgröße
  \item Betriebsspezifische Kalibrierung (mangels Ist-Daten)
  \item Endpreis 3.112\,€ liegt 17,6\% unter MBS — Abweichung noch zu hoch für verlässliche Angebote
\end{itemize}

\textbf{Nächste Schritte:}
\begin{enumerate}[leftmargin=*,itemsep=2pt]
  \item Klären: Wird geschweißt? Reicht eine Maschine?
  \item Ist-Zeiten Teil 1 + 2 erfassen → Kalibrierung
  \item Limitationen L1--L3 in CNC Planer Pro implementieren
\end{enumerate}
\end{protocolbox}

% Dokument endet nach Abschnitt 8
% (Abschnitt 9+10 entfernt)

\vspace{12pt}
\begin{center}
{\color{AinaryGold}\rule{0.6\linewidth}{0.5pt}}\\[4pt]
{\small\color{gray} Nachbetrachtung auf Basis von CNC Planer Pro v0.20\\
und MBS Vorkalkulation Nr. 74374 (b-logic, 09.02.2026).\\[2pt]
Ist-Daten ausstehend — Dokument wird nach Fertigung aktualisiert.}\\[4pt]
{\small\color{AinaryGold}\textbf{CNC Planer Pro}}
\end{center}

\end{document}
