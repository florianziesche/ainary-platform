\documentclass[a4paper,10pt]{article}

% XeLaTeX mit Fontspec
\usepackage{fontspec}
\setmainfont{Helvetica}
\setmonofont{Menlo}

% Deutsche Sprache
\usepackage{polyglossia}
\setdefaultlanguage{german}

% Layout
\usepackage[left=2.5cm,right=2.5cm,top=3cm,bottom=3cm]{geometry}
\usepackage{fancyhdr}
\usepackage{lastpage}

% Tabellen & Farben
\usepackage{booktabs}
\usepackage{longtable}
\usepackage{array}
\usepackage{xcolor}
\usepackage{colortbl}

% Boxes
\usepackage{tcolorbox}
\tcbuselibrary{skins,breakable}

% Mathematik für Pfeile
\usepackage{amsmath}

% Hyperlinks
\usepackage{hyperref}
\hypersetup{
    colorlinks=false,
    pdfborder={0 0 0}
}

% Farben definieren
\definecolor{AinaryGold}{HTML}{C8AA50}
\definecolor{AinaryDark}{HTML}{1F2937}
\definecolor{AinaryRed}{HTML}{DC2626}
\definecolor{AinaryGreen}{HTML}{059669}
\definecolor{AinaryBlue}{HTML}{2563EB}
\definecolor{AinaryOrange}{HTML}{D97706}
\definecolor{SurfaceAlt}{HTML}{F9FAFB}
\definecolor{SurfaceWarn}{HTML}{FFFBEB}
\definecolor{SurfaceGreen}{HTML}{E8F5E8}
\definecolor{SurfaceRed}{HTML}{FDE8EA}
\definecolor{SurfaceBlue}{HTML}{E7F1FF}
\definecolor{BorderLight}{HTML}{E5E7EB}

% tcolorbox Styles
\newtcolorbox{insightbox}[1][]{
    enhanced,
    colback=SurfaceBlue,
    colframe=AinaryBlue,
    boxrule=1pt,
    arc=3pt,
    left=8pt,
    right=8pt,
    top=6pt,
    bottom=6pt,
    breakable,
    #1
}

\newtcolorbox{warningbox}[1][]{
    enhanced,
    colback=SurfaceWarn,
    colframe=AinaryOrange,
    boxrule=1pt,
    arc=3pt,
    left=8pt,
    right=8pt,
    top=6pt,
    bottom=6pt,
    breakable,
    #1
}

\newtcolorbox{successbox}[1][]{
    enhanced,
    colback=SurfaceGreen,
    colframe=AinaryGreen,
    boxrule=1pt,
    arc=3pt,
    left=8pt,
    right=8pt,
    top=6pt,
    bottom=6pt,
    breakable,
    #1
}

\newtcolorbox{protocolbox}[1][]{
    enhanced,
    colback=SurfaceAlt,
    colframe=BorderLight,
    boxrule=1pt,
    arc=3pt,
    left=8pt,
    right=8pt,
    top=6pt,
    bottom=6pt,
    breakable,
    #1
}

% Header/Footer
\pagestyle{fancy}
\fancyhf{}
\fancyhead[L]{\small\color{AinaryGold}\textbf{CNC Planer Pro}\color{AinaryDark} | Nachbetrachtung}
\fancyhead[R]{\small\color{AinaryDark}Zeichnung 10028104.79}
\fancyfoot[L]{\tiny\color{AinaryDark}INTERN — Nachbetrachtung | CNC Planer Pro v0.20}
\fancyfoot[C]{\tiny\color{AinaryDark}Erstellt: 11.02.2026}
\fancyfoot[R]{\tiny\color{AinaryDark}Seite \thepage}
\renewcommand{\headrulewidth}{0.5pt}
\renewcommand{\footrulewidth}{0.5pt}

% Keine Einrückung, Absatzabstand
\setlength{\parindent}{0pt}
\setlength{\parskip}{6pt}

% Titel-Definitionen
\title{
    \color{AinaryGold}\textbf{Nachbetrachtung} \\[0.3cm]
    \large\color{AinaryDark}Kalkulationsvergleich: CNC Planer Pro vs. MBS Vorkalkulation \\[0.2cm]
    \normalsize Lagerungstraverse — Zeichnung 10028104.79 | KBA Koenig \& Bauer
}
\date{}
\author{}

\begin{document}

\maketitle
\thispagestyle{fancy}

\vspace{1cm}

% ============================================================
% 1. KOPFDATEN
% ============================================================
\section{Kopfdaten}

\begin{table}[h]
\centering
\small
\begin{tabular}{p{4cm}p{5cm}p{5cm}}
\toprule
\textbf{} & \textbf{MBS Vorkalkulation} & \textbf{CNC Planer Pro} \\
\midrule
Dokument & Nr. 74374 & Kalkulationsbericht v0.20 \\
System & b-logic & CNC Planer Pro \\
Datum & 09.02.2026 & 11.02.2026 \\
Bearbeiter & Björn Krügel & System (geprüft) \\
Kunde & KBA Koenig \& Bauer & KBA Koenig \& Bauer \\
Stückzahl Auftrag & 4 Stk. & 4 Stk. \\
Kalk.-Bezug & 1 Stück (Menge: 1) & 1 Stück \\
Methodik & Systemzeiten & REFA-basiert \\
Scope & 6 AG (Sägen, Fräsen 2×, Schweißen, Lackieren, Kontrolle) & 11 AG (Sägen, Fräsen 8×, Entgraten, QS) \\
\bottomrule
\end{tabular}
\end{table}

\begin{insightbox}
\textbf{Abgrenzung:} Dieser Vergleich fokussiert sich auf die CNC-Fräsvorgänge, da nur diese in beiden Systemen vergleichbar kalkuliert wurden. Beide Kalkulationen sind Planwerte — es liegen noch keine Ist-Zeiten vor. Die Gesamtherstellkosten und DB-Rechnungen basieren auf unterschiedlichen Scopes und Zuschlagslogiken und sind daher nur begrenzt direkt vergleichbar.
\end{insightbox}

% ============================================================
% 2. ARBEITSGÄNGE — NUR FRÄSEN
% ============================================================
\section{Arbeitsgänge — Nur Fräsen}

\subsection{MBS Vorkalkulation}

Die MBS Vorkalkulation sieht zwei Fräspositionen vor:

\begin{table}[h]
\centering
\small
\begin{tabular}{lrrrrrr}
\toprule
\textbf{Pos} & \textbf{Maschine} & \textbf{Tr [min]} & \textbf{Te [min]} & \textbf{HK Lohn} & \textbf{HK Maschine} & \textbf{HK Gesamt} \\
\midrule
20 & AXA (Fräszentrum) & 450 & 1.800 & 1.417,50\,€ & 1.050,00\,€ & 2.467,50\,€ \\
25 & FLP 8000 (Portalfräse) & 240 & 435 & 410,06\,€ & 900,00\,€ & 1.310,06\,€ \\
\midrule
\textbf{Summe Fräsen} & & \textbf{690} & \textbf{2.235} & \textbf{1.827,56\,€} & \textbf{1.950,00\,€} & \textbf{3.777,56\,€} \\
\bottomrule
\end{tabular}
\end{table}

\textit{Alle Werte pro Stück (Menge: 1).}

\subsection{CNC Planer Pro}

Der CNC Planer kalkuliert 8 Fräsarbeitsgänge auf einer CNC-Maschine:

\begin{table}[h]
\centering
\small
\begin{tabular}{llp{4.5cm}rrr}
\toprule
\textbf{AG} & \textbf{Werkzeug} & \textbf{Beschreibung} & \textbf{Zeit [min]} & \textbf{Satz [€/h]} & \textbf{Kosten} \\
\midrule
20 & Planfräser Ø80 & Planfräsen Unterseite & 55 & 70 & 64,17\,€ \\
30 & Bohrer Ø10-30 & Bohrungen Unterseite & 44 & 70 & 51,33\,€ \\
40 & Planfräser Ø80 & Planfräsen Oberseite & 52 & 70 & 60,67\,€ \\
50 & Fräser Ø20 & Taschen fräsen (4×) & 46 & 70 & 53,67\,€ \\
60 & Fräser Ø16 & Langlöcher (3×) & 24 & 70 & 28,00\,€ \\
70 & Schaftfräser Ø25 & Konturfräsen Außen & 28 & 70 & 32,67\,€ \\
80 & Fräser Ø25/Ø12 & Stirnseite 1 & 40 & 70 & 46,67\,€ \\
90 & Fräser Ø25/Ø12 & Stirnseite 2 & 57 & 70 & 66,50\,€ \\
\midrule
\multicolumn{3}{l}{\textbf{Summe Fräsen (AG 20-90)}} & \textbf{346} & & \textbf{403,68\,€} \\
\midrule
\multicolumn{3}{l}{Rüstzeiten anteilig (164 min ÷ 4 Stk.)} & \textbf{41} & 70 & \textbf{47,83\,€} \\
\midrule
\multicolumn{3}{l}{\textbf{Gesamt Fräsen inkl. Rüst}} & \textbf{387} & & \textbf{451,51\,€} \\
\bottomrule
\end{tabular}
\end{table}

\textit{Alle Werte pro Stück. Rüstzeiten: 4 Aufspannungen à 50+38+39+37 min = 164 min, anteilig auf 4 Stück verteilt.}

% ============================================================
% 3. FERTIGUNGSZEITEN — GEGENÜBERSTELLUNG
% ============================================================
\section{Fertigungszeiten — Gegenüberstellung}

\begin{table}[h]
\centering
\begin{tabular}{lrrr}
\toprule
\textbf{System} & \textbf{Rüst/Tr [min]} & \textbf{Bearbeitung/Te [min]} & \textbf{Gesamt [min]} \\
\midrule
\textbf{MBS — Nur Fräsen (Pos 20+25)} & & & \\
\quad Tr (Rüst) & 690 & — & — \\
\quad Te (Bearbeitung) & — & 2.235 & — \\
\quad Gesamt Fräsen & & & \textbf{2.925 (48,8h)} \\
\midrule
\textbf{CNC Planer — Nur Fräsen (AG 20-90)} & & & \\
\quad Bearbeitung & — & 346 & — \\
\quad Rüst anteilig (÷4) & 41 & — & — \\
\quad Gesamt Fräsen inkl. Rüst & & & \textbf{387 (6,5h)} \\
\midrule
\textbf{CNC Planer — Alle 11 AG} & & & \\
\quad Fertigung gesamt & — & 497 & — \\
\quad Rüst anteilig (÷4) & 41 & — & — \\
\quad Gesamt alle AG & & & \textbf{538 (9,0h)} \\
\bottomrule
\end{tabular}
\end{table}

\textbf{Faktoren:}
\begin{itemize}
    \item Fräsen: 2.925 min (MBS) / 387 min (CNC) = \textbf{7,6×}
    \item Gesamt: 2.925 min (MBS Fräsen) / 538 min (CNC alle AG) = \textbf{5,4×}
\end{itemize}

\textit{ACHTUNG: Der Gesamt-Faktor vergleicht unterschiedliche Scopes und ist nicht 1:1 interpretierbar.}

\begin{warningbox}
\textbf{Erklärung der Abweichung:}
\begin{itemize}
    \item \textbf{2 Maschinen vs. 1:} MBS plant AXA + FLP 8000, CNC Planer nur eine CNC-Maschine. Die FLP 8000 (675 min) fehlt im CNC-Scope komplett.
    \item \textbf{Systemzeiten vs. REFA:} MBS nutzt betriebliche Systemzeiten, die erfahrungsgemäß 40-60\% über REFA-Idealwerten liegen (Faktor 1,4-1,6).
    \item \textbf{Rüstzeiten:} MBS kalkuliert 690 min Rüst (Tr), CNC Planer 164 min für 4 Aufspannungen $\rightarrow$ 41 min pro Stück.
    \item \textbf{Scope-Unterschied:} CNC Planer hat 8 Fräs-AG auf einer Maschine, MBS verteilt auf zwei Maschinen mit höheren Zeiten.
\end{itemize}
\end{warningbox}

% ============================================================
% 4. MBS KALKULATION (ORIGINAL, VOLLSTÄNDIG)
% ============================================================
\section{MBS Kalkulation (Original, vollständig)}

Die MBS Vorkalkulation rechnet nach klassischer Zuschlagskalkulation:

\begin{table}[h]
\centering
\begin{tabular}{lr}
\toprule
\textbf{Position} & \textbf{Betrag} \\
\midrule
Material & 1.228,60\,€ \\
Maschinen & 2.056,67\,€ \\
Lohn & 1.504,75\,€ \\
\midrule
\textbf{(1) Grenzkosten} & \textbf{4.790,02\,€} \\
\midrule
+ Gemeinkostenzuschlag (GKZ) 12,28\% & 588,09\,€ \\
\midrule
\textbf{(2) Herstellkosten} & \textbf{5.378,11\,€} \\
\midrule
+ Vertrieb und Verwaltung (VuV) 20\% & 1.075,62\,€ \\
\midrule
\textbf{(3) Selbstkosten} & \textbf{6.453,73\,€} \\
\midrule
+ Gewinn 10\% & 645,37\,€ \\
\midrule
\textbf{Kalkulierter Verkaufspreis} & \textbf{7.099,10\,€} \\
\bottomrule
\end{tabular}
\end{table}

\textbf{Deckungsbeiträge:}
\begin{itemize}
    \item \textbf{DB I} (VK - Grenzkosten): 2.309,09\,€ (32,5\%)
    \item \textbf{DB II} (VK - Herstellkosten): 1.720,99\,€ (24,2\%)
    \item \textbf{Gewinn} (VK - Selbstkosten): 645,37\,€ (9,1\%)
\end{itemize}

% ============================================================
% 5. CNC PLANER KALKULATION (VOLLSTÄNDIG)
% ============================================================
\section{CNC Planer Kalkulation (vollständig)}

\subsection{Kostenaufbau pro Stück}

\begin{table}[h]
\centering
\begin{tabular}{lr}
\toprule
\textbf{Position} & \textbf{Betrag} \\
\midrule
\textbf{Material} & \\
\quad Rohteil GJS-700 (Beistellung) & 1.200,00\,€ \\
\quad MGK 5\% & 60,00\,€ \\
\quad \textit{Summe Material} & \textit{1.260,00\,€} \\
\midrule
\textbf{Fertigung} & \\
\quad Fertigung alle 11 AG (497 min) & 523,97\,€ \\
\quad Rüst anteilig (41 min) & 47,83\,€ \\
\quad \textit{Summe Fertigungskosten} & \textit{571,80\,€} \\
\midrule
\textbf{Gesamtkosten} & \textbf{1.831,80\,€} \\
\bottomrule
\end{tabular}
\end{table}

\subsection{Preisfindung CNC Planer}

Der CNC Planer schlägt mehrere Preispunkte vor:

\begin{table}[h]
\centering
\small
\begin{tabular}{lr}
\toprule
\textbf{Preispunkt} & \textbf{Betrag} \\
\midrule
Basispreis (System) & 2.383,58\,€ \\
Betriebsleiter-Korrektur & 2.680,00\,€ \\
\textbf{Empfehlung (korrigierte REFA-Zeiten + Risikopuffer)} & \textbf{2.750,00\,€} \\
REFA-Korrektur & 2.900,00\,€ \\
Premiumkunde & 2.950,00\,€ \\
Sicherheitsmarge & 3.100,00\,€ \\
\bottomrule
\end{tabular}
\end{table}

\textbf{Zuschläge zum Angebotspreis:}

\begin{table}[h]
\centering
\begin{tabular}{lr}
\toprule
\textbf{Position} & \textbf{Betrag} \\
\midrule
Empfehlung & 2.750,00\,€ \\
+ Kran-Handling & 178,00\,€ \\
+ NC-Programmierung (÷4) & 70,00\,€ \\
+ QS H7 (30 min) & 35,00\,€ \\
+ Großkunde 15\% & 79,00\,€ \\
\midrule
\textbf{Angebotspreis} & \textbf{3.112,00\,€} \\
\bottomrule
\end{tabular}
\end{table}

\subsection{DB-Betrachtung bei 3.112\,€}

\begin{itemize}
    \item \textbf{DB I} (Angebotspreis - Gesamtkosten): 3.112\,€ - 1.832\,€ = \textbf{1.280\,€ (41,1\%)}
\end{itemize}

\begin{insightbox}
\textbf{Hinweis:} Dieser DB I ist NICHT 1:1 mit dem MBS DB I vergleichbar. Die Kostenbasen und Zuschlagslogiken unterscheiden sich grundlegend:
\begin{itemize}
    \item MBS rechnet auf Grenzkosten mit GKZ, VuV und Gewinnzuschlag
    \item CNC Planer addiert Zuschläge (Kran, NC-Prog, QS, Großkunde) auf Empfehlung
    \item Verschiedener Fertigungs-Scope (11 AG vs. 6 AG)
\end{itemize}
\end{insightbox}

% ============================================================
% 6. PREISGEGENÜBERSTELLUNG
% ============================================================
\section{Preisgegenüberstellung}

\subsection*{6.1\quad Reines Fräsen (direkt vergleichbar)}

\renewcommand{\arraystretch}{1.35}
\begin{tabularx}{\textwidth}{@{}Xrrr@{}}
\toprule
\textbf{Position} & \textbf{MBS} & \textbf{CNC Planer} & \textbf{Faktor} \\
\midrule
Fräskosten pro Stück & 3.777,56\,€ & 451,51\,€ & 8,4$\times$ \\
\bottomrule
\end{tabularx}

{\scriptsize MBS: Pos\,20 AXA (2.467,50\,€) + Pos\,25 FLP (1.310,06\,€). CNC Planer: AG\,20--90 (403,68\,€) + Rüst anteilig 41\,min (47,83\,€).}

\vspace{8pt}

\subsection*{6.2\quad Brücke: CNC Planer Fräsen $\rightarrow$ Angebotspreis}

{\small Wie kommt der CNC Planer von 451,51\,€ Fräskosten auf 3.112\,€ Angebotspreis?}

\vspace{4pt}
\renewcommand{\arraystretch}{1.3}
\begin{tabularx}{\textwidth}{@{}Xrr@{}}
\toprule
\textbf{Position} & \textbf{pro Stück} & \textbf{kumuliert} \\
\midrule
Fräsen (AG\,20--90 + Rüst anteilig) & 451,51\,€ & 451,51\,€ \\
+ Material (GJS-700, Beistellung) & 1.260,00\,€ & 1.711,51\,€ \\
+ Andere AG (Sägen, Entgraten, QS) & 120,29\,€ & 1.831,80\,€ \\
\midrule
\textbf{= Gesamtkosten (Basis)} & & \textbf{1.831,80\,€} \\
\midrule
+ Risiko/REFA-Korrektur & 918,20\,€ & 2.750,00\,€ \\
\midrule
\textbf{= Empfehlung} & & \textbf{2.750,00\,€} \\
\midrule
+ Kran-Handling (1.415\,kg) & 178,00\,€ & 2.928,00\,€ \\
+ NC-Programmierung (÷ 4\,Stk) & 70,00\,€ & 2.998,00\,€ \\
+ QS H7-Passungen (30\,min) & 35,00\,€ & 3.033,00\,€ \\
+ Großkunde KBA (15\%) & 79,00\,€ & 3.112,00\,€ \\
\midrule
\rowcolor{SurfaceGreen}
\textbf{= Angebotspreis} & & \textbf{3.112,00\,€} \\
\bottomrule
\end{tabularx}

\vspace{8pt}

\subsection*{6.3\quad Vergleich Endpreise}

\renewcommand{\arraystretch}{1.35}
\begin{tabularx}{\textwidth}{@{}Xrrr@{}}
\toprule
& \textbf{MBS Fräsen} & \textbf{CNC Planer AP} & \textbf{Delta} \\
\midrule
\rowcolor{SurfaceGreen}
\textbf{Pro Stück} & \textbf{3.777,56\,€} & \textbf{3.112,00\,€} & \textbf{$-$17,6\%} \\
$\times$ 4 Stück & 15.110,24\,€ & 12.448,00\,€ & $-$17,6\% \\
\bottomrule
\end{tabularx}

{\scriptsize Vergleich: MBS Fräskosten (reiner Fertigungswert) vs. CNC Planer Angebotspreis (inkl. Material, aller AG, Zuschläge, Risiko).}

\vspace{6pt}

\begin{successbox}
\textbf{Interpretation:}

\textbf{Reines Fräsen:} CNC Planer kalkuliert 451\,€, MBS 3.778\,€ — Faktor 8,4$\times$. Die Differenz entsteht durch 2 Maschinen bei MBS, höhere Systemzeiten und die fehlende Portalfräse.

\textbf{Mit Zuschlägen:} Durch Risikokorrekturen (+918\,€), Kran, NC-Programmierung und Großkunde-Zuschlag nähert sich der CNC Planer auf \textbf{3.112\,€} an — nur noch \textbf{17,6\% unter den MBS-Fräskosten}.

\textbf{Offene Frage:} Liegt der CNC Planer zu niedrig oder MBS zu hoch? Nur Ist-Zeiten können das klären. Die Peer-Reviews im CNC Planer (REFA-Kalkulateur, Betriebsleiter) deuten darauf hin, dass einzelne Zeiten zu knapp sind (AG50 Taschen: 46 $\rightarrow$ 110\,min empfohlen).
\end{successbox}

% ============================================================
% 7. LIMITATIONEN
% ============================================================
\section{Limitationen}

\begin{table}[h]
\centering
\small
\begin{tabular}{p{1cm}p{6cm}p{7cm}}
\toprule
\textbf{ID} & \textbf{Limitation} & \textbf{Auswirkung \& Mitigation} \\
\midrule
L1 & \textbf{1-Maschinen-Scope:} FLP 8000 fehlt komplett & \textbf{Auswirkung:} 675 min (11,3h) fehlen. \newline \textbf{Mitigation:} Multi-Maschinen-Logik implementieren, Portalfräse als zweite Maschine konfigurierbar machen. \\
\midrule
L2 & \textbf{Schweißen nicht im Scope} & \textbf{Auswirkung:} Wenn geschweißt werden muss, fehlen Kosten. \newline \textbf{Mitigation:} Klären ob Schweißen nötig ist. Falls ja: Schweißmodul entwickeln. \\
\midrule
L3 & \textbf{Verfahrwege nicht geprüft} (Bauteil 2.095mm) & \textbf{Auswirkung:} Passt das Bauteil überhaupt auf die Maschine? \newline \textbf{Mitigation:} Verfahrweg-Check gegen Maschinendaten (X/Y/Z) implementieren. \\
\midrule
L4 & \textbf{REFA vs. Betriebszeiten} (Faktor 1,4-1,6) & \textbf{Auswirkung:} REFA-Idealzeiten $\neq$ Praxiszeiten. \newline \textbf{Mitigation:} Ist-Zeiten erfassen, Korrektur-Faktoren lernen und anwenden. \\
\midrule
L5 & \textbf{Keine Ist-Daten} & \textbf{Auswirkung:} Beide Kalkulationen sind Planwerte — unvalidiert. \newline \textbf{Mitigation:} Teil 1+2 fertigen, Ist-Zeiten erfassen, System kalibrieren. \\
\bottomrule
\end{tabular}
\end{table}

% ============================================================
% 8. NÄCHSTE SCHRITTE
% ============================================================
\section{Nächste Schritte}

\textbf{1. Scope klären}
\begin{itemize}
    \item Wird geschweißt? Falls ja: L2 (Schweißmodul) priorisieren
    \item Reicht eine Maschine oder braucht es zwei? Falls zwei: L1 (Multi-Maschinen-Logik) implementieren
\end{itemize}

\textbf{2. Ist-Zeiten erfassen}
\begin{itemize}
    \item Teil 1 und Teil 2 fertigen (bereits in Produktion?)
    \item Zeiten pro Arbeitsgang erfassen (MDE oder manuell)
    \item Vergleich: Plan vs. Ist für beide Systeme
\end{itemize}

\textbf{3. Limitationen L1-L3 implementieren}
\begin{itemize}
    \item Multi-Maschinen-Logik (L1)
    \item Verfahrweg-Check (L3)
    \item Schweißmodul (L2, falls benötigt)
\end{itemize}

\vspace{0.5cm}

\begin{protocolbox}
\textbf{Fazit}

CNC Planer Pro ist ein \textbf{lernendes System}. Der Prozess ist:

\begin{center}
\textbf{Kalkulieren $\rightarrow$ Vergleichen $\rightarrow$ Kalibrieren $\rightarrow$ Besser werden}
\end{center}

Diese Nachbetrachtung ist der erste Schritt. Mit Ist-Zeiten wird das System präziser. Mit jeder Kalkulation lernt es dazu.
\end{protocolbox}

\vspace{1cm}

\hrule
\vspace{0.3cm}
{\small\color{AinaryDark}\textit{Nachbetrachtung auf Basis CNC Planer Pro v0.20 und MBS Vorkalkulation Nr. 74374.}}

\end{document}
