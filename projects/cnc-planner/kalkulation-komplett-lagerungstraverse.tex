% ================================================================
% CNC Planer Pro — Vollständiger Kalkulationsbericht
% Lagerungstraverse KBA | 10028104.79
% ================================================================
\documentclass[10pt,a4paper]{article}
\usepackage[left=20mm,right=20mm,top=20mm,bottom=22mm]{geometry}
\usepackage{fontspec}
\usepackage{xcolor}
\usepackage{booktabs}
\usepackage{tabularx}
\usepackage{colortbl}
\usepackage{enumitem}
\usepackage{fancyhdr}
\usepackage{titlesec}
\usepackage{graphicx}
\usepackage{amssymb}
\usepackage{multicol}
\usepackage{tikz}
\usepackage{tcolorbox}
\tcbuselibrary{skins,breakable}

% Fonts
\setmainfont{Helvetica}
\setmonofont{Menlo}[Scale=0.85]

% Colors
\definecolor{AinaryGold}{HTML}{C8AA50}
\definecolor{AinaryDark}{HTML}{1F2937}
\definecolor{AinaryRed}{HTML}{DC2626}
\definecolor{AinaryGreen}{HTML}{059669}
\definecolor{AinaryBlue}{HTML}{2563EB}
\definecolor{AinaryOrange}{HTML}{D97706}
\definecolor{SurfaceAlt}{HTML}{F9FAFB}
\definecolor{SurfaceWarn}{HTML}{FFFBEB}
\definecolor{SurfaceGreen}{HTML}{E8F5E9}
\definecolor{SurfaceRed}{HTML}{FDE8EA}
\definecolor{SurfaceBlue}{HTML}{E7F1FF}
\definecolor{BorderLight}{HTML}{E5E7EB}

% Section formatting
\titleformat{\section}
  {\large\bfseries\color{AinaryDark}}
  {\thesection}{1em}{}[\titlerule]
\titleformat{\subsection}
  {\normalsize\bfseries\color{AinaryDark}}
  {\thesubsection}{1em}{}
\titlespacing{\section}{0pt}{14pt}{6pt}
\titlespacing{\subsection}{0pt}{10pt}{4pt}

% Header/Footer
\pagestyle{fancy}
\fancyhf{}
\renewcommand{\headrulewidth}{0pt}
\fancyhead[L]{\small\color{AinaryGold}\textbf{CNC Planer Pro} \color{gray}| Kalkulationsbericht}
\fancyhead[R]{\small\color{gray}Zeichnung 10028104.79}
\fancyfoot[L]{\scriptsize\color{gray} INTERN — Nicht für Kunden | CNC Planer Pro v0.20}
\fancyfoot[C]{\scriptsize\color{gray} Erstellt: 11.02.2026 | Methodik: REFA-Zeitermittlung, VDI 3321}
\fancyfoot[R]{\scriptsize\color{gray} Seite \thepage}

% tcolorbox styles
\newtcolorbox{insightbox}[2][]{%
  enhanced,breakable,
  colback=#2,colframe=#2,
  boxrule=0pt,left=8pt,right=8pt,top=6pt,bottom=6pt,
  borderline west={3pt}{0pt}{#1},
  fontupper=\small
}

\newtcolorbox{protocolbox}{%
  enhanced,
  colback=SurfaceAlt,colframe=BorderLight,
  boxrule=0.5pt,left=8pt,right=8pt,top=6pt,bottom=6pt,
  fontupper=\small
}

\setlength{\parindent}{0pt}
\setlength{\parskip}{4pt}

\begin{document}

% ================================================================
% TITLE
% ================================================================
\begin{center}
{\color{AinaryGold}\rule{\linewidth}{2pt}}\\[8pt]
{\LARGE\bfseries\color{AinaryDark} Kalkulationsbericht}\\[3pt]
{\large\color{gray} Lagerungstraverse — Zeichnung 10028104.79}\\[3pt]
{\small\color{AinaryGold} Anfrage 6000063225 | KBA Koenig \& Bauer}\\[6pt]
{\color{AinaryGold}\rule{\linewidth}{0.5pt}}
\end{center}

\vspace{-2pt}

% ================================================================
% 1. WERKSTÜCK (Project Data)
% ================================================================
\section*{1\quad Werkstück}

\begin{tabularx}{\textwidth}{@{}lXlX@{}}
\toprule
\textbf{Bauteil} & Lagerungstraverse & \textbf{Zeichnung} & 10028104.79 \\
\textbf{Anfrage} & 6000063225 & \textbf{Auftraggeber} & KBA Koenig \& Bauer \\
\textbf{Werkstoff} & GJS-700-2 (Sphäroguss) & \textbf{Rohteil} & Beistellung durch KBA \\
\textbf{Abmessungen} & 2095 $\times$ 500 $\times$ 190\,mm & \textbf{Rohgewicht} & ca. 1.415\,kg \\
\textbf{Stückzahl} & 4 Stück & \textbf{Material\-kosten} & EUR\,1.200 (fest, Beistellung) \\
\textbf{Maschine} & 3-Achs BAZ (Hermle C\,400) & \textbf{Aufspannungen} & 4 (Tischspannung) \\
\bottomrule
\end{tabularx}

\smallskip
{\scriptsize\color{gray} Quelle: Zeichnung 10028104.79, Anfrage 6000063225}

% ================================================================
% 2. PRÜFPROTOKOLL
% ================================================================
\section*{2\quad Prüfprotokoll — Fehlende Angaben klären}

{\small Vor der Kalkulation prüft CNC Planer Pro kritische Parameter. Fehlende oder unklare Angaben werden markiert und beeinflussen die Kalkulation direkt.}

\begin{protocolbox}
\renewcommand{\arraystretch}{1.3}
\begin{tabularx}{\textwidth}{@{}p{1.2cm}p{2.8cm}p{3cm}X@{}}
\toprule
\textbf{Status} & \textbf{Prüfpunkt} & \textbf{Antwort} & \textbf{KI-Bewertung} \\
\midrule
{\color{AinaryRed}\textbf{PFLICHT}} & Werkstoff & GJS-700-2 & GJS-700 ist ca. 3$\times$ teurer als S355, Bearbeitungszeit steigt nur um ca. 20\% (Guss = besser zerspanbar). Zeitfaktor 1.18$\times$. \\
{\color{AinaryRed}\textbf{PFLICHT}} & Rohteil-Herkunft & Beistellung KBA & Materialkosten entfallen $\rightarrow$ Festpreis EUR\,1.200. Risiko: Rohteil-Qualität nicht in eigener Hand. \\
{\color{AinaryRed}\textbf{PFLICHT}} & Rohkontur & Gussteil (kontur\-nah) & Guss-Aufmaß ca. 2--3\,mm. Gusshaut aufgehärtet $\rightarrow$ Vorschub in erster Zustellung um 30\% reduzieren. \\
\rowcolor{SurfaceWarn}
{\color{AinaryOrange}\textbf{EMPF.}} & Engste Toleranz & $\pm$0.1\,mm & Normal — kein Aufschlag nötig. H7-Passungen (12$\times$) sind separat in AG30 berücksichtigt. \\
\rowcolor{SurfaceWarn}
{\color{AinaryOrange}\textbf{EMPF.}} & Wiederholauftrag? & Einmalauftrag & Keine Vorrichtung wirtschaftlich. Staffelpreis für Nachbestellung empfehlen. \\
\bottomrule
\end{tabularx}
\end{protocolbox}

% ================================================================
% 3. FERTIGUNGSANWEISUNG
% ================================================================
\section*{3\quad Fertigungsanweisung}

\subsection*{Aufspannplan (4 Aufspannungen)}

{\small\renewcommand{\arraystretch}{1.25}
\begin{tabularx}{\textwidth}{@{}clcX@{}}
\toprule
\textbf{Nr.} & \textbf{Typ} & \textbf{Rüstzeit} & \textbf{Beschreibung} \\
\midrule
1 & Tischspannung & 50\,min & Unterseite — Pratzen auf Maschinentisch, WZ einwechseln, Nullpunkt \\
2 & Tischspannung & 38\,min & Oberseite — Wenden, Planfräsen, Taschen, Langlöcher \\
3 & Tischspannung & 39\,min & Stirnseite 1 — Kontrollmaß 1508$\pm$0.1 \\
4 & Tischspannung & 37\,min & Stirnseite 2 — Kontrollmaß 1400$\pm$0.1, Gesamtlänge 2095 \\
\midrule
& & \textbf{164\,min} & \textbf{Rüstzeit gesamt} \\
\bottomrule
\end{tabularx}}

\subsection*{Arbeitsgänge (11 Operationen)}

{\small\renewcommand{\arraystretch}{1.2}
\begin{tabularx}{\textwidth}{@{}r p{3.8cm} p{2.8cm} r r r@{}}
\toprule
\textbf{AG} & \textbf{Beschreibung} & \textbf{Werkzeug} & \textbf{Zeit} & \textbf{Satz} & \textbf{Kosten} \\
\midrule
10 & Sägen \& Vorbereitung & Bandsäge & 28\,min & 45\,€/h & 21,00\,€ \\
20 & Planfräsen Unterseite & T1 Ø80 Planfräser & 55\,min & 70\,€/h & 64,17\,€ \\
   & \multicolumn{5}{@{}l@{}}{\scriptsize\color{gray} n\,600 · vf\,750 · ap\,2{,}0 — Ref.-Fläche 2095$\times$500\,mm, ae=64\,mm, 8 Bahnen} \\
30 & Bohrungen Unterseite & T2 Ø16 + T3 Ø10 VHM & 44\,min & 70\,€/h & 51,33\,€ \\
   & \multicolumn{5}{@{}l@{}}{\scriptsize\color{gray} 8$\times$ Ø16 H7 (4\,min/Loch) + 4$\times$ Ø10 H7 (3\,min/Loch) — Zentrierung, Bohren, Reiben} \\
40 & Planfräsen Oberseite & T1 Ø80 Planfräser & 52\,min & 70\,€/h & 60,67\,€ \\
   & \multicolumn{5}{@{}l@{}}{\scriptsize\color{gray} Parallelfläche zu Unterseite, Sollmaß 190\,mm $\pm$0.05, Ra\,3{,}2} \\
50 & Taschen fräsen (4$\times$) & T4 Ø20 VHM & 46\,min & 70\,€/h & 53,67\,€ \\
   & \multicolumn{5}{@{}l@{}}{\scriptsize\color{gray} T1: 400$\times$280$\times$50 (18'), T2: 275$\times$180$\times$35 (12'), T3-4: 150$\times$100$\times$25 (2$\times$8')} \\
60 & Langlöcher (3$\times$) & T4 Ø20 VHM & 24\,min & 70\,€/h & 28,00\,€ \\
70 & Konturfräsen Außen & T5 Ø16 VHM & 28\,min & 70\,€/h & 32,67\,€ \\
   & \multicolumn{5}{@{}l@{}}{\scriptsize\color{gray} Gusshaut 2\,mm, Umfang 6\,m, 2 Zustellungen (Schrupp + Schlicht)} \\
80 & Stirnseite 1 & T1 Ø80 + T6 Ø12 & 40\,min & 70\,€/h & 46,67\,€ \\
   & \multicolumn{5}{@{}l@{}}{\scriptsize\color{gray} Kontrollmaß 1508$\pm$0.1, 6$\times$ Ø12 H8 t=80\,mm} \\
90 & Stirnseite 2 & T1 Ø80 + T6 Ø12 & 57\,min & 70\,€/h & 66,50\,€ \\
   & \multicolumn{5}{@{}l@{}}{\scriptsize\color{gray} Kontrollmaße 1400$\pm$0.1, 335$\pm$0.1, Gesamt 2095} \\
100 & Entgraten komplett & Manuell & 68\,min & 31\,€/h & 35,13\,€ \\
    & \multicolumn{5}{@{}l@{}}{\scriptsize\color{gray} Außen 6\,m (22'), Taschen 4$\times$2\,m (18'), Langlöcher 3$\times$ (12'), Bohrungen 24$\times$ (16')} \\
110 & QS \& Messprotokoll & 3D-Messarm / KMG & 55\,min & 70\,€/h & 64,17\,€ \\
    & \multicolumn{5}{@{}l@{}}{\scriptsize\color{gray} Kalibrierung (8'), 4$\times$ krit. Maße (18'), 8$\times$ Bohrungen (12'), Oberfläche (5'), Protokoll (12')} \\
\midrule
\multicolumn{3}{@{}l}{\textbf{Summe Fertigung}} & \textbf{497\,min} & & \textbf{523,97\,€} \\
\bottomrule
\end{tabularx}}

% ================================================================
% 4. KALKULATION (Zuschlagskalkulation)
% ================================================================
\section*{4\quad Kalkulation — Zuschlagskalkulation (REFA)}

\begin{minipage}[t]{0.55\textwidth}
\renewcommand{\arraystretch}{1.3}
\begin{tabular}{@{}p{5.5cm}r@{}}
\toprule
\textbf{Position} & \textbf{EUR/Stück} \\
\midrule
Materialkosten (Beistellung, fest) & 1.200,00 \\
\quad + MGK 5\% & 60,00 \\
\textbf{Materialkosten gesamt} & \textbf{1.260,00} \\
\midrule
Fertigungskosten (11 AG) & 523,97 \\
\quad + AV-Aufschlag 12\% & 62,88 \\
\textbf{Fertigungskosten gesamt} & \textbf{586,84} \\
\midrule
Rüstkosten (164\,min / 4\,Stk) & 47,83 \\
Werkzeugkosten & 24,47 \\
\midrule
\textbf{Herstellkosten} & \textbf{1.919,15} \\
\quad + VwGK 10\% & 191,91 \\
\quad + VtGK 5\% & 95,96 \\
\textbf{Selbstkosten} & \textbf{2.207,02} \\
\quad + Gewinn 8\% & 176,56 \\
\midrule
\textbf{Stückpreis (netto)} & \textbf{2.383,58} \\
\rowcolor{SurfaceAlt}
\textbf{Auftragswert (4 Stk, netto)} & \textbf{9.534,33} \\
\quad + MwSt 19\% & 1.811,52 \\
\rowcolor{SurfaceAlt}
\textbf{Auftragswert (brutto)} & \textbf{11.345,85} \\
\bottomrule
\end{tabular}
\end{minipage}%
\hfill
\begin{minipage}[t]{0.40\textwidth}
{\small\color{AinaryDark}\textbf{Parameter (Einstellungen)}}

\smallskip
{\footnotesize
\begin{tabular}{@{}lr@{}}
\toprule
\textbf{Stundensätze} & \\
\midrule
CNC 3-Achs (Lohn + Masch.) & 38 + 32 = 70\,€/h \\
Sägen & 35 + 10 = 45\,€/h \\
Entgraten & 28 + 3 = 31\,€/h \\
\midrule
\textbf{Zuschläge} & \\
\midrule
Verschnitt & 10\% \\
MGK & 5\% \\
AV-Aufschlag & 12\% \\
VwGK & 10\% \\
VtGK & 5\% \\
Gewinn & 8\% \\
MwSt & 19\% \\
\midrule
\textbf{Sondervereinbarungen} & \\
\midrule
MEK & 0\,€ \\
Transport & 0\,€ \\
Beistellung & 0\,€ \\
Sonstiges & 0\,€ \\
\bottomrule
\end{tabular}}

\smallskip
{\scriptsize\color{gray} Alle Parameter im CNC Planer Pro unter ,,Preise \& Sätze'' änderbar. Änderungen werden automatisch in die Kalkulation propagiert und im Änderungsprotokoll dokumentiert.}
\end{minipage}

\smallskip
{\scriptsize\color{gray} Methodik: Zuschlagskalkulation nach REFA. Zeiten: REFA-Richtwerte + VDI\,3321 Schnittdaten. Richtwert — Abgleich mit betrieblicher Nachkalkulation empfohlen.}

% ================================================================
% 5. PRICING INSIGHTS
% ================================================================
\section*{5\quad KI-Insights — Preisempfehlungen}

\begin{insightbox}[AinaryRed]{SurfaceRed}
\textbf{$\blacktriangle$ Schwerlast — Handling-Zuschlag prüfen}\\[2pt]
Bauteil wiegt ca. 1.415\,kg. Kran-/Staplernutzung für jede Aufspannung nötig. Transport, Verpackung und Versicherung als separate Positionen anbieten.\\[2pt]
{\footnotesize $\rightarrow$ Manuell anpassen unter: \textbf{Preise \& Sätze $\rightarrow$ Zuschläge} (geschätzte Auswirkung: \textbf{+EUR\,178/Stück})}\\
{\scriptsize\color{gray} Quelle: Erfahrungswerte Lohnfertigung}
\end{insightbox}

\begin{insightbox}[AinaryGreen]{SurfaceGreen}
\textbf{$\blacktriangle$ Großkunde — höherer Stundensatz möglich}\\[2pt]
KBA (Koenig \& Bauer) ist börsennotiert mit {>}\,EUR\,1\,Mrd. Umsatz. Solche Kunden sind Stundensätze von EUR\,85--95/h gewohnt (vs. EUR\,70/h kalkuliert). Empfehlung: +15--20\% auf Fertigungskosten.\\[2pt]
{\footnotesize $\rightarrow$ Manuell anpassen unter: \textbf{Preise \& Sätze $\rightarrow$ Stundensätze} (geschätzte Auswirkung: \textbf{+EUR\,79--105/Stück})}\\
{\scriptsize\color{gray} Quelle: Marktdaten Sachsen Q4/2025}
\end{insightbox}

\begin{insightbox}[AinaryBlue]{SurfaceBlue}
\textbf{$\circlearrowleft$ Lange Bearbeitungszeit — Maschinenbelegung beachten}\\[2pt]
8,3\,h Bearbeitung pro Stück = mehr als eine Schicht. Prüfen: Maschinenauslastung, ggf. Nachtschicht-Zuschlag kalkulieren. Bei 4 Stück: ca. 5 Maschinentage (inkl. Rüstung).\\
{\scriptsize\color{gray} Quelle: REFA-Richtwerte 2024}
\end{insightbox}

\begin{insightbox}[AinaryOrange]{SurfaceWarn}
\textbf{H7-Passungen (12$\times$) — QS-Aufwand einplanen}\\[2pt]
Jede H7-Bohrung muss mit Lehrring/Innenmessschraube geprüft werden. Zeitaufwand: mind. 30\,min/Stück zusätzlich zur AG110. Messprotokoll archivieren für KBA-Qualitätsanforderung.\\
{\scriptsize\color{gray} Quelle: REFA-Richtwerte 2024, VDI 3321}
\end{insightbox}

% ================================================================
% 6. RISIKOANALYSE + PEER REVIEW
% ================================================================
\section*{6\quad Risikoanalyse}

\subsection*{{\color{AinaryRed}Hohes Risiko ({>}20\% Abweichung)}}

\begin{itemize}[leftmargin=*,itemsep=4pt]
  \item \textbf{R1: Bearbeitungszeiten ($\pm$30\%)} — KI-Schätzung basiert auf REFA-Richtwerten, nicht Ist-Zeiten. Korridor: 350--650\,min. Besonders kritisch: AG50 Taschen fräsen (kalkuliert 46\,min, REFA-Korrekturbedarf: 90--120\,min).\\
  {\small\color{gray}Kosteneffekt: $\pm$EUR\,176/Stk}
  
  \item \textbf{R2: GJS-700 Zerspanbarkeit ($\pm$20\%)} — Gusshaut, Lunker, Härteschwankungen. Worst Case: Lunker in Bohrungsbereich $\rightarrow$ Ausschuss.\\
  {\small\color{gray}Kosteneffekt: $\pm$EUR\,200/Stk}
  
  \item \textbf{R3: Aufspannung 2m-Großteil ($\pm$25\%)} — Durchbiegung, Gussspannungen, Kran-Handling. REFA-Korrektur: 55--75\,min/Aufspannung (vs. 41\,min kalkuliert).\\
  {\small\color{gray}Kosteneffekt: $\pm$EUR\,50/Stk}
\end{itemize}

\subsection*{{\color{AinaryOrange}Mittleres Risiko (10--20\%)}}

\begin{itemize}[leftmargin=*,itemsep=4pt]
  \item \textbf{R4: Stundensätze nicht kalibriert} — EUR\,70/h ist Richtwert Sachsen. Hermle C\,400 (Premiummaschine) rechtfertigt EUR\,75--85/h. {\small\color{gray}$\pm$EUR\,350/Stk}
  \item \textbf{R5: Fehlende Positionen} — NC-Programmierung (3--4\,h), Waschen, Konservierung, Verpackung Sondergröße. {\small\color{gray}$+$EUR\,365--453/Stk}
\end{itemize}

\subsection*{Unabhängige Peer Reviews}

\begin{protocolbox}
\textbf{REFA-Kalkulateur (20\,J. Erfahrung):} ,,Kalkulation handwerklich sauber, aber zu eng. 6 von 11 AGs zu knapp. Besonders Taschen (AG50: 46$\rightarrow$110\,min) und Bohrungen (AG30: 44$\rightarrow$65\,min). Rüstzeiten um 60\% unterschätzt. \textbf{Empfehlung: EUR\,2.800--2.900/Stk.}''

\smallskip
\textbf{Betriebsleiter CNC-Lohnfertigung:} ,,EUR\,2.384 ist Brot-Preis, kein Butter-Preis. NC-Programmierung fehlt komplett (3--4\,h = EUR\,280), QS-Aufwand für 12$\times$ H7 fehlt, kein Risikozuschlag Erstfertigung. Ein Ausschussteil und der Gewinn des Auftrags ist weg. \textbf{Empfehlung: EUR\,2.650--2.700/Stk.}''
\end{protocolbox}

\subsection*{Sensitivitätsanalyse}

\renewcommand{\arraystretch}{1.25}
\begin{tabularx}{\textwidth}{@{}Xrrp{5cm}@{}}
\toprule
\textbf{Szenario} & \textbf{Stückpreis} & \textbf{$\Delta$} & \textbf{Anmerkung} \\
\midrule
Optimistisch & EUR\,1.950 & $-$18\% & Alles glatt, erfahrene Mannschaft \\
\rowcolor{SurfaceAlt}
Basiskalkulation (CNC Planer) & EUR\,2.384 & Basis & REFA-Richtwerte, ohne Korrekturen \\
Betriebsleiter-Korrektur & EUR\,2.680 & $+$12\% & + fehlende Positionen \\
\rowcolor{SurfaceGreen}
\textbf{Empfehlung} & \textbf{EUR\,2.750} & $+$15\% & \textbf{Korrigierte Zeiten + Risikopuffer} \\
REFA-Korrektur & EUR\,2.900 & $+$22\% & Konservative Zeitkorrektur \\
Premiumkunde (KBA-Satz) & EUR\,2.950 & $+$24\% & Mit EUR\,85/h Stundensatz \\
Worst Case (Ausschuss) & EUR\,3.500 & $+$47\% & 1 Ausschussteil bei 4er-Los \\
\bottomrule
\end{tabularx}

% ================================================================
% 7. ANGEBOTSOPTIONEN
% ================================================================
\section*{7\quad Angebotsoptionen}

\renewcommand{\arraystretch}{1.35}
\begin{tabularx}{\textwidth}{@{}p{3.5cm}rrX@{}}
\toprule
\textbf{Variante} & \textbf{Stückpreis} & \textbf{4 Stück} & \textbf{Empfehlung} \\
\midrule
Basispreis & EUR\,2.384 & EUR\,9.534 & Nur bei Bestandskunde mit Folgeaufträgen \\
\rowcolor{SurfaceGreen}
\textbf{Empfehlung} & \textbf{EUR\,2.750} & \textbf{EUR\,11.000} & Korrigierte Zeiten, fehlende Positionen eingepreist \\
Premiumkunde & EUR\,2.950 & EUR\,11.800 & EUR\,85/h CNC-Satz (Großkunde KBA) \\
Sicherheitsmarge & EUR\,3.100 & EUR\,12.400 & Erstauftrag + Risikopuffer für GJS-700 \\
\bottomrule
\end{tabularx}

\smallskip
{\footnotesize\textbf{Empfehlung:} Angebot bei \textbf{EUR\,2.750/Stk} (EUR\,11.000 für 4 Stück). \\
Klausel: ,,Preis gilt nach Erstteileprüfung bei Gussqualität wie Muster. NC-Programmierung einmalig EUR\,280.'' \\
Staffelangebot: Bei 10+ Stk ca. EUR\,2.450/Stk (Programmierung amortisiert, Lernkurve).}

% ================================================================
% 8. KRITISCHE PRÜFUNG
% ================================================================
\section*{8\quad Kritische Prüfung — Offene Punkte}

\begin{itemize}[leftmargin=*,itemsep=3pt,label=$\square$]
  \item NC-Programmierung (Erstauftrag!) — 3--4\,h = EUR\,210--280, nicht im Stückpreis
  \item Spannungsarmglühen vor Bearbeitung? (Kosten ca. EUR\,150/Teil, bei Guss empfohlen)
  \item Konservierung + Verpackung Sondergröße (2\,m+)?
  \item Ausschussrisiko — bei 4\,Stk kein Ersatzteil. 1 Ausschuss = +25\% Kosten
  \item Rohteil-Qualität: Lunker, Härteunterschiede? Eingangsprüfung (Ultraschall) empfehlen
  \item Messprotokoll nach KBA-Anforderung dokumentieren (QS-Standard Maschinenbau)
  \item Ausschussregelung mit KBA vereinbaren — wer trägt Risiko bei Materialfehlern?
\end{itemize}

% ================================================================
% 9. NÄCHSTE SCHRITTE
% ================================================================
\section*{9\quad Empfohlene nächste Schritte}

\begin{enumerate}[leftmargin=*,itemsep=4pt]
  \item \textbf{Betriebsspezifische Stundensätze und Zuschlagssätze erfragen} \\
  $\rightarrow$ in CNC Planer Pro eintragen $\rightarrow$ sofortige Neuberechnung
  
  \item \textbf{Rohteil-Zustand klären} \\
  Aufmaße, Gusshaut, spannungsarmgeglüht? Materialzertifikat anfordern.
  
  \item \textbf{NC-Programmierzeit als separate Position} \\
  Erstauftrag: EUR\,280 einmalig, entfällt bei Folgeaufträgen.
  
  \item \textbf{Ist-Zeiten der ersten 2 Teile protokollieren} \\
  AG für AG messen $\rightarrow$ Nachkalkulation $\rightarrow$ Abweichungen analysieren
  
  \item \textbf{Korrektur-Faktoren für GJS-700 ableiten} \\
  $\rightarrow$ Material-Datenbank im CNC Planer Pro kalibrieren $\rightarrow$ Folgekalkulation verbessern
\end{enumerate}

\vspace{8pt}
\begin{center}
{\color{AinaryGold}\rule{0.6\linewidth}{0.5pt}}\\[4pt]
{\small\color{gray} Vorkalkulation auf Basis von REFA-Zeitermittlung und VDI\,3321 Schnittdaten.\\
Richtwerte — Abgleich mit betrieblicher Nachkalkulation empfohlen.\\
Für Präzisionsangebote bei Serienproduktion empfehlen wir eine manuelle Nachkalkulation.}\\[4pt]
{\small\color{AinaryGold}\textbf{CNC Planer Pro} — \color{gray}Kalkulationszeit: ca. 3 Minuten (vs. 2--4 Stunden manuell)}
\end{center}

\end{document}
