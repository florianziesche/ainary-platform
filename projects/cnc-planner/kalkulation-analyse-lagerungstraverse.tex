\documentclass[11pt,a4paper]{article}

% Packages
\usepackage{fontspec}
\usepackage[ngerman]{babel}
\usepackage[margin=2.5cm]{geometry}
\usepackage{xcolor}
\usepackage{graphicx}
\usepackage{tabularx}
\usepackage{booktabs}
\usepackage{fancyhdr}
\usepackage{enumitem}
\usepackage{titlesec}
\usepackage{calc}
\usepackage{amssymb}

% Fonts
\setmainfont{Helvetica}
\setsansfont{Helvetica}

% Colors - Ainary CI
\definecolor{AinaryDark}{HTML}{1F2937}
\definecolor{AinaryGold}{HTML}{C8AA50}
\definecolor{AinaryRed}{HTML}{DC2626}

% Global text color
\color{AinaryDark}

% Section formatting
\titleformat{\section}
  {\Large\bfseries\color{AinaryGold}}
  {\thesection}{1em}{}[\titlerule]
\titleformat{\subsection}
  {\large\bfseries\color{AinaryDark}}
  {\thesubsection}{1em}{}

% Header & Footer
\pagestyle{fancy}
\fancyhf{}
\renewcommand{\headrulewidth}{0pt}
\fancyfoot[C]{\small\color{gray}INTERN — Nicht für Kunden \textbar\ CNC Planer Pro \textbar\ 11.02.2026}

% Remove page numbers from first page
\fancypagestyle{plain}{
  \fancyhf{}
  \fancyfoot[C]{\small\color{gray}INTERN — Nicht für Kunden \textbar\ CNC Planer Pro \textbar\ 11.02.2026}
  \renewcommand{\headrulewidth}{0pt}
}

% Document
\begin{document}

% Title
\begin{center}
{\Huge\bfseries\color{AinaryGold} Kalkulationsanalyse}\\[0.3cm]
{\LARGE\color{AinaryDark} Lagerungstraverse}\\[0.5cm]
{\large\color{gray} Detaillierte Kostenkalkulation \& Risikoanalyse}
\end{center}

\vspace{1cm}

% Section 1: Projektübersicht
\section*{Projektübersicht}

\begin{itemize}[leftmargin=*,itemsep=0.3em]
  \item \textbf{Bauteil:} Lagerungstraverse, Zeichnung 10028104.79
  \item \textbf{Kunde:} KBA Koenig \& Bauer
  \item \textbf{Werkstoff:} GJS-700-2 (Sphäroguss)
  \item \textbf{Abmessungen:} 2095 × 500 × 190 mm, ca. 1.415 kg
  \item \textbf{Material:} Beistellung KBA, Wert €1.200
  \item \textbf{Stückzahl:} 4 Stück
  \item \textbf{Maschine:} 3-Achs-BAZ (Hermle C 400)
\end{itemize}

\vspace{0.5cm}

% Section 2: Kalkulationsergebnis
\section*{Kalkulationsergebnis}

\subsection*{Arbeitsgänge}

\begin{table}[h]
\small
\begin{tabularx}{\textwidth}{lXlrrr}
\toprule
\textbf{AG} & \textbf{Beschreibung} & \textbf{Werkzeug} & \textbf{Zeit} & \textbf{Satz} & \textbf{Kosten} \\
\midrule
10 & Sägen \& Vorbereitung & Bandsäge & 28 min & €45/h & €21,00 \\
20 & Planfräsen Unterseite & Ø80 Planfräser & 55 min & €70/h & €64,17 \\
30 & Bohrungen Unterseite & Ø16+Ø10 VHM & 44 min & €70/h & €51,33 \\
40 & Planfräsen Oberseite & Ø80 Planfräser & 52 min & €70/h & €60,67 \\
50 & Taschen fräsen (4×) & Ø20 VHM & 46 min & €70/h & €53,67 \\
60 & Langlöcher (3×) & Ø20 VHM & 24 min & €70/h & €28,00 \\
70 & Konturfräsen Außen & Ø16 VHM & 28 min & €70/h & €32,67 \\
80 & Stirnseite 1 & Ø80+Ø12 VHM & 40 min & €70/h & €46,67 \\
90 & Stirnseite 2 & Ø80+Ø12 VHM & 57 min & €70/h & €66,50 \\
100 & Entgraten & Manuell & 68 min & €31/h & €35,13 \\
110 & QS + Messprotokoll & 3D-Messarm & 55 min & €70/h & €64,17 \\
\midrule
\textbf{Σ} & & & \textbf{497 min} & & \textbf{€523,97} \\
\bottomrule
\end{tabularx}
\end{table}

\subsection*{Zuschlagskalkulation}

\begin{table}[h]
\begin{tabularx}{\textwidth}{Xr}
\toprule
\textbf{Position} & \textbf{Betrag} \\
\midrule
Material & €1.200,00 \\
\quad + MGK 5\% & €60,00 \\
\textbf{Materialkosten} & \textbf{€1.260,00} \\[0.3em]
Fertigung & €523,97 \\
\quad + AV 12\% & €62,87 \\
\textbf{Fertigungskosten} & \textbf{€586,84} \\[0.3em]
Rüstung (164 min / 4 Stk) & €47,83 \\
Werkzeug & €24,47 \\
\midrule
\textbf{\color{AinaryGold}Herstellkosten} & \textbf{\color{AinaryGold}€1.919,15} \\[0.3em]
+ VwGK 10\% & €191,91 \\
+ VtGK 5\% & €95,96 \\
\midrule
\textbf{Selbstkosten} & \textbf{€2.207,02} \\
+ Gewinn 8\% & €176,56 \\
\midrule
\textbf{\color{AinaryGold}Stückpreis} & \textbf{\color{AinaryGold}€2.383,58} \\
\textbf{\color{AinaryGold}Auftragswert (4 Stk)} & \textbf{\color{AinaryGold}€9.534,33} \\
\bottomrule
\end{tabularx}
\end{table}

\newpage

% Section 3: Risikoanalyse
\section*{Risikoanalyse}

\subsection*{\color{AinaryRed}Hohes Risiko (>20\%)}

\begin{itemize}[leftmargin=*,itemsep=0.5em]
  \item \textbf{R1: Bearbeitungszeiten ±30\%} \\
  KI-Schätzung, keine Ist-Daten. Korridor: 350--650 min. \\
  Kosteneffekt: \textcolor{AinaryRed}{\textbf{±€176/Stk}}
  
  \item \textbf{R2: GJS-700 Zerspanbarkeit ±20\%} \\
  Gusshaut, Lunker, Hartflecken können Werkzeugstandzeit drastisch reduzieren. \\
  Kosteneffekt: \textcolor{AinaryRed}{\textbf{±€200/Stk}}
  
  \item \textbf{R3: Aufspannung Großteil ±25\%} \\
  Durchbiegung und Gussspannungen bei 2m-Teil. Toleranzprobleme möglich. \\
  Kosteneffekt: \textcolor{AinaryRed}{\textbf{±€50/Stk}}
\end{itemize}

\subsection*{\color{AinaryGold}Mittleres Risiko (10--20\%)}

\begin{itemize}[leftmargin=*,itemsep=0.5em]
  \item \textbf{R4: Stundensätze nicht kalibriert} \\
  Richtwerte verwendet, nicht MBS-spezifisch. \\
  Kosteneffekt: \textcolor{AinaryGold}{\textbf{±€350/Stk}}
  
  \item \textbf{R5: Zuschlagssätze geschätzt} \\
  MGK, VwGK, VtGK basieren auf Branchendurchschnitt. \\
  Kosteneffekt: \textcolor{AinaryGold}{\textbf{±€200/Stk}}
\end{itemize}

\subsection*{Sensitivitätsanalyse}

\begin{table}[h]
\begin{tabularx}{\textwidth}{Xrr}
\toprule
\textbf{Szenario} & \textbf{Stückpreis} & \textbf{Abweichung} \\
\midrule
Optimistisch & €1.950/Stk & -18\% \\
\textbf{Berechnet (Basis)} & \textbf{€2.384/Stk} & \textbf{0\%} \\
Pessimistisch & €3.100/Stk & +30\% \\
\textcolor{AinaryRed}{\textbf{Worst Case}} & \textcolor{AinaryRed}{\textbf{€3.500/Stk}} & \textcolor{AinaryRed}{\textbf{+47\%}} \\
\bottomrule
\end{tabularx}
\end{table}

\vspace{0.5cm}

% Section 4: KI-Insights
\section*{KI-Insights (aus CNC Planer Pro)}

\begin{itemize}[leftmargin=*,itemsep=0.8em]
  \item {\color{AinaryRed}\textbf{$\blacktriangle$ Schwerlast — Handling-Zuschlag prüfen}} \\
  Bauteil wiegt ca. 1.415\,kg. Kran-/Staplernutzung für jede Aufspannung nötig. Transport, Verpackung und Versicherung als separate Positionen anbieten. \\
  \textit{\footnotesize Geschätzte Auswirkung: +EUR\,178/Stück} \\
  {\scriptsize\color{gray} Quelle: Erfahrungswerte Lohnfertigung}
  
  \item {\color{AinaryGold}\textbf{$\blacktriangle$ Großkunde — höherer Stundensatz möglich}} \\
  KBA (Koenig \& Bauer) ist börsennotiert mit {>}\,EUR\,1\,Mrd. Umsatz. Solche Kunden sind Stundensätze von EUR\,85--95/h gewohnt (vs. EUR\,70/h kalkuliert). Empfehlung: +15--20\% auf Fertigungskosten. \\
  {\scriptsize\color{gray} Quelle: Marktdaten Sachsen Q4/2025}
  
  \item \textbf{Beistellmaterial:} Kein Materialrisiko für MBS, aber Eingangsqualität prüfen (Gussfehler!). Klare Ausschussregelung mit KBA vereinbaren. \\
  {\scriptsize\color{gray} Quelle: Erfahrungswerte Lohnfertigung}
  
  \item \textbf{H7-Passungen (12×):} Reiben ist Pflicht. Bei GJS-700 zuverlässig, aber Prüfung jeder Bohrung mit Lehrring nötig. QS-Aufwand mind. 30\,min/Stk. \\
  {\scriptsize\color{gray} Quelle: REFA-Richtwerte 2024, VDI 3321}
  
  \item \textbf{4 Aufspannungen:} Hoher Rüstanteil (164--260\,min). Bei Wiederholauftrag: Vorrichtungsbau prüfen $\rightarrow$ Rüstzeit -40\%. \\
  {\scriptsize\color{gray} Quelle: REFA-Richtwerte 2024}
  
  \item \textbf{GJS-700 timeFactor 1.18:} Moderate Erschwerniszulage. Vergleich: Edelstahl wäre 1.6--2.0. Werkzeugstandzeiten bei Gusshaut beachten. \\
  {\scriptsize\color{gray} Quelle: VDI 3321}
\end{itemize}

\vspace{0.5cm}

% Section 5: Angebotsoptionen
\section*{Angebotsoptionen}

\begin{tabular}{@{}p{3.5cm}rrp{5cm}@{}}
\toprule
\textbf{Variante} & \textbf{Stückpreis} & \textbf{4 Stück} & \textbf{Anmerkung} \\
\midrule
Basiskalkulation & EUR\,2.384 & EUR\,9.534 & KI-Richtwert, ohne Korrekturen \\
\rowcolor{gray!10}
\textbf{Empfehlung} & \textbf{EUR\,2.750} & \textbf{EUR\,11.000} & Korrigierte Zeiten + fehlende Pos. \\
Premiumkunde (KBA) & EUR\,2.950 & EUR\,11.800 & Mit €85/h Stundensatz \\
Sicherheitsmarge & EUR\,3.100 & EUR\,12.400 & Erstauftrag + Risikopuffer \\
\bottomrule
\end{tabular}

\smallskip
{\footnotesize Empfehlung: Angebot bei \textbf{EUR\,2.750/Stk} (EUR\,11.000 für 4 Stück) mit Klausel: \\
,,Preis gilt nach Erstteileprüfung bei Gussqualität wie Muster. NC-Programmierung einmalig EUR\,280.''}

\vspace{0.5cm}

% Section 6: Kritische Prüfung
\section*{Kritische Prüfung — Was fehlen könnte}

\begin{itemize}[leftmargin=*,itemsep=0.3em,label=$\square$]
  \item Spannungsarmglühen vor Bearbeitung? (Kosten ca. €150/Teil, bei Gussteil empfohlen)
  \item Oberflächenschutz/Konservierung nach Bearbeitung?
  \item Verpackung und Transport (Großteil, 2m+)?
  \item Programmierzeit für NC-Code (Erstauftrag!) — oft 2--4h zusätzlich
  \item Messprotokoll-Dokumentation nach Kundenanforderung?
  \item Ausschussrisiko — bei 4 Stk kein Ersatzteil. Wenn 1 Teil Ausschuss $\rightarrow$ +25\% Kosten
  \item Wartezeiten/Lieferzeit Material (Gussteil-Lieferzeit oft 6--8 Wochen)
\end{itemize}

\vspace{0.5cm}

% Section 6: Empfohlene nächste Schritte
\section*{Empfohlene nächste Schritte}

\begin{enumerate}[leftmargin=*,itemsep=0.5em]
  \item \textbf{Betriebsspezifische Stundensätze und Zuschlagssätze erfragen} \\
  $\rightarrow$ in CNC Planer eintragen für präzisere Kalkulation
  
  \item \textbf{Rohteil-Zustand klären} \\
  Aufmaße, Gusshaut, spannungsarmgeglüht? Eingangsqualität definieren
  
  \item \textbf{NC-Programmierzeit als separate Position kalkulieren} \\
  Erstauftrag: 2--4h Programmierzeit zusätzlich
  
  \item \textbf{Ist-Zeiten der ersten 2 Teile protokollieren} \\
  $\rightarrow$ Nachkalkulation durchführen, Abweichungen analysieren
  
  \item \textbf{Korrektur-Faktoren für GJS-700 ableiten} \\
  $\rightarrow$ Folgekalkulation verbessern, Material-Datenbank aufbauen
\end{enumerate}

\end{document}
