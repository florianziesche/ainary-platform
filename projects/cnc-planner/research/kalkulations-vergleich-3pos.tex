\documentclass[a4paper,11pt]{article}

% Fonts
\usepackage{fontspec}
\setmainfont{Helvetica Neue}[
  BoldFont=Helvetica Neue Bold,
  ItalicFont=Helvetica Neue Italic
]

% Layout
\usepackage[top=28mm, bottom=34mm, left=28mm, right=28mm]{geometry}
\usepackage{parskip}
\setlength{\parskip}{8pt}
\setlength{\parindent}{0pt}

% Colors
\usepackage{xcolor}
\definecolor{primary}{HTML}{2563EB}
\definecolor{darkbg}{HTML}{0A0F1E}
\definecolor{darkblue}{HTML}{1E3A5F}
\definecolor{bodytext}{HTML}{374151}
\definecolor{heading}{HTML}{111827}
\definecolor{subtitle}{HTML}{64748B}
\definecolor{lightgray}{HTML}{F8F9FA}
\definecolor{border}{HTML}{E5E7EB}
\definecolor{accent}{HTML}{93C5FD}
\definecolor{lightondark}{HTML}{D1D5DB}
\definecolor{darkred}{HTML}{B91C1C}
\definecolor{darkgreen}{HTML}{15803D}
\definecolor{darkyellow}{HTML}{92400E}
\definecolor{lightblue}{HTML}{F0F4FF}
\definecolor{lightred}{HTML}{FEF2F2}
\definecolor{lightgreen}{HTML}{F0FDF4}
\definecolor{lightyellow}{HTML}{FFFBEB}

% Headers & Footers
\usepackage{fancyhdr}
\pagestyle{fancy}
\fancyhf{}
\renewcommand{\headrulewidth}{0pt}
\renewcommand{\footrulewidth}{0.4pt}
\fancyfoot[L]{\footnotesize\color{subtitle}\textls[50]{CNC PLANER PRO · KALKULATIONS-VERGLEICH}}
\fancyfoot[R]{\footnotesize\color{subtitle}\thepage}

% Headings
\usepackage{titlesec}
\usepackage{needspace}
\titleformat{\section}{\needspace{6\baselineskip}\fontsize{22}{26}\selectfont\bfseries\color{heading}}{}{0em}{}[\vspace{-2pt}]
\titleformat{\subsection}{\needspace{4\baselineskip}\fontsize{13}{17}\selectfont\bfseries\color{heading}}{}{0em}{}
\titlespacing*{\section}{0pt}{0pt}{4pt}
\titlespacing*{\subsection}{0pt}{14pt}{6pt}

% Tables
\usepackage{tabularx}
\usepackage{booktabs}
\usepackage{colortbl}

% Lists
\usepackage{enumitem}
\setlist[itemize]{leftmargin=1.2em, itemsep=2pt, parsep=0pt, topsep=4pt}

% Links
\usepackage{hyperref}
\hypersetup{colorlinks=true, linkcolor=primary, urlcolor=primary}

% Drawing
\usepackage{tikz}
\usetikzlibrary{calc,positioning}

% Boxes
\usepackage{tcolorbox}
\tcbuselibrary{skins,breakable}

% Misc
\usepackage{microtype}
\usepackage{float}
\usepackage{amssymb}

% Language
\usepackage{polyglossia}
\setdefaultlanguage{german}
\tolerance=2000
\emergencystretch=15pt
\hbadness=2000
\hyphenpenalty=50
\widowpenalty=10000
\clubpenalty=10000

% Custom
\newcommand{\ssubtitle}[1]{%
  \par\textcolor{subtitle}{\fontsize{11}{15}\selectfont #1}%
  \par\vspace{2pt}\textcolor{border}{\rule{\linewidth}{0.4pt}}\vspace{10pt}%
}

\newtcolorbox{highlightbox}{
  colback=lightblue, colframe=primary,
  leftrule=3pt, rightrule=0pt, toprule=0pt, bottomrule=0pt,
  arc=0pt, outer arc=4pt,
  boxsep=3pt, left=10pt, right=10pt, top=6pt, bottom=6pt,
  fontupper=\fontsize{10}{14}\selectfont\color{darkblue}
}

\newtcolorbox{darkhighlight}{
  colback=darkbg, colframe=primary,
  leftrule=3pt, rightrule=0pt, toprule=0pt, bottomrule=0pt,
  arc=0pt, outer arc=4pt,
  boxsep=3pt, left=10pt, right=10pt, top=6pt, bottom=6pt,
  fontupper=\fontsize{10}{14}\selectfont\color{white}
}

\newtcolorbox{warnbox}{
  colback=lightyellow, colframe=darkyellow,
  leftrule=3pt, rightrule=0pt, toprule=0pt, bottomrule=0pt,
  arc=0pt, outer arc=4pt,
  boxsep=3pt, left=10pt, right=10pt, top=6pt, bottom=6pt,
  fontupper=\fontsize{10}{14}\selectfont\color{darkyellow}
}

\newtcolorbox{redbox}{
  colback=lightred, colframe=darkred,
  leftrule=3pt, rightrule=0pt, toprule=0pt, bottomrule=0pt,
  arc=0pt, outer arc=4pt,
  boxsep=3pt, left=10pt, right=10pt, top=6pt, bottom=6pt,
  fontupper=\fontsize{10}{14}\selectfont\color{darkred}
}

\newtcolorbox{greenbox}{
  colback=lightgreen, colframe=darkgreen,
  leftrule=3pt, rightrule=0pt, toprule=0pt, bottomrule=0pt,
  arc=0pt, outer arc=4pt,
  boxsep=3pt, left=10pt, right=10pt, top=6pt, bottom=6pt,
  fontupper=\fontsize{10}{14}\selectfont\color{darkgreen}
}

\newcommand{\statcard}[2]{%
  \begin{tikzpicture}
    \node[fill=lightgray, rounded corners=4pt, minimum width=3.8cm, minimum height=2cm, inner sep=5pt, align=center, text width=3.5cm] {
      {\fontsize{20}{24}\selectfont\bfseries\color{primary}#1}\\[3pt]
      {\fontsize{8}{10}\selectfont\color{subtitle}\MakeUppercase{#2}}
    };
  \end{tikzpicture}%
}

\newcommand{\statcardgreen}[2]{%
  \begin{tikzpicture}
    \node[fill=lightgreen, rounded corners=4pt, minimum width=3.8cm, minimum height=2cm, inner sep=5pt, align=center, text width=3.5cm] {
      {\fontsize{20}{24}\selectfont\bfseries\color{darkgreen}#1}\\[3pt]
      {\fontsize{8}{10}\selectfont\color{subtitle}\MakeUppercase{#2}}
    };
  \end{tikzpicture}%
}

\newcommand{\statcardred}[2]{%
  \begin{tikzpicture}
    \node[fill=lightred, rounded corners=4pt, minimum width=3.8cm, minimum height=2cm, inner sep=5pt, align=center, text width=3.5cm] {
      {\fontsize{20}{24}\selectfont\bfseries\color{darkred}#1}\\[3pt]
      {\fontsize{8}{10}\selectfont\color{subtitle}\MakeUppercase{#2}}
    };
  \end{tikzpicture}%
}

\color{bodytext}

\begin{document}

% ============================================
% COVER
% ============================================
\thispagestyle{empty}
\begin{tikzpicture}[remember picture, overlay]
  \fill[darkbg] (current page.north west) rectangle (current page.south east);
  \node[anchor=north west, text width=14cm] at ($(current page.north west)+(2.8cm,-4cm)$) {
    {\fontsize{10}{12}\selectfont\color{accent}\textls[100]{EHRLICHER VERGLEICH}}\\[16pt]
    {\fontsize{34}{38}\selectfont\bfseries\color{white}Kalkulations-Vergleich\\[4pt]3 Demo-Bauteile}\\[20pt]
    {\fontsize{13}{17}\selectfont\color{lightondark}CNC Planer Pro vs. MBS b-logic ERP\\[4pt]Angebot 20260072 · Klöber Industrie GmbH}\\[40pt]
    {\fontsize{10}{13}\selectfont\color{accent}Bauteile}\\[4pt]
    {\fontsize{12}{15}\selectfont\color{white}Verbindungsplatte · Adapterplatte · Block}\\[4pt]
    {\fontsize{10}{13}\selectfont\color{lightondark}Werkstoffe: S235JR / 1.4571, AlMg3}\\[30pt]
    {\fontsize{10}{13}\selectfont\color{accent}Ergebnis}\\[4pt]
    {\fontsize{18}{22}\selectfont\bfseries\color{white}Ø 17\,\% Abweichung mit Einkaufspreis}\\[4pt]
    {\fontsize{11}{14}\selectfont\color{lightondark}Kein CAD, keine Geometrie-Analyse — trotzdem brauchbar}
  };
  \node[anchor=south west, text width=14cm] at ($(current page.south west)+(2.8cm,3cm)$) {
    {\fontsize{9}{12}\selectfont\color{subtitle}Florian Ziesche · 6. Februar 2026}\\[3pt]
    {\fontsize{8}{11}\selectfont\color{subtitle}Referenz: MBS Angebot 20260072 (b-logic ERP) · CNC Planer Pro v0.18 · Alle Angaben EUR netto}
  };
\end{tikzpicture}
\clearpage


% ============================================
% GESAMTÜBERSICHT
% ============================================
\section{Gesamtübersicht}
\ssubtitle{3 Demo-Bauteile aus CNC Planer Pro vs. MBS-Kalkulation}

\begin{center}
\statcardgreen{Ø\,1,17×}{Durchschnitt\\mit Einkaufspreis}
\hspace{10pt}
\statcard{0,99×}{Best Case\\(Verbindungsplatte)}
\hspace{10pt}
\statcardred{1,27×}{Worst Case\\(Block)}
\end{center}

\vspace{12pt}

\begin{tabularx}{\textwidth}{l l r r r r}
\toprule
\rowcolor{lightgray} \textbf{Demo-Bauteil} & \textbf{Werkstoff} & \textbf{Stk} & \textbf{MBS} & \textbf{CNC Planer} & \textbf{Abw.} \\
\midrule
Verbindungsplatte 440×50×20 & S235 / 1.4571 & 29 & 26,30 & 26,07 & \textcolor{darkgreen}{\textbf{--1\,\%}} \\
\rowcolor{lightgray} Adapterplatte 85×70×55 & AlMg3 & 10 & 72,89 & 91,41 & \textcolor{darkyellow}{\textbf{+25\,\%}} \\
Block 120×105×40 & AlMg3 & 5 & 64,16 & 81,57 & \textcolor{darkyellow}{\textbf{+27\,\%}} \\
\midrule
\rowcolor{lightgray} \multicolumn{4}{l}{\textbf{Durchschnitt}} & & \textbf{+17\,\%} \\
\bottomrule
\end{tabularx}

{\footnotesize Alle Preise EUR/Stück netto. CNC Planer Pro mit Einkaufspreis (= MBS Materialpreis). MBS-Daten aus Angebot 20260072, Positionen 1, 6 und 4.}

\begin{highlightbox}
\textbf{Kernaussage:} CNC Planer Pro liegt im Schnitt 17\,\% über MBS — ohne CAD-Analyse, ohne Geometrieerkennung, ohne historische Daten. Die Verbindungsplatte ist nahezu identisch (--1\,\%). Die Alu-Teile weichen stärker ab (+25--27\,\%), primär durch die Zuschlagsstruktur.
\end{highlightbox}


% ============================================
% POSITION 1: VERBINDUNGSPLATTE
% ============================================
\clearpage
\section{Verbindungsplatte}
\ssubtitle{440 × 50 × 20 mm · S235JR (MBS: 1.4571) · 29 Stück}

\subsection{Kostenvergleich}

\begin{tabularx}{\textwidth}{X r r r}
\toprule
\rowcolor{lightgray} \textbf{Kostenart} & \textbf{MBS b-logic} & \textbf{CNC Planer Pro} & \textbf{Differenz} \\
\midrule
Material / Stk & 5,17 & 5,17 & 0\,\% \\
\rowcolor{lightgray} Maschine + Lohn / Stk & 16,34 & 16,73 & +2\,\% \\
Herstellkosten / Stk & 21,51 & 22,00 & +2\,\% \\
\rowcolor{lightgray} \textbf{Angebotspreis / Stk} & \textbf{26,30} & \textbf{26,07} & \textcolor{darkgreen}{\textbf{--1\,\%}} \\
\midrule
MBS Marge & \multicolumn{3}{l}{+18,2\,\% (profitable Position)} \\
\bottomrule
\end{tabularx}

\subsection{Bearbeitungszeit}

\begin{tabularx}{\textwidth}{X r r}
\toprule
\rowcolor{lightgray} & \textbf{MBS (rückgerechnet)} & \textbf{CNC Planer Pro} \\
\midrule
Bearbeitungszeit & ca. 12 min & 12,5 min \\
\rowcolor{lightgray} Rüstzeit / Stk & ca. 0,5 min & 0,5 min \\
\textbf{Gesamt / Stk} & \textbf{ca. 12,5 min} & \textbf{13,0 min} \\
\bottomrule
\end{tabularx}

\begin{greenbox}
\textbf{Bewertung: Ausgezeichnet.} Einfache Geometrie + hohe Stückzahl = idealer Anwendungsfall für CNC Planer Pro. Die Zuschlagskalkulation trifft fast exakt. Bearbeitungszeit weicht nur 4\,\% ab.

\vspace{4pt}

\textbf{Warum so gut?} Platte = prismatische Grundform. Wenige OPs (Planfräsen, Kontur, Bohren). Kein Geometrie-Wissen nötig — Abmessungen reichen.
\end{greenbox}

\subsection{Ohne Einkaufspreis (Vollvolumen-Fallback)}

\begin{tabularx}{\textwidth}{X r r}
\toprule
\rowcolor{lightgray} & \textbf{Einkaufspreis} & \textbf{Vollvolumen} \\
\midrule
Materialkosten / Stk & 5,17 & 18,26 \\
\rowcolor{lightgray} Angebotspreis / Stk & 26,07 & 43,46 \\
\textbf{Abweichung vs. MBS} & \textcolor{darkgreen}{\textbf{--1\,\%}} & \textcolor{darkred}{\textbf{+65\,\%}} \\
\bottomrule
\end{tabularx}

\begin{warnbox}
\textbf{Ohne Einkaufspreis: +65\,\%.} Der Vollvolumen-Fallback (Dichte × Abmessungen × kg-Preis) schätzt Edelstahl-Flachstahl massiv zu hoch, weil Halbzeuge günstiger sind als Rohmaterial. → \textbf{Einkaufspreis-Override ist entscheidend.}
\end{warnbox}


% ============================================
% POSITION 2: ADAPTERPLATTE
% ============================================
\clearpage
\section{Adapterplatte}
\ssubtitle{85 × 70 × 55 mm · AlMg3 · 10 Stück}

\subsection{Kostenvergleich}

\begin{tabularx}{\textwidth}{X r r r}
\toprule
\rowcolor{lightgray} \textbf{Kostenart} & \textbf{MBS b-logic} & \textbf{CNC Planer Pro} & \textbf{Differenz} \\
\midrule
Material / Stk & 15,67 & 15,67 & 0\,\% \\
\rowcolor{lightgray} Maschine + Lohn / Stk & 58,64 & 60,68 & +3\,\% \\
Herstellkosten / Stk & 74,31 & 77,13 & +4\,\% \\
\rowcolor{lightgray} \textbf{Angebotspreis / Stk} & \textbf{72,89} & \textbf{91,41} & \textcolor{darkyellow}{\textbf{+25\,\%}} \\
\midrule
MBS Marge & \multicolumn{3}{l}{--1,9\,\% (leichter Verlust — Mischkalkulation)} \\
\bottomrule
\end{tabularx}

\subsection{Analyse der Abweichung}

\begin{tabularx}{\textwidth}{r X r}
\toprule
\rowcolor{lightgray} \textbf{\#} & \textbf{Ursache} & \textbf{Anteil} \\
\midrule
1 & \textbf{MBS verkauft unter HK} (--1,9\,\% Marge). CNC Planer Pro rechnet +8\,\% Gewinn. Allein das erklärt 10\,\% der 25\,\% Differenz. & ca. 10\,\% \\
\rowcolor{lightgray} 2 & \textbf{Kumulative Zuschläge} (46,7\,\% vs. ~42,6\,\%): VwGK + VtGK auf höhere Basis. & ca. 8\,\% \\
3 & \textbf{Bearbeitungszeiten} ohne Geometrie: Planer kennt keine Taschen/Features. & ca. 7\,\% \\
\bottomrule
\end{tabularx}

\begin{warnbox}
\textbf{Bewertung: Akzeptabel als Erstschätzung.} 25\,\% über MBS, aber ein Großteil (10\,\%) kommt daher, dass MBS dieses Teil unter HK verkauft. Gegenüber den reinen Herstellkosten liegt CNC Planer Pro nur +4\,\% daneben.

\vspace{4pt}

\textbf{Mitigation:} Zuschläge auf MBS-Niveau anpassen (konfigurierbar). Gewinnzuschlag pro Position statt pauschal. Dann: ca. +10\,\%.
\end{warnbox}

\subsection{Ohne Einkaufspreis (Vollvolumen-Fallback)}

\begin{tabularx}{\textwidth}{X r r}
\toprule
\rowcolor{lightgray} & \textbf{Einkaufspreis} & \textbf{Vollvolumen} \\
\midrule
Materialkosten / Stk & 15,67 & 4,18 \\
\rowcolor{lightgray} Angebotspreis / Stk & 91,41 & 77,40 \\
\textbf{Abweichung vs. MBS} & \textcolor{darkyellow}{\textbf{+25\,\%}} & \textcolor{darkyellow}{\textbf{+6\,\%}} \\
\bottomrule
\end{tabularx}

\begin{highlightbox}
\textbf{Kurioser Effekt:} Bei Alu ist der Vollvolumen-Fallback \textbf{günstiger} als der Einkaufspreis — weil Alu-Blöcke pro kg günstig sind, aber als Halbzeug (Platte/Block) einen Aufschlag haben. Hier ist der Einkaufspreis der ehrlichere Wert.
\end{highlightbox}


% ============================================
% POSITION 3: BLOCK
% ============================================
\clearpage
\section{Block}
\ssubtitle{120 × 105 × 40 mm · AlMg3 · 5 Stück}

\subsection{Kostenvergleich}

\begin{tabularx}{\textwidth}{X r r r}
\toprule
\rowcolor{lightgray} \textbf{Kostenart} & \textbf{MBS b-logic} & \textbf{CNC Planer Pro} & \textbf{Differenz} \\
\midrule
Material / Stk & 16,90 & 16,90 & 0\,\% \\
\rowcolor{lightgray} Maschine + Lohn / Stk & 49,25 & 50,94 & +3\,\% \\
Herstellkosten / Stk & 66,15 & 68,82 & +4\,\% \\
\rowcolor{lightgray} \textbf{Angebotspreis / Stk} & \textbf{64,16} & \textbf{81,57} & \textcolor{darkyellow}{\textbf{+27\,\%}} \\
\midrule
MBS Marge & \multicolumn{3}{l}{--3,0\,\% (Verlust — Mischkalkulation)} \\
\bottomrule
\end{tabularx}

\subsection{Analyse der Abweichung}

\begin{tabularx}{\textwidth}{r X r}
\toprule
\rowcolor{lightgray} \textbf{\#} & \textbf{Ursache} & \textbf{Anteil} \\
\midrule
1 & \textbf{MBS verkauft unter HK} (--3,0\,\% Marge). CNC Planer Pro rechnet +8\,\% Gewinn. & ca. 11\,\% \\
\rowcolor{lightgray} 2 & \textbf{Kumulative Zuschläge} (46,7\,\% vs. ~42,6\,\%). & ca. 8\,\% \\
3 & \textbf{Kleine Stückzahl} (5 Stk): Rüstanteil höher. & ca. 5\,\% \\
\rowcolor{lightgray} 4 & \textbf{Bearbeitungszeiten}: Geometrie unbekannt. & ca. 3\,\% \\
\bottomrule
\end{tabularx}

\begin{warnbox}
\textbf{Bewertung: Gleiches Muster wie Adapterplatte.} Gegenüber MBS-HK nur +4\,\% daneben. Die 27\,\% kommen primär aus dem Marge-Unterschied (MBS: --3\,\%, CNC Planer: +8\,\%) und kumulierten Zuschlägen.

\vspace{4pt}

\textbf{Mitigation:} Gleich wie Adapterplatte — Zuschläge konfigurieren, Positions-Marge einführen.
\end{warnbox}

\subsection{Ohne Einkaufspreis (Vollvolumen-Fallback)}

\begin{tabularx}{\textwidth}{X r r}
\toprule
\rowcolor{lightgray} & \textbf{Einkaufspreis} & \textbf{Vollvolumen} \\
\midrule
Materialkosten / Stk & 16,90 & 6,44 \\
\rowcolor{lightgray} Angebotspreis / Stk & 81,57 & 68,97 \\
\textbf{Abweichung vs. MBS} & \textcolor{darkyellow}{\textbf{+27\,\%}} & \textcolor{darkyellow}{\textbf{+8\,\%}} \\
\bottomrule
\end{tabularx}


% ============================================
% FEHLERQUELLEN & MITIGATIONS
% ============================================
\clearpage
\section{Fehlerquellen \& Maßnahmen}
\ssubtitle{Was die Abweichung verursacht — und was wir dagegen tun können}

\subsection{Die 4 Hauptursachen}

\begin{tabularx}{\textwidth}{r X r r}
\toprule
\rowcolor{lightgray} \textbf{\#} & \textbf{Ursache} & \textbf{Beitrag} & \textbf{Status} \\
\midrule
1 & \textbf{MBS-Mischkalkulation:} MBS verkauft 2 von 3 Positionen unter HK. CNC Planer rechnet mit pauschalem Gewinnzuschlag. & 10--11\,\% & \textcolor{darkyellow}{Feature nötig} \\
\rowcolor{lightgray} 2 & \textbf{Kumulative Zuschläge:} 46,7\,\% (CNC Planer) vs. ~42,6\,\% (MBS). & 4--8\,\% & \textcolor{darkgreen}{Konfigurierbar} \\
3 & \textbf{Materialpreis:} Ohne Einkaufspreis bis 65\,\% daneben (Stahl) oder --70\,\% (Alu). & 0--65\,\% & \textcolor{darkgreen}{Override vorhanden} \\
\rowcolor{lightgray} 4 & \textbf{Bearbeitungszeiten:} Ohne Geometrie ±5--10\,\% bei einfachen, bis ±30\,\% bei komplexen Teilen. & 3--7\,\% & \textcolor{darkyellow}{Editierbar machen} \\
\bottomrule
\end{tabularx}

\subsection{Maßnahmen-Roadmap}

\begin{tabularx}{\textwidth}{r X r r}
\toprule
\rowcolor{lightgray} \textbf{Prio} & \textbf{Maßnahme} & \textbf{Aufwand} & \textbf{Effekt} \\
\midrule
P0 & \textbf{Einkaufspreis-Override} — User hinterlegt Materialpreis & \textcolor{darkgreen}{Vorhanden} & Eliminiert \#3 \\
\rowcolor{lightgray} P0 & \textbf{Zuschläge konfigurierbar} — Betriebseigene GK-Sätze & \textcolor{darkgreen}{Vorhanden} & --4--8\,\% \\
P1 & \textbf{Editierbare OP-Zeiten} — Meister korrigiert Schätzungen & 1 Tag & --3--7\,\% \\
\rowcolor{lightgray} P1 & \textbf{Positions-Marge} — Individuelle Marge statt pauschal 8\,\% & 2 Tage & Mischkalkulation \\
P1 & \textbf{Halbzeug-Kalkulator} — Stangen/Platten statt Vollvolumen & 2 Tage & Besserer Fallback \\
\rowcolor{lightgray} P2 & \textbf{Nachkalk. → Lernschleife} — Ist-Zeiten verbessern Soll & 1 Woche & --5\,\%/Jahr \\
P2 & \textbf{Fremdleistungs-Modul} — Härten, Beschichten etc. & 1 Tag & Vollständigkeit \\
\bottomrule
\end{tabularx}

\begin{highlightbox}
\textbf{Realistisches Ziel nach P0 + P1:}
\begin{itemize}
  \item Einfache Teile (Platte): \textbf{±5\,\%} — direkt nutzbar für Angebote
  \item Mittlere Teile (Block, Adapterplatte): \textbf{±10--15\,\%} — gute Erstschätzung
\end{itemize}
\textbf{Ohne CAD-Analyse ist ±10\,\% das physikalische Limit.} Für Betriebe die heute im Kopf oder mit Excel kalkulieren, ist das eine klare Verbesserung.
\end{highlightbox}


% ============================================
% FAZIT
% ============================================
\clearpage
\section{Fazit}
\ssubtitle{Was dieser Vergleich zeigt}

\begin{darkhighlight}
\textbf{Die ehrliche Bilanz:}

\vspace{6pt}

\begin{tabularx}{\textwidth}{l X}
\textcolor{accent}{✓} & \textbf{Verbindungsplatte: --1\,\%.} Bei einfachen Teilen mit Einkaufspreis ist CNC Planer Pro sofort einsetzbar. \\[4pt]
\textcolor{accent}{✓} & \textbf{Herstellkosten stimmen.} Bei allen 3 Teilen nur +2--4\,\% über MBS-HK. Die Kalkulationslogik funktioniert. \\[4pt]
\textcolor{accent}{✓} & \textbf{Die Zuschlagskalkulation ist solide.} Abweichungen kommen nicht aus Rechenfehlern, sondern aus Strategie-Unterschieden (Mischkalk., variable Margen). \\[8pt]
\textcolor{darkred}{✗} & \textbf{MBS macht Mischkalkulation.} 2 von 3 Positionen unter HK — das kann CNC Planer Pro nicht automatisieren. \\[4pt]
\textcolor{darkred}{✗} & \textbf{Ohne Einkaufspreis: unbrauchbar} bei Stahl (+65\,\%), zufällig ok bei Alu. \\[4pt]
\textcolor{darkred}{✗} & \textbf{Bearbeitungszeiten} ohne Geometrie sind Richtwerte, keine Präzision.
\end{tabularx}
\end{darkhighlight}

\vspace{16pt}

\begin{center}
\statcardgreen{--1\,\%}{Verbindungsplatte\\(Best Case)}
\hspace{10pt}
\statcard{+17\,\%}{Durchschnitt\\(3 Bauteile)}
\hspace{10pt}
\statcard{< 10\,\%}{Ziel nach\\Mitigations}
\end{center}

\vspace{16pt}

\begin{highlightbox}
\textbf{Bottom Line:} CNC Planer Pro ist kein ERP-Ersatz. Aber für Betriebe die heute im Kopf oder mit dem Taschenrechner kalkulieren, liefert es bei einfachen Teilen sofort brauchbare Ergebnisse — und bei komplexeren Teilen einen strukturierten Startpunkt, den der Meister korrigieren kann.

\vspace{6pt}

\textbf{Die Herstellkosten stimmen. Die Angebotspreis-Differenz ist Strategie, nicht Mathematik.}
\end{highlightbox}

\vspace{16pt}

{\small\color{subtitle}\textit{Datengrundlage: MBS Angebot 20260072 (b-logic ERP, Stand 28.01.2026). CNC Planer Pro v0.18 mit MBS-kalibrierten Stundensätzen (EUR 70/h CNC). Materialpreise „Einkauf" = MBS-Materialkosten HK. Bearbeitungszeiten rückgerechnet aus MBS Maschinen-/Lohnkosten bei EUR 70/h.}}

\end{document}
