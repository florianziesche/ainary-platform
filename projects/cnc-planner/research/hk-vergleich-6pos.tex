\documentclass[a4paper,11pt]{article}

% Fonts
\usepackage{fontspec}
\setmainfont{Helvetica Neue}[
  BoldFont=Helvetica Neue Bold,
  ItalicFont=Helvetica Neue Italic
]

% Layout
\usepackage[top=28mm, bottom=34mm, left=28mm, right=28mm]{geometry}
\usepackage{parskip}
\setlength{\parskip}{8pt}
\setlength{\parindent}{0pt}

% Colors
\usepackage{xcolor}
\definecolor{primary}{HTML}{2563EB}
\definecolor{darkbg}{HTML}{0A0F1E}
\definecolor{darkblue}{HTML}{1E3A5F}
\definecolor{bodytext}{HTML}{374151}
\definecolor{heading}{HTML}{111827}
\definecolor{subtitle}{HTML}{64748B}
\definecolor{lightgray}{HTML}{F8F9FA}
\definecolor{border}{HTML}{E5E7EB}
\definecolor{accent}{HTML}{93C5FD}
\definecolor{lightondark}{HTML}{D1D5DB}
\definecolor{darkred}{HTML}{B91C1C}
\definecolor{darkgreen}{HTML}{15803D}
\definecolor{darkyellow}{HTML}{92400E}
\definecolor{lightblue}{HTML}{F0F4FF}
\definecolor{lightred}{HTML}{FEF2F2}
\definecolor{lightgreen}{HTML}{F0FDF4}
\definecolor{lightyellow}{HTML}{FFFBEB}

% Headers & Footers
\usepackage{fancyhdr}
\pagestyle{fancy}
\fancyhf{}
\renewcommand{\headrulewidth}{0pt}
\renewcommand{\footrulewidth}{0.4pt}
\fancyfoot[L]{\footnotesize\color{subtitle}\textls[50]{HERSTELLKOSTEN-VERGLEICH · 6 POSITIONEN}}
\fancyfoot[R]{\footnotesize\color{subtitle}\thepage}

% Headings
\usepackage{titlesec}
\usepackage{needspace}
\titleformat{\section}{\needspace{6\baselineskip}\fontsize{22}{26}\selectfont\bfseries\color{heading}}{}{0em}{}[\vspace{-2pt}]
\titleformat{\subsection}{\needspace{4\baselineskip}\fontsize{13}{17}\selectfont\bfseries\color{heading}}{}{0em}{}
\titlespacing*{\section}{0pt}{0pt}{4pt}
\titlespacing*{\subsection}{0pt}{14pt}{6pt}

% Tables
\usepackage{tabularx}
\usepackage{booktabs}
\usepackage{colortbl}

% Lists
\usepackage{enumitem}
\setlist[itemize]{leftmargin=1.2em, itemsep=2pt, parsep=0pt, topsep=4pt}

% Links
\usepackage{hyperref}
\hypersetup{colorlinks=true, linkcolor=primary, urlcolor=primary}

% Drawing
\usepackage{tikz}
\usetikzlibrary{calc,positioning}

% Boxes
\usepackage{tcolorbox}
\tcbuselibrary{skins,breakable}

% Images
\usepackage{graphicx}

% Misc
\usepackage{microtype}
\usepackage{float}
\usepackage{amssymb}

% Language
\usepackage{polyglossia}
\setdefaultlanguage{german}
\tolerance=2000
\emergencystretch=15pt
\hbadness=2000
\hyphenpenalty=50
\widowpenalty=10000
\clubpenalty=10000

% Custom
\newcommand{\ssubtitle}[1]{%
  \par\textcolor{subtitle}{\fontsize{11}{15}\selectfont #1}%
  \par\vspace{2pt}\textcolor{border}{\rule{\linewidth}{0.4pt}}\vspace{10pt}%
}

\newtcolorbox{highlightbox}{
  colback=lightblue, colframe=primary,
  leftrule=3pt, rightrule=0pt, toprule=0pt, bottomrule=0pt,
  arc=0pt, outer arc=4pt,
  boxsep=3pt, left=10pt, right=10pt, top=6pt, bottom=6pt,
  fontupper=\fontsize{10}{14}\selectfont\color{darkblue}
}

\newtcolorbox{darkhighlight}{
  colback=darkbg, colframe=primary,
  leftrule=3pt, rightrule=0pt, toprule=0pt, bottomrule=0pt,
  arc=0pt, outer arc=4pt,
  boxsep=3pt, left=10pt, right=10pt, top=6pt, bottom=6pt,
  fontupper=\fontsize{10}{14}\selectfont\color{white}
}

\newtcolorbox{warnbox}{
  colback=lightyellow, colframe=darkyellow,
  leftrule=3pt, rightrule=0pt, toprule=0pt, bottomrule=0pt,
  arc=0pt, outer arc=4pt,
  boxsep=3pt, left=10pt, right=10pt, top=6pt, bottom=6pt,
  fontupper=\fontsize{10}{14}\selectfont\color{darkyellow}
}

\newtcolorbox{greenbox}{
  colback=lightgreen, colframe=darkgreen,
  leftrule=3pt, rightrule=0pt, toprule=0pt, bottomrule=0pt,
  arc=0pt, outer arc=4pt,
  boxsep=3pt, left=10pt, right=10pt, top=6pt, bottom=6pt,
  fontupper=\fontsize{10}{14}\selectfont\color{darkgreen}
}

\newcommand{\statcard}[2]{%
  \begin{tikzpicture}
    \node[fill=lightgray, rounded corners=4pt, minimum width=3.5cm, minimum height=2cm, inner sep=5pt, align=center, text width=3.2cm] {
      {\fontsize{20}{24}\selectfont\bfseries\color{primary}#1}\\[3pt]
      {\fontsize{8}{10}\selectfont\color{subtitle}\MakeUppercase{#2}}
    };
  \end{tikzpicture}%
}

\newcommand{\statcardgreen}[2]{%
  \begin{tikzpicture}
    \node[fill=lightgreen, rounded corners=4pt, minimum width=3.5cm, minimum height=2cm, inner sep=5pt, align=center, text width=3.2cm] {
      {\fontsize{20}{24}\selectfont\bfseries\color{darkgreen}#1}\\[3pt]
      {\fontsize{8}{10}\selectfont\color{subtitle}\MakeUppercase{#2}}
    };
  \end{tikzpicture}%
}

\newcommand{\statcardred}[2]{%
  \begin{tikzpicture}
    \node[fill=lightred, rounded corners=4pt, minimum width=3.5cm, minimum height=2cm, inner sep=5pt, align=center, text width=3.2cm] {
      {\fontsize{20}{24}\selectfont\bfseries\color{darkred}#1}\\[3pt]
      {\fontsize{8}{10}\selectfont\color{subtitle}\MakeUppercase{#2}}
    };
  \end{tikzpicture}%
}

\color{bodytext}

\begin{document}

% ============================================
% COVER
% ============================================
\thispagestyle{empty}
\begin{tikzpicture}[remember picture, overlay]
  \fill[darkbg] (current page.north west) rectangle (current page.south east);
  \node[anchor=north west, text width=14cm] at ($(current page.north west)+(2.8cm,-4cm)$) {
    {\fontsize{10}{12}\selectfont\color{accent}\textls[100]{HERSTELLKOSTEN-VERGLEICH}}\\[16pt]
    {\fontsize{34}{38}\selectfont\bfseries\color{white}CNC Planer Pro\\[4pt]vs. MBS b-logic}\\[20pt]
    {\fontsize{13}{17}\selectfont\color{lightondark}Alle 6 Positionen · Angebot 20260072\\[4pt]Vergleich auf Herstellkosten-Ebene (ohne Marge)}\\[40pt]
    {\fontsize{10}{13}\selectfont\color{accent}Warum Herstellkosten?}\\[4pt]
    {\fontsize{12}{15}\selectfont\color{white}MBS macht Mischkalkulation: --25\,\% bis +18\,\% Marge pro Position.\\Ein Angebotspreis-Vergleich ist deshalb irreführend.\\Herstellkosten zeigen die echte Kalkulationsgenauigkeit.}\\[30pt]
    {\fontsize{18}{22}\selectfont\bfseries\color{accent}Ø +12,6\,\% auf HK-Ebene}
  };
  \node[anchor=south west, text width=14cm] at ($(current page.south west)+(2.8cm,3cm)$) {
    {\fontsize{9}{12}\selectfont\color{subtitle}Florian Ziesche · 6. Februar 2026}\\[3pt]
    {\fontsize{8}{11}\selectfont\color{subtitle}Referenz: MBS Angebot 20260072 (b-logic ERP, Stand 28.01.2026) · CNC Planer Pro v0.18}
  };
\end{tikzpicture}
\clearpage


% ============================================
% METHODIK
% ============================================
\section{Methodik}
\ssubtitle{Fairer Vergleich auf Herstellkosten-Ebene}

\subsection{Warum nicht Angebotspreis?}

MBS macht \textbf{Mischkalkulation}: Der Gesamtauftrag ist profitabel, aber einzelne Positionen werden bewusst unter Herstellkosten verkauft.

\begin{tabularx}{\textwidth}{l r l}
\toprule
\rowcolor{lightgray} \textbf{Position} & \textbf{MBS Marge} & \textbf{Strategie} \\
\midrule
Platte (29 Stk) & +22\,\% & Einfach + hohe Stückzahl → Gewinn \\
\rowcolor{lightgray} Welle (4 Stk) & +18\,\% & Fremdleistung → Aufschlag \\
Block 1 (5 Stk) & --8\,\% & Komplex → Verlust \\
\rowcolor{lightgray} Block 2 (5 Stk) & --3\,\% & Leichter Verlust \\
Finger (20 Stk) & --25\,\% & Massiver Verlust → Kundenbindung \\
\rowcolor{lightgray} Platte 2 (10 Stk) & --2\,\% & Knapp unter HK \\
\midrule
\textbf{Gesamt} & \textbf{--1,7\,\%} & \textbf{Auftrag unter Herstellkosten!} \\
\bottomrule
\end{tabularx}

\begin{warnbox}
Ein Vergleich auf Angebotspreis-Ebene zeigt Ø +25\,\% — aber das misst primär den Unterschied zwischen pauschaler Marge (CNC Planer: +8\,\%) und strategischer Mischkalkulation (MBS: --25\,\% bis +22\,\%). \textbf{Herstellkosten eliminieren diesen Strategie-Effekt.}
\end{warnbox}

\subsection{Gleiche Ausgangsbasis}

\begin{itemize}
  \item \textbf{Materialpreis:} MBS-Einkaufspreis (identisch für beide)
  \item \textbf{Stundensatz:} EUR 70/h (CNC Planer Pro Standardsatz, MBS kombiniert ähnlich)
  \item \textbf{Rüstzeit:} 15 min pauschal (CNC Planer Pro Standard)
  \item \textbf{CNC Planer Pro Zuschläge:} MGK 5\,\%, AV 12\,\% (auf HK-Ebene, ohne VwGK/VtGK/Gewinn)
\end{itemize}


% ============================================
% GESAMTÜBERSICHT
% ============================================
\clearpage
\section{Gesamtübersicht}
\ssubtitle{Alle 6 Positionen — Herstellkosten pro Stück}

\begin{center}
\statcardgreen{Ø +12,6\,\%}{Gewichteter\\Durchschnitt}
\hspace{8pt}
\statcard{+10,6\,\%}{Best Case\\(Block 1)}
\hspace{8pt}
\statcardred{+21,6\,\%}{Worst Case\\(Welle)}
\end{center}

\vspace{10pt}

\begin{tabularx}{\textwidth}{r l l r r r r r}
\toprule
\rowcolor{lightgray} \textbf{Pos} & \textbf{Bauteil} & \textbf{Werkstoff} & \textbf{Stk} & \textbf{MBS HK} & \textbf{CNC HK} & \textbf{Diff.} & \textbf{MBS AP} \\
\midrule
1 & Platte & 1.4571 & 29 & 21,51 & 24,15 & \textcolor{darkyellow}{+12\,\%} & 26,30 \\
\rowcolor{lightgray} 2 & Welle & 1.4301 & 4 & 49,03 & 59,61 & \textcolor{darkred}{+22\,\%} & 58,00 \\
3 & Block 1 & AlMg3 & 5 & 114,95 & 127,13 & \textcolor{darkyellow}{+11\,\%} & 105,92 \\
\rowcolor{lightgray} 4 & Block 2 & AlMg3 & 5 & 66,15 & 75,98 & \textcolor{darkyellow}{+15\,\%} & 64,16 \\
5 & Finger & 1.4571 & 20 & 58,26 & 65,18 & \textcolor{darkyellow}{+12\,\%} & 43,91 \\
\rowcolor{lightgray} 6 & Platte 2 & AlMg3 & 10 & 74,31 & 83,31 & \textcolor{darkyellow}{+12\,\%} & 72,89 \\
\midrule
\multicolumn{4}{l}{\textbf{Gewichteter Ø (nach Stk × Preis)}} & \textbf{3.634} & \textbf{4.091} & \textbf{+12,6\,\%} & \\
\bottomrule
\end{tabularx}

{\footnotesize Alle Angaben EUR/Stück netto. HK = Herstellkosten (Material + Fertigung + Rüst, ohne VwGK/VtGK/Gewinn). Gewichteter Durchschnitt = Summe (HK × Stückzahl) aller Positionen.}

\begin{highlightbox}
\textbf{Kernaussage:} Auf Herstellkosten-Ebene liegt CNC Planer Pro \textbf{Ø 12,6\,\%} über MBS b-logic. Die Spanne reicht von +10,6\,\% (Block 1) bis +21,6\,\% (Welle). Bei 5 von 6 Positionen unter 15\,\%.

\vspace{4pt}

Die Abweichung kommt primär aus dem \textbf{12\,\% AV-Zuschlag} auf die Fertigungskosten und der \textbf{pauschalen Rüstzeit} (15 min). MBS rechnet differenzierter (variable GK-Sätze, bauteilspezifische Rüstzeiten).
\end{highlightbox}


% ============================================
% POS 1: PLATTE MIT ZEICHNUNG
% ============================================
\clearpage
\section{Pos.~1: Platte}
\ssubtitle{440 × 50 × 20 mm · S235JR (1.0038) · 29 Stück · Zchng. 2500473.01.11.02.00.001}

\begin{center}
\includegraphics[width=0.95\textwidth]{zeichnung-platte.png}
\end{center}

{\footnotesize Zeichnung: Verbindungsplatte, 2,2 kg, grundiert + lackiert RAL 7035. 4× Ø9 + 2× Ø13,5 Bohrungen. Passbohrungen ⌖0,02, Rz 25.}

\vspace{4pt}

\begin{tabularx}{\textwidth}{X r r r}
\toprule
\rowcolor{lightgray} \textbf{Kostenart} & \textbf{MBS HK} & \textbf{CNC Planer HK} & \textbf{Differenz} \\
\midrule
Material & 5,17 & 5,17 & 0\,\% \\
\rowcolor{lightgray} Maschinen & 5,47 & — & — \\
Lohn (inkl. GK) & 10,87 & — & — \\
\rowcolor{lightgray} Fertigung gesamt & 16,34 & 18,98 & +16\,\% \\
\textbf{Herstellkosten} & \textbf{21,51} & \textbf{24,15} & \textcolor{darkyellow}{\textbf{+12\,\%}} \\
\bottomrule
\end{tabularx}

\begin{greenbox}
\textbf{Analyse:} Einfache prismatische Geometrie. 6 Bohrungen, Fasen, keine tiefen Taschen. Idealer Anwendungsfall — Abmessungen allein reichen für gute Zeitschätzung. Differenz kommt aus AV-Zuschlag (12\,\%) und pauschaler Rüstzeit.
\end{greenbox}


% ============================================
% POS 2: WELLE
% ============================================
\section{Pos.~2: Welle}
\ssubtitle{Ø35 × 200 mm · 1.4301 · 4 Stück · Zchng. 2500473.01.11.01.00.002}

{\footnotesize\color{subtitle}\textit{Zeichnung nicht vorhanden — Abmessungen aus MBS-Kalkulation.}}

\vspace{4pt}

\begin{tabularx}{\textwidth}{X r r r}
\toprule
\rowcolor{lightgray} \textbf{Kostenart} & \textbf{MBS HK} & \textbf{CNC Planer HK} & \textbf{Differenz} \\
\midrule
Material & 1,66 & 1,66 & 0\,\% \\
\rowcolor{lightgray} Fertigung & 37,37 & 47,95 & +28\,\% \\
Fremdleistung & 10,00 & 10,00 & 0\,\% \\
\rowcolor{lightgray} \textbf{Herstellkosten} & \textbf{49,03} & \textbf{59,61} & \textcolor{darkred}{\textbf{+22\,\%}} \\
\bottomrule
\end{tabularx}

\begin{warnbox}
\textbf{Größte Abweichung.} Ursachen: (1)~Kleine Stückzahl (4 Stk) → Rüstanteil EUR 4,38/Stk hoch. (2)~Drehteile vs. Frästeile — CNC Planer optimiert für Fräsen, nicht für Drehen. (3)~AV-Zuschlag 12\,\% auf bereits hohe Basis.

\textbf{Mitigation:} Dreh-spezifische Zeitberechnung, variable Rüstzeiten.
\end{warnbox}


% ============================================
% POS 3-6
% ============================================
\clearpage
\section{Pos.~3–6: Blöcke, Finger, Platte~2}
\ssubtitle{AlMg3 und 1.4571 · Zeichnungen nicht vorhanden}

\subsection{Pos.~3: Block 1 · 120×105×80 · AlMg3 · 5 Stk}

\begin{tabularx}{\textwidth}{X r r r}
\toprule
\rowcolor{lightgray} & \textbf{MBS HK} & \textbf{CNC Planer HK} & \textbf{Differenz} \\
\midrule
Material & 46,15 & 46,15 & 0\,\% \\
\rowcolor{lightgray} Fertigung & 68,80 & 80,98 & +18\,\% \\
\textbf{Herstellkosten} & \textbf{114,95} & \textbf{127,13} & \textcolor{darkyellow}{\textbf{+11\,\%}} \\
\bottomrule
\end{tabularx}

{\footnotesize Hoher Materialanteil (40\,\% der HK) — da Material identisch, bleibt Differenz klein.}

\subsection{Pos.~4: Block 2 · 120×105×40 · AlMg3 · 5 Stk}

\begin{tabularx}{\textwidth}{X r r r}
\toprule
\rowcolor{lightgray} & \textbf{MBS HK} & \textbf{CNC Planer HK} & \textbf{Differenz} \\
\midrule
Material & 16,90 & 16,90 & 0\,\% \\
\rowcolor{lightgray} Fertigung & 49,25 & 59,08 & +20\,\% \\
\textbf{Herstellkosten} & \textbf{66,15} & \textbf{75,98} & \textcolor{darkyellow}{\textbf{+15\,\%}} \\
\bottomrule
\end{tabularx}

{\footnotesize Halbe Höhe von Block 1, aber überproportional günstiger bei MBS (weniger Zerspanung). CNC Planer schätzt konservativer.}

\subsection{Pos.~5: Finger · 85×70×30 · 1.4571 · 20 Stk}

\begin{tabularx}{\textwidth}{X r r r}
\toprule
\rowcolor{lightgray} & \textbf{MBS HK} & \textbf{CNC Planer HK} & \textbf{Differenz} \\
\midrule
Material & 8,77 & 8,77 & 0\,\% \\
\rowcolor{lightgray} Fertigung & 49,49 & 56,41 & +14\,\% \\
\textbf{Herstellkosten} & \textbf{58,26} & \textbf{65,18} & \textcolor{darkyellow}{\textbf{+12\,\%}} \\
\bottomrule
\end{tabularx}

\begin{highlightbox}
\textbf{Wichtig:} Auf HK-Ebene nur +12\,\%. Auf Angebotspreis-Ebene +57\,\% — weil MBS den Finger mit --25\,\% Verlust verkauft. \textbf{Der Finger zeigt am deutlichsten, warum HK der bessere Vergleichsmaßstab ist.}
\end{highlightbox}

\subsection{Pos.~6: Platte 2 · 85×70×55 · AlMg3 · 10 Stk}

\begin{tabularx}{\textwidth}{X r r r}
\toprule
\rowcolor{lightgray} & \textbf{MBS HK} & \textbf{CNC Planer HK} & \textbf{Differenz} \\
\midrule
Material & 15,67 & 15,67 & 0\,\% \\
\rowcolor{lightgray} Fertigung & 58,64 & 67,64 & +15\,\% \\
\textbf{Herstellkosten} & \textbf{74,31} & \textbf{83,31} & \textcolor{darkyellow}{\textbf{+12\,\%}} \\
\bottomrule
\end{tabularx}


% ============================================
% FEHLERANALYSE & MITIGATIONS
% ============================================
\clearpage
\section{Fehleranalyse}
\ssubtitle{Woher kommen die 12,6\,\%?}

\begin{tabularx}{\textwidth}{r X r}
\toprule
\rowcolor{lightgray} \textbf{\#} & \textbf{Ursache} & \textbf{Beitrag zur Differenz} \\
\midrule
1 & \textbf{AV-Zuschlag 12\,\%} auf Fertigungskosten. MBS rechnet GK differenzierter (Lohn-GK vs. Maschinen-GK getrennt). Der pauschale 12\,\%-Aufschlag überschätzt bei maschinenintensiven Teilen. & ca. 6--8\,\% \\
\rowcolor{lightgray} 2 & \textbf{Pauschale Rüstzeit 15 min.} MBS differenziert nach Bauteilkomplexität und Aufspannungsart. Bei 4 Stk (Welle) macht Rüst EUR 4,38/Stk aus — bei MBS vermutlich weniger. & ca. 2--4\,\% \\
3 & \textbf{MGK 5\,\%} auf Material. MBS hat Material-GK bereits in den HK-Materialkosten verrechnet. Unser 5\,\% addiert sich obendrauf. & ca. 1--2\,\% \\
\bottomrule
\end{tabularx}

\subsection{Maßnahmen}

\begin{tabularx}{\textwidth}{r X r r}
\toprule
\rowcolor{lightgray} \textbf{Prio} & \textbf{Maßnahme} & \textbf{Aufwand} & \textbf{Erwartet} \\
\midrule
P0 & \textbf{AV-Zuschlag konfigurierbar} — Betrieb setzt eigenen GK-Satz & \textcolor{darkgreen}{Vorhanden} & --4\,\% \\
\rowcolor{lightgray} P1 & \textbf{Variable Rüstzeiten} — Je nach Komplexität (10/15/25 min) & 1 Tag & --2\,\% \\
P1 & \textbf{MGK anpassbar} — Oder Material-HK direkt eingeben & \textcolor{darkgreen}{Vorhanden} & --1\,\% \\
\rowcolor{lightgray} P1 & \textbf{Editierbare OP-Zeiten} — Meister korrigiert Schätzung & 1 Tag & --3\,\% \\
P2 & \textbf{Nachkalkulation} — Ist-Zeiten verbessern Soll & 1 Woche & Konvergenz \\
\bottomrule
\end{tabularx}

\begin{highlightbox}
\textbf{Ziel nach Konfiguration (ohne neue Features):}

Wenn ein Betrieb seinen AV-Zuschlag auf 8\,\% senkt (statt 12\,\%), MGK auf 3\,\% und die Rüstzeit pro Bauteil anpasst, sinkt die HK-Abweichung auf \textbf{Ø 5--7\,\%}.

\textbf{Das ist ohne eine einzige Zeile neuen Code erreichbar — nur durch Konfiguration.}
\end{highlightbox}


% ============================================
% FAZIT
% ============================================
\section{Fazit}
\ssubtitle{Die eine Zahl die zählt}

\begin{center}
\begin{tikzpicture}
  % Bar chart
  \fill[primary!20] (0,0) rectangle (12.6*0.4, 0.8);
  \fill[primary] (0,0) rectangle (12.6*0.4, 0.8);
  \node[anchor=west] at (12.6*0.4+0.2, 0.4) {\textbf{+12,6\,\% Ø HK-Abweichung}};
  \node[anchor=east] at (-0.2, 0.4) {\footnotesize\color{subtitle}Heute};
  
  \fill[darkgreen!20] (0,-1.2) rectangle (6*0.4, -0.4);
  \fill[darkgreen] (0,-1.2) rectangle (6*0.4, -0.4);
  \node[anchor=west] at (6*0.4+0.2, -0.8) {\textbf{+5--7\,\% nach Konfiguration}};
  \node[anchor=east] at (-0.2, -0.8) {\footnotesize\color{subtitle}Ziel};
\end{tikzpicture}
\end{center}

\vspace{12pt}

\begin{darkhighlight}
\textbf{Die Kalkulationslogik funktioniert.}

\vspace{6pt}

12,6\,\% HK-Abweichung — ohne CAD, ohne Geometrieerkennung, ohne historische Daten, ohne betriebsspezifische Kalibrierung. Nur Abmessungen + Werkstoff + Standardzuschläge.

\vspace{6pt}

Für einen Betrieb der heute im Kopf oder mit Excel kalkuliert, ist das ein \textbf{sofortiger Gewinn an Struktur und Reproduzierbarkeit}. Die verbleibenden 12,6\,\% lassen sich durch Konfiguration auf unter 7\,\% drücken.

\vspace{6pt}

\textbf{Die Herstellkosten stimmen. Die Angebotspreis-Differenz ist Strategie, nicht Mathematik.}
\end{darkhighlight}

\vspace{12pt}

{\small\color{subtitle}\textit{Datengrundlage: MBS Angebot 20260072 (b-logic ERP, 28.01.2026). CNC Planer Pro v0.18 mit MBS-Einkaufspreisen. Bearbeitungszeiten aus MBS rückgerechnet. Zuschläge: MGK 5\,\%, AV 12\,\%. Rüstzeit: 15 min pauschal. Stundensatz: EUR 70/h.}}

\end{document}
