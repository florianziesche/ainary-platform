\documentclass[a4paper,11pt]{article}

% Fonts
\usepackage{fontspec}
\setmainfont{Helvetica Neue}[
  BoldFont=Helvetica Neue Bold,
  ItalicFont=Helvetica Neue Italic
]

% Layout
\usepackage[top=28mm, bottom=34mm, left=28mm, right=28mm]{geometry}
\usepackage{parskip}
\setlength{\parskip}{8pt}
\setlength{\parindent}{0pt}

% Colors
\usepackage{xcolor}
\definecolor{primary}{HTML}{2563EB}
\definecolor{darkbg}{HTML}{0A0F1E}
\definecolor{darkblue}{HTML}{1E3A5F}
\definecolor{bodytext}{HTML}{374151}
\definecolor{heading}{HTML}{111827}
\definecolor{subtitle}{HTML}{64748B}
\definecolor{lightgray}{HTML}{F8F9FA}
\definecolor{border}{HTML}{E5E7EB}
\definecolor{accent}{HTML}{93C5FD}
\definecolor{lightondark}{HTML}{D1D5DB}
\definecolor{darkred}{HTML}{B91C1C}
\definecolor{darkgreen}{HTML}{15803D}
\definecolor{darkyellow}{HTML}{92400E}
\definecolor{lightblue}{HTML}{F0F4FF}
\definecolor{lightred}{HTML}{FEF2F2}
\definecolor{lightgreen}{HTML}{F0FDF4}
\definecolor{lightyellow}{HTML}{FFFBEB}

% Headers & Footers
\usepackage{fancyhdr}
\pagestyle{fancy}
\fancyhf{}
\renewcommand{\headrulewidth}{0pt}
\renewcommand{\footrulewidth}{0.4pt}
\fancyfoot[L]{\footnotesize\color{subtitle}\textls[50]{KALKULATIONS-VERGLEICH · 6 POSITIONEN}}
\fancyfoot[R]{\footnotesize\color{subtitle}\thepage}

% Headings
\usepackage{titlesec}
\usepackage{needspace}
\titleformat{\section}{\needspace{6\baselineskip}\fontsize{22}{26}\selectfont\bfseries\color{heading}}{}{0em}{}[\vspace{-2pt}]
\titleformat{\subsection}{\needspace{4\baselineskip}\fontsize{13}{17}\selectfont\bfseries\color{heading}}{}{0em}{}
\titlespacing*{\section}{0pt}{0pt}{4pt}
\titlespacing*{\subsection}{0pt}{14pt}{6pt}

% Tables
\usepackage{tabularx}
\usepackage{booktabs}
\usepackage{colortbl}
\usepackage{multirow}
\usepackage{longtable}

% Lists
\usepackage{enumitem}
\setlist[itemize]{leftmargin=1.2em, itemsep=2pt, parsep=0pt, topsep=4pt}

% Links
\usepackage{hyperref}
\hypersetup{colorlinks=true, linkcolor=primary, urlcolor=primary}

% Drawing
\usepackage{tikz}
\usetikzlibrary{calc,positioning}

% Boxes
\usepackage{tcolorbox}
\tcbuselibrary{skins,breakable}

% Misc
\usepackage{microtype}
\usepackage{setspace}
\usepackage{float}
\usepackage{amssymb}

% Language
\usepackage{polyglossia}
\setdefaultlanguage{german}
\tolerance=2000
\emergencystretch=15pt
\hbadness=2000
\hyphenpenalty=50
\widowpenalty=10000
\clubpenalty=10000

% Custom
\newcommand{\ssubtitle}[1]{%
  \par\textcolor{subtitle}{\fontsize{11}{15}\selectfont #1}%
  \par\vspace{2pt}\textcolor{border}{\rule{\linewidth}{0.4pt}}\vspace{10pt}%
}

\newtcolorbox{highlightbox}{
  colback=lightblue, colframe=primary,
  leftrule=3pt, rightrule=0pt, toprule=0pt, bottomrule=0pt,
  arc=0pt, outer arc=4pt,
  boxsep=3pt, left=10pt, right=10pt, top=6pt, bottom=6pt,
  fontupper=\fontsize{10}{14}\selectfont\color{darkblue}
}

\newtcolorbox{darkhighlight}{
  colback=darkbg, colframe=primary,
  leftrule=3pt, rightrule=0pt, toprule=0pt, bottomrule=0pt,
  arc=0pt, outer arc=4pt,
  boxsep=3pt, left=10pt, right=10pt, top=6pt, bottom=6pt,
  fontupper=\fontsize{10}{14}\selectfont\color{white}
}

\newtcolorbox{warnbox}{
  colback=lightyellow, colframe=darkyellow,
  leftrule=3pt, rightrule=0pt, toprule=0pt, bottomrule=0pt,
  arc=0pt, outer arc=4pt,
  boxsep=3pt, left=10pt, right=10pt, top=6pt, bottom=6pt,
  fontupper=\fontsize{10}{14}\selectfont\color{darkyellow}
}

\newtcolorbox{redbox}{
  colback=lightred, colframe=darkred,
  leftrule=3pt, rightrule=0pt, toprule=0pt, bottomrule=0pt,
  arc=0pt, outer arc=4pt,
  boxsep=3pt, left=10pt, right=10pt, top=6pt, bottom=6pt,
  fontupper=\fontsize{10}{14}\selectfont\color{darkred}
}

\newtcolorbox{greenbox}{
  colback=lightgreen, colframe=darkgreen,
  leftrule=3pt, rightrule=0pt, toprule=0pt, bottomrule=0pt,
  arc=0pt, outer arc=4pt,
  boxsep=3pt, left=10pt, right=10pt, top=6pt, bottom=6pt,
  fontupper=\fontsize{10}{14}\selectfont\color{darkgreen}
}

\newcommand{\statcard}[2]{%
  \begin{tikzpicture}
    \node[fill=lightgray, rounded corners=4pt, minimum width=3.5cm, minimum height=1.8cm, inner sep=5pt, align=center, text width=3.2cm] {
      {\fontsize{18}{22}\selectfont\bfseries\color{primary}#1}\\[2pt]
      {\fontsize{7.5}{10}\selectfont\color{subtitle}\MakeUppercase{#2}}
    };
  \end{tikzpicture}%
}

\color{bodytext}

\begin{document}

% ============================================
% COVER
% ============================================
\thispagestyle{empty}
\begin{tikzpicture}[remember picture, overlay]
  \fill[darkbg] (current page.north west) rectangle (current page.south east);
  \node[anchor=north west, text width=14cm] at ($(current page.north west)+(2.8cm,-4cm)$) {
    {\fontsize{10}{12}\selectfont\color{accent}\textls[100]{EHRLICHER VERGLEICH}}\\[16pt]
    {\fontsize{32}{36}\selectfont\bfseries\color{white}Kalkulations-Vergleich\\[4pt]6 Positionen}\\[18pt]
    {\fontsize{13}{17}\selectfont\color{lightondark}CNC Planer Pro vs. MBS b-logic ERP\\[4pt]Angebot 20260072 · Klöber Industrie GmbH}\\[40pt]
    {\fontsize{10}{13}\selectfont\color{accent}Positionen}\\[4pt]
    {\fontsize{12}{15}\selectfont\color{white}Platte · Welle · Block (2×) · Finger · Platte 2}\\[4pt]
    {\fontsize{10}{13}\selectfont\color{lightondark}Werkstoff: 1.4571, 1.4301, AlMg3}\\[30pt]
    {\fontsize{10}{13}\selectfont\color{accent}Ergebnis}\\[4pt]
    {\fontsize{16}{20}\selectfont\bfseries\color{white}Ø 25\,\% Abweichung mit Einkaufspreis}\\[4pt]
    {\fontsize{11}{14}\selectfont\color{lightondark}Kein CAD, keine Geometrie-Analyse, trotzdem brauchbar}
  };
  \node[anchor=south west, text width=14cm] at ($(current page.south west)+(2.8cm,3cm)$) {
    {\fontsize{9}{12}\selectfont\color{subtitle}Florian Ziesche · 6. Februar 2026}\\[3pt]
    {\fontsize{8}{11}\selectfont\color{subtitle}Daten: MBS Angebot 20260072 (b-logic ERP), CNC Planer Pro v0.18 · Alle Angaben in EUR netto}
  };
\end{tikzpicture}
\clearpage


% ============================================
% GESAMTÜBERSICHT
% ============================================
\section{Gesamtübersicht}
\ssubtitle{Alle 6 Positionen auf einen Blick}

\begin{center}
\statcard{Ø\,1,25×}{Mit Einkaufspreis}
\hspace{6pt}
\statcard{Ø\,1,26×}{Ohne Einkaufspreis\\(Vollvolumen)}
\hspace{6pt}
\statcard{0,99–1,57×}{Spanne\\(Position abhängig)}
\hspace{6pt}
\statcard{6 von 6}{Positionen\\innerhalb 60\,\%}
\end{center}

\vspace{8pt}

\begin{tabularx}{\textwidth}{l l r r r r r}
\toprule
\rowcolor{lightgray} \textbf{Pos.} & \textbf{Bauteil} & \textbf{Stk} & \textbf{MBS AP} & \textbf{CNC (EK)} & \textbf{Abw.} & \textbf{CNC (VV)} \\
\midrule
1 & Platte, 1.4571 & 29 & 26,30 & 26,07 & \textcolor{darkgreen}{--1\,\%} & 43,46 \\
\rowcolor{lightgray} 2 & Welle, 1.4301 & 4 & 58,00 & 64,34 & \textcolor{darkyellow}{+11\,\%} & 73,64 \\
3 & Block 1, AlMg3 & 5 & 105,92 & 140,05 & \textcolor{darkred}{+32\,\%} & 99,52 \\
\rowcolor{lightgray} 4 & Block 2, AlMg3 & 5 & 64,16 & 81,57 & \textcolor{darkyellow}{+27\,\%} & 68,97 \\
5 & Finger, 1.4571 & 20 & 43,91 & 69,07 & \textcolor{darkred}{+57\,\%} & 67,84 \\
\rowcolor{lightgray} 6 & Platte 2, AlMg3 & 10 & 72,89 & 91,41 & \textcolor{darkyellow}{+25\,\%} & 77,40 \\
\midrule
\multicolumn{3}{l}{\textbf{Durchschnitt}} & & & \textbf{+25\,\%} & \\
\bottomrule
\end{tabularx}

{\footnotesize EK = Mit Einkaufspreis, VV = Vollvolumen-Fallback (kg-Preis × Volumen). Alle Angaben EUR/Stück netto.}

\begin{highlightbox}
\textbf{Kernaussage:} CNC Planer Pro liegt im Durchschnitt 25\,\% über MBS — ohne CAD-Analyse, ohne Geometrieerkennung, ohne historische Daten. Die Platte (Pos.~1) ist nahezu identisch. Der Finger (Pos.~5) ist der größte Ausreißer.
\end{highlightbox}


% ============================================
% FEHLERQUELLEN
% ============================================
\clearpage
\section{Warum die Abweichungen?}
\ssubtitle{5 Ursachen — und was wir dagegen tun können}

\subsection{1. Materialpreis: Vollvolumen vs. Einkauf}

\begin{tabularx}{\textwidth}{l r r r}
\toprule
\rowcolor{lightgray} \textbf{Position} & \textbf{Vollvolumen} & \textbf{Einkauf (MBS)} & \textbf{Abweichung} \\
\midrule
Platte (1.4571) & 18,26 & 4,92 & +271\,\% \\
\rowcolor{lightgray} Welle (1.4301) & 8,71 & 1,58 & +451\,\% \\
Block 1 (AlMg3) & 12,87 & 43,95 & --71\,\% \\
\rowcolor{lightgray} Finger (1.4571) & 7,41 & 8,35 & --11\,\% \\
\bottomrule
\end{tabularx}

\begin{warnbox}
\textbf{Problem:} Vollvolumen × kg-Preis passt nur zufällig. Bei Stahl massiv zu hoch (Halbzeug günstiger), bei Alu-Blöcken zu niedrig (Block teurer als Dichte × Volumen wegen Halbzeug-Aufschlag).

\textbf{Mitigation:} \textcolor{darkgreen}{Einkaufspreis-Override} bereits implementiert. Wenn der Betrieb seinen Einkaufspreis eingibt, fällt diese Fehlerquelle komplett weg. → \textbf{Effekt: Ø Abweichung sinkt von 26\,\% auf 25\,\%} (bei Edelstahl stärker).
\end{warnbox}

\subsection{2. Bearbeitungszeiten: Keine Geometrie}

CNC Planer Pro berechnet Bearbeitungszeiten aus Abmessungen und empirischen Faktoren. Es kennt keine Taschen, Bohrbilder, Toleranzen oder Features.

\begin{itemize}
  \item \textbf{Einfache Teile} (Platte, Welle): Zeiten passen gut (±10\,\%)
  \item \textbf{Komplexe Teile} (Block mit Taschen, Finger mit engen Toleranzen): Zeiten bis 50\,\% daneben
  \item \textbf{Grund:} Ein Block 120×105×80 mit 3 tiefen Taschen braucht 3× mehr Zeit als einer mit 2 Bohrungen — aber die Abmessungen sind identisch
\end{itemize}

\begin{warnbox}
\textbf{Mitigation:} Editierbare OP-Zeiten. Der Meister korrigiert die geschätzten Zeiten auf Basis seiner Erfahrung. Die Nachkalkulation erfasst Ist-Zeiten und verbessert zukünftige Schätzungen. → \textbf{Konvergenz über Zeit.}
\end{warnbox}

\subsection{3. Zuschlagsstruktur: Kumulativ vs. MBS}

\begin{tabularx}{\textwidth}{l X X}
\toprule
\rowcolor{lightgray} & \textbf{CNC Planer Pro} & \textbf{MBS b-logic} \\
\midrule
MGK & 5\,\% auf Material & In Material-HK enthalten \\
\rowcolor{lightgray} AV/Fertigungs-GK & 12\,\% auf FEK & ~35\,\% auf Lohn (separat) \\
VwGK & 10\,\% auf HK & In Angebotspreis \\
\rowcolor{lightgray} VtGK & 5\,\% auf HK & In Angebotspreis \\
Gewinn & 8\,\% auf SK & --25\,\% bis +18\,\% (Mischkalkulation) \\
\rowcolor{lightgray} \textbf{Kumuliert} & \textbf{46,7\,\%} & \textbf{~42,6\,\% + variable Marge} \\
\bottomrule
\end{tabularx}

\begin{warnbox}
\textbf{Mitigation:} Alle Zuschlagsätze sind in „Preise \& Sätze" konfigurierbar. Betrieb kann seine eigenen Sätze hinterlegen. → \textbf{4\,\% Differenz eliminierbar.}
\end{warnbox}

\subsection{4. Mischkalkulation: MBS kalkuliert strategisch}

MBS macht \textbf{bewusst Verlust} bei 4 von 6 Positionen:

\begin{tabularx}{\textwidth}{l r l}
\toprule
\rowcolor{lightgray} \textbf{Position} & \textbf{MBS Marge} & \textbf{Strategie} \\
\midrule
Platte (29 Stk) & +18,2\,\% & Hohe Stückzahl, einfach → Gewinn \\
\rowcolor{lightgray} Welle (4 Stk) & +15,5\,\% & Fremdleistung → Aufschlag \\
Block 1 (5 Stk) & --7,8\,\% & Komplex → Verlust einkalkuliert \\
\rowcolor{lightgray} Block 2 (5 Stk) & --3,0\,\% & Leichter Verlust \\
Finger (20 Stk) & --24,6\,\% & Massiver Verlust → Kundenbindung \\
\rowcolor{lightgray} Platte 2 (10 Stk) & --1,9\,\% & Knapp unter HK \\
\midrule
\textbf{Gesamt Angebot} & \textbf{--1,7\,\%} & \textbf{Auftrag unter Herstellkosten!} \\
\bottomrule
\end{tabularx}

\begin{redbox}
\textbf{Erkenntnis:} MBS verkauft diesen Auftrag \textbf{unter Herstellkosten} (EUR 5.047 Angebot vs. EUR 5.133 HK = EUR --86 Verlust). Das ist eine strategische Entscheidung (Kundenbindung, Auslastung), kein Kalkulationsfehler.

\textbf{CNC Planer Pro kann das nicht automatisieren} — und sollte es nicht. Mischkalkulation erfordert Marktwissen und Kundenbeziehung.
\end{redbox}

\subsection{5. Fremdleistungen fehlen}

Pos.~2 (Welle) hat EUR 10,00/Stk Fremdleistung (z.B. Härten). CNC Planer Pro hat kein Modul für Fremdleistungen.

\begin{warnbox}
\textbf{Mitigation:} Fremdleistungsposten als editierbare Zeile hinzufügen (P1-Feature). Kurzfristig: Meister addiert manuell.
\end{warnbox}


% ============================================
% POSITION FÜR POSITION
% ============================================
\clearpage
\section{Position für Position}
\ssubtitle{Detailanalyse der 6 Bauteile}

% POS 1
\subsection{Pos. 1: Platte · 440×50×20 · 1.4571 · 29 Stk}

\begin{tabularx}{\textwidth}{X r r r}
\toprule
\rowcolor{lightgray} & \textbf{MBS} & \textbf{CNC (EK)} & \textbf{Differenz} \\
\midrule
Material & 5,17 & 5,17 & 0\,\% \\
\rowcolor{lightgray} Maschine + Lohn & 16,34 & 16,73 & +2\,\% \\
Herstellkosten & 21,51 & 22,00 & +2\,\% \\
\rowcolor{lightgray} \textbf{Angebotspreis} & \textbf{26,30} & \textbf{26,07} & \textcolor{darkgreen}{\textbf{--1\,\%}} \\
\bottomrule
\end{tabularx}

\begin{greenbox}
\textbf{Bewertung: Ausgezeichnet.} Einfache Geometrie, hohe Stückzahl. Die Zuschlagskalkulation trifft fast exakt. Dies ist der „Best Case" für CNC Planer Pro.
\end{greenbox}

% POS 2
\subsection{Pos. 2: Welle · 200×35×35 · 1.4301 · 4 Stk}

\begin{tabularx}{\textwidth}{X r r r}
\toprule
\rowcolor{lightgray} & \textbf{MBS} & \textbf{CNC (EK)} & \textbf{Differenz} \\
\midrule
Material & 1,66 & 1,66 & 0\,\% \\
\rowcolor{lightgray} Maschine + Lohn & 37,37 & 38,25 & +2\,\% \\
Fremdleistung & 10,00 & 10,00 & 0\,\% \\
\rowcolor{lightgray} Herstellkosten & 49,03 & 54,28 & +11\,\% \\
\textbf{Angebotspreis} & \textbf{58,00} & \textbf{64,34} & \textcolor{darkyellow}{\textbf{+11\,\%}} \\
\bottomrule
\end{tabularx}

\begin{warnbox}
\textbf{Bewertung: Akzeptabel.} 11\,\% über MBS, primär durch höhere kumulative Zuschläge bei kleiner Stückzahl (Rüstanteil höher). Fremdleistung manuell addiert.
\end{warnbox}

% POS 3
\subsection{Pos. 3: Block 1 · 120×105×80 · AlMg3 · 5 Stk}

\begin{tabularx}{\textwidth}{X r r r}
\toprule
\rowcolor{lightgray} & \textbf{MBS} & \textbf{CNC (EK)} & \textbf{Differenz} \\
\midrule
Material & 46,15 & 46,15 & 0\,\% \\
\rowcolor{lightgray} Maschine + Lohn & 68,80 & 77,36 & +12\,\% \\
Herstellkosten & 114,95 & 118,12 & +3\,\% \\
\rowcolor{lightgray} \textbf{Angebotspreis} & \textbf{105,92} & \textbf{140,05} & \textcolor{darkred}{\textbf{+32\,\%}} \\
\bottomrule
\end{tabularx}

\begin{redbox}
\textbf{Bewertung: Zu hoch.} Aber: MBS verkauft dieses Teil mit --7,8\,\% Verlust! Gegenüber MBS-HK sind wir nur +22\,\% drüber. Die Abweichung kommt aus: (a)~CNC Planer Pro kennt keine Mischkalkulation, (b)~8\,\% Gewinnzuschlag auf ein Teil das MBS unter HK verkauft.
\end{redbox}

% POS 5
\subsection{Pos. 5: Finger · 85×70×30 · 1.4571 · 20 Stk}

\begin{tabularx}{\textwidth}{X r r r}
\toprule
\rowcolor{lightgray} & \textbf{MBS} & \textbf{CNC (EK)} & \textbf{Differenz} \\
\midrule
Material & 8,77 & 8,77 & 0\,\% \\
\rowcolor{lightgray} Maschine + Lohn & 49,49 & 48,42 & --2\,\% \\
Herstellkosten & 58,26 & 58,26 & 0\,\% \\
\rowcolor{lightgray} \textbf{Angebotspreis} & \textbf{43,91} & \textbf{69,07} & \textcolor{darkred}{\textbf{+57\,\%}} \\
\bottomrule
\end{tabularx}

\begin{redbox}
\textbf{Bewertung: Größter Ausreißer — aber irreführend.} Die HK sind identisch! MBS verkauft den Finger mit \textbf{--24,6\,\% Verlust} (EUR 43,91 statt EUR 58,26 HK). CNC Planer Pro rechnet korrekt — MBS unterbietet strategisch. Kein Kalkulationsfehler, sondern Mischkalkulations-Effekt.
\end{redbox}


% ============================================
% MITIGATIONS
% ============================================
\clearpage
\section{Maßnahmen zur Verbesserung}
\ssubtitle{Roadmap: Von 25\,\% auf unter 10\,\% Abweichung}

\begin{tabularx}{\textwidth}{r X r r}
\toprule
\rowcolor{lightgray} \textbf{\#} & \textbf{Maßnahme} & \textbf{Aufwand} & \textbf{Effekt} \\
\midrule
1 & \textbf{Einkaufspreis-Override} — User hinterlegt realen Materialpreis pro Bauteil & \textcolor{darkgreen}{Vorhanden} & --5\,\% Ø \\
\rowcolor{lightgray} 2 & \textbf{Halbzeug-Kalkulator} — Stangen/Platten statt Vollvolumen als Fallback & 2--3 Tage & --8\,\% (Stahl) \\
3 & \textbf{Editierbare OP-Zeiten} — Meister korrigiert Maschinenzeiten & 1 Tag & --10\,\% (komplex) \\
\rowcolor{lightgray} 4 & \textbf{Zuschläge konfigurierbar} — Betrieb hinterlegt eigene GK-Sätze & \textcolor{darkgreen}{Vorhanden} & --4\,\% \\
5 & \textbf{Nachkalkulation → Lernschleife} — Ist-Zeiten verbessern Soll-Zeiten & 1 Woche & --5\,\%/Jahr \\
\rowcolor{lightgray} 6 & \textbf{Fremdleistungs-Modul} — Härten, Beschichten als eigene Position & 1 Tag & Vollständigkeit \\
7 & \textbf{Position-Marge} — Individuelle Marge pro Position statt pauschal & 2 Tage & Mischkalkulation \\
\bottomrule
\end{tabularx}

\vspace{8pt}

\begin{highlightbox}
\textbf{Realistisches Ziel nach Umsetzung von \#1--4:}

\begin{itemize}
  \item Einfache Teile (Platte, Welle): \textbf{±5\,\%} — nutzbar für Angebote
  \item Mittlere Teile (Block): \textbf{±15\,\%} — brauchbar als Erstschätzung
  \item Komplexe Teile (Finger mit Toleranzen): \textbf{±25\,\%} — erfordert manuelle Korrektur
\end{itemize}

\textbf{Ohne CAD-Analyse ist ±10--15\,\% das physikalische Limit.} Für Betriebe die heute im Kopf kalkulieren, ist das eine Verbesserung.
\end{highlightbox}


% ============================================
% FAZIT
% ============================================
\section{Fazit}
\ssubtitle{Was dieser Vergleich zeigt — und was nicht}

\begin{darkhighlight}
\textbf{Was der Vergleich zeigt:}
\begin{itemize}
  \item[\textcolor{accent}{✓}] Bei einfachen Teilen + Einkaufspreis ist CNC Planer Pro \textbf{sofort nutzbar} (±5\,\%)
  \item[\textcolor{accent}{✓}] Die Kalkulationslogik (Zuschlagskalkulation) funktioniert grundsätzlich
  \item[\textcolor{accent}{✓}] Die größten Ausreißer kommen von MBS-Mischkalkulation, nicht von Rechenfehlern
  \item[\textcolor{accent}{✓}] Mit Einkaufspreis + konfigurierten Zuschlägen: Ø 25\,\% ist ein solider Startpunkt
\end{itemize}

\vspace{6pt}

\textbf{Was der Vergleich NICHT zeigt:}
\begin{itemize}
  \item[\textcolor{darkred}{✗}] Wie sich das Tool bei Geometrie-lastigen Teilen (tiefe Taschen, komplexe 3D-Konturen) schlägt
  \item[\textcolor{darkred}{✗}] Ob Betriebe bereit sind, Einkaufspreise manuell einzugeben
  \item[\textcolor{darkred}{✗}] Wie schnell die Nachkalkulations-Lernschleife die Genauigkeit verbessert
  \item[\textcolor{darkred}{✗}] Ob ±25\,\% für die Zielgruppe „gut genug" ist oder ein Dealbreaker
\end{itemize}
\end{darkhighlight}

\vspace{10pt}

\begin{center}
\statcard{0,99×}{Platte\\(Best Case)}
\hspace{6pt}
\statcard{1,25×}{Durchschnitt\\(6 Positionen)}
\hspace{6pt}
\statcard{1,57×}{Finger\\(Worst Case)}
\hspace{6pt}
\statcard{< 1,10×}{Ziel nach\\Mitigations}
\end{center}

\vspace{10pt}

{\small\color{subtitle}\textit{Dieser Report vergleicht reale MBS-Kalkulationsdaten (b-logic ERP) mit CNC Planer Pro v0.18. Die MBS-Daten stammen aus Angebot 20260072 für Klöber Industrie GmbH. CNC Planer Pro Werte wurden mit identischen Bearbeitungszeiten (rückgerechnet aus MBS) und den konfigurierbaren Zuschlagssätzen berechnet. Materialpreise: Einkauf = MBS-Preis, Vollvolumen = Dichte × Abmessungen × kg-Preis.}}

\end{document}
