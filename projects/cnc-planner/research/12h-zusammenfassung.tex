\documentclass[a4paper,11pt]{article}

% Fonts
\usepackage{fontspec}
\setmainfont{Helvetica Neue}[
  BoldFont=Helvetica Neue Bold,
  ItalicFont=Helvetica Neue Italic
]

% Layout
\usepackage[top=30mm, bottom=35mm, left=28mm, right=28mm]{geometry}
\usepackage{parskip}
\setlength{\parskip}{8pt}
\setlength{\parindent}{0pt}

% Colors
\usepackage{xcolor}
\definecolor{primary}{HTML}{2563EB}
\definecolor{darkbg}{HTML}{0A0F1E}
\definecolor{darkblue}{HTML}{1E3A5F}
\definecolor{bodytext}{HTML}{374151}
\definecolor{heading}{HTML}{111827}
\definecolor{subtitle}{HTML}{64748B}
\definecolor{lightgray}{HTML}{F8F9FA}
\definecolor{border}{HTML}{E5E7EB}
\definecolor{accent}{HTML}{93C5FD}
\definecolor{lightondark}{HTML}{D1D5DB}
\definecolor{darkgreen}{HTML}{15803D}
\definecolor{lightblue}{HTML}{F0F4FF}
\definecolor{lightgreen}{HTML}{F0FDF4}
\definecolor{lightyellow}{HTML}{FEFCE8}
\definecolor{darkyellow}{HTML}{854D0E}

% Headers & Footers
\usepackage{fancyhdr}
\pagestyle{fancy}
\fancyhf{}
\renewcommand{\headrulewidth}{0pt}
\renewcommand{\footrulewidth}{0.4pt}
\fancyfoot[L]{\footnotesize\color{subtitle}\textls[50]{12-STUNDEN SPRINT $\cdot$ 5.--6. FEBRUAR 2026}}
\fancyfoot[R]{\footnotesize\color{subtitle}\thepage}

% Headings
\usepackage{titlesec}
\usepackage{needspace}
\titleformat{\section}{\needspace{6\baselineskip}\fontsize{22}{26}\selectfont\bfseries\color{heading}}{}{0em}{}[\vspace{-2pt}]
\titleformat{\subsection}{\needspace{4\baselineskip}\fontsize{13}{17}\selectfont\bfseries\color{heading}}{}{0em}{}
\titlespacing*{\section}{0pt}{0pt}{4pt}
\titlespacing*{\subsection}{0pt}{14pt}{6pt}

% Tables
\usepackage{tabularx}
\usepackage{booktabs}
\usepackage{colortbl}

% Lists
\usepackage{enumitem}
\setlist[itemize]{leftmargin=1.2em, itemsep=2pt, parsep=0pt, topsep=4pt}

% Links
\usepackage{hyperref}
\hypersetup{colorlinks=true, linkcolor=primary, urlcolor=primary}

% Drawing
\usepackage{tikz}
\usetikzlibrary{calc,positioning}

% Boxes
\usepackage{tcolorbox}
\tcbuselibrary{skins,breakable}

% Misc
\usepackage{microtype}
\usepackage{graphicx}
\usepackage{float}

% Language
\usepackage{polyglossia}
\setdefaultlanguage{german}
\tolerance=2000
\emergencystretch=15pt
\hbadness=2000
\widowpenalty=10000
\clubpenalty=10000

% Custom
\newcommand{\ssubtitle}[1]{%
  \par\textcolor{subtitle}{\fontsize{11}{15}\selectfont #1}%
  \par\vspace{2pt}\textcolor{border}{\rule{\linewidth}{0.4pt}}\vspace{10pt}%
}

\newtcolorbox{highlightbox}{
  colback=lightblue, colframe=primary,
  leftrule=3pt, rightrule=0pt, toprule=0pt, bottomrule=0pt,
  arc=0pt, outer arc=4pt,
  boxsep=4pt, left=12pt, right=12pt, top=8pt, bottom=8pt,
  fontupper=\fontsize{10.5}{14}\selectfont\color{darkblue}
}

\newtcolorbox{kpibox}[1]{
  colback=lightgreen, colframe=darkgreen,
  leftrule=3pt, rightrule=0pt, toprule=0pt, bottomrule=0pt,
  arc=0pt, outer arc=4pt,
  boxsep=4pt, left=12pt, right=12pt, top=8pt, bottom=8pt,
  title={\textbf{#1}},
  coltitle=darkgreen,
  fonttitle=\fontsize{12}{14}\selectfont,
  fontupper=\fontsize{10.5}{14}\selectfont\color{bodytext}
}

\newtcolorbox{warnbox}{
  colback=lightyellow, colframe=darkyellow,
  leftrule=3pt, rightrule=0pt, toprule=0pt, bottomrule=0pt,
  arc=0pt, outer arc=4pt,
  boxsep=4pt, left=12pt, right=12pt, top=8pt, bottom=8pt,
  fontupper=\fontsize{10.5}{14}\selectfont\color{darkyellow}
}

\color{bodytext}

\begin{document}

% ============================================
% COVER
% ============================================
\thispagestyle{empty}
\begin{tikzpicture}[remember picture, overlay]
  \fill[darkbg] (current page.north west) rectangle (current page.south east);
  \node[anchor=north west, text width=14cm] at ($(current page.north west)+(2.8cm,-4cm)$) {
    {\fontsize{11}{13}\selectfont\color{accent}\textls[100]{SPRINT-REPORT}}\\[18pt]
    {\fontsize{36}{42}\selectfont\bfseries\color{white}12 Stunden.\\[4pt]70 Commits.\\[4pt]24 Deliverables.}\\[24pt]
    {\fontsize{14}{18}\selectfont\color{lightondark}Was in einer Nacht entsteht, wenn ein\\Mensch und ein KI-System zusammenarbeiten.}\\[60pt]
    {\fontsize{11}{14}\selectfont\color{accent}Zeitraum}\\[4pt]
    {\fontsize{13}{16}\selectfont\color{white}5. Februar 2026, 21:00 -- 6. Februar 2026, 09:20}\\[20pt]
    {\fontsize{11}{14}\selectfont\color{accent}Methode}\\[4pt]
    {\fontsize{13}{16}\selectfont\color{white}1 Mensch + 1 KI-Agent + 5 Sub-Agenten (parallel)}
  };
  \node[anchor=south west, text width=14cm] at ($(current page.south west)+(2.8cm,3cm)$) {
    {\fontsize{10}{13}\selectfont\color{subtitle}Florian Ziesche $\cdot$ florian@ziesche.co $\cdot$ +1\,347\,740\,1465}\\[4pt]
    {\fontsize{9}{12}\selectfont\color{subtitle}KI-Agent: Mia (OpenClaw) $\cdot$ Claude Opus 4.6}
  };
\end{tikzpicture}

\clearpage

% ============================================
% KPIs AUF EINEN BLICK
% ============================================
\section{KPIs auf einen Blick}
\ssubtitle{Messbare Ergebnisse in 12 Stunden}

\begin{kpibox}{Output-Metriken}
\begin{tabularx}{\textwidth}{X r}
\toprule
\textbf{Git Commits} & \textbf{70} \\
\midrule
\rowcolor{lightgray} LaTeX-Reports erstellt (PDF) & 9 Dokumente, 70+ Seiten \\
PowerPoint-Decks erstellt & 2 Decks (9 + 40 Slides) \\
\rowcolor{lightgray} Interaktive HTML-Demos & 6 (inkl. Index-Seite) \\
CNC Planer Pro Updates & v17 $\rightarrow$ v18 (6.117 Zeilen) \\
\rowcolor{lightgray} Forschungsdokumente (Markdown) & 8 Research-Docs (42 KB) \\
LaTeX-Code geschrieben & 4.475 Zeilen \\
\rowcolor{lightgray} HTML/CSS/JS geschrieben & 8.517 Zeilen \\
\textbf{Gesamte Deliverables} & \textbf{24 fertige Dateien} \\
\bottomrule
\end{tabularx}
\end{kpibox}

\vspace{8pt}

\begin{kpibox}{Qualitäts-Metriken}
\begin{tabularx}{\textwidth}{X r}
\toprule
LaTeX-Kompilierung: Overfull Warnings & \textbf{0 von 9 Reports} \\
\rowcolor{lightgray} JavaScript-Syntaxfehler & 0 (node --check verifiziert) \\
HTML-Tag-Balancierung & 100\,\% (502 DIV, 275 SPAN) \\
\rowcolor{lightgray} Event-Handler definiert & 25/25 (100\,\%) \\
\bottomrule
\end{tabularx}
\end{kpibox}


% ============================================
% TIMELINE
% ============================================
\clearpage
\section{Was ist entstanden}
\ssubtitle{Chronologische Übersicht aller Deliverables}

\subsection{Phase 1: CNC Planer Pro v18 (21:00--04:30)}

\begin{tabularx}{\textwidth}{l X}
\toprule
\rowcolor{lightgray} \textbf{Uhrzeit} & \textbf{Ergebnis} \\
\midrule
21:00--23:00 & 8 Research-Dokumente für v18 Industrial Redesign \\
\rowcolor{lightgray} 23:00--00:30 & MBS-Angebot Line-by-Line Analyse (34 KB) \\
00:30--02:00 & Design System + Entwicklungsprozess + Komponenten-Doku \\
\rowcolor{lightgray} 02:00--03:00 & Angebots-Sektion: DEFormatter, dynamische Tabelle, Kunden-Sync \\
03:00--04:30 & Hardening: 16 Detail-IDs, Print CSS, 3. Demo-Bauteil, Settings \\
\bottomrule
\end{tabularx}

\vspace{6pt}

\textbf{Ergebnis:} CNC Planer Pro v18 -- 6.117 Zeilen, 10 Tabs, 3 Demo-Bauteile, vollständige Zuschlagskalkulation, professionelles Angebotslayout, Fertigungsanweisung, Nachkalkulation.

\subsection{Phase 2: Reports \& Analyse (04:30--07:30)}

\begin{tabularx}{\textwidth}{c l r}
\toprule
\rowcolor{lightgray} \textbf{Nr} & \textbf{Report} & \textbf{Seiten} \\
\midrule
1 & MBS vs CNC Planer Pro -- Vergleichsreport & 10 \\
\rowcolor{lightgray} 2 & Kalkulations-Vergleich 6 Positionen & 8 \\
3 & Kalkulations-Vergleich 3 Demo-Bauteile & 7 \\
\rowcolor{lightgray} 4 & HK-Vergleich alle 6 Positionen mit Zeichnungen & 11 \\
5 & Strategie-Report: USP und Positionierung & 11 \\
\rowcolor{lightgray} 6 & Vergleichsreport v2 (3 Bauteile + 6 Positionen) & 14 \\
7 & Strategie-Report v2 (mit Kalkulationsdaten belegt) & 10 \\
\rowcolor{lightgray} 8 & Kalkulationsausgabe 3 Bauteile (mit Mitigationen) & 14 \\
\midrule
& \textbf{Gesamt} & \textbf{85 Seiten} \\
\bottomrule
\end{tabularx}

\vspace{4pt}

Jeder Report: Helvetica Neue, professionelles Layout, technische Zeichnungen eingebettet, 0 Compile-Warnings.

\subsection{Phase 3: KI-Mittelstand Sales Package (07:30--09:00)}

\begin{tabularx}{\textwidth}{c l l}
\toprule
\rowcolor{lightgray} \textbf{Nr} & \textbf{Deliverable} & \textbf{Format} \\
\midrule
1 & KI-Umsetzung Mittelstand Report & LaTeX/PDF, 8 Seiten \\
\rowcolor{lightgray} 2 & KI Sales Deck & PowerPoint, 9 Slides \\
3 & KI-Schulungsdeck (5 Module) & PowerPoint, 40 Slides \\
\bottomrule
\end{tabularx}

\subsection{Phase 4: Interaktive Live-Demos (08:30--09:10)}

5 Demos parallel durch Sub-Agenten erstellt (je 90--180 Sekunden Buildzeit):

\begin{tabularx}{\textwidth}{l X r}
\toprule
\rowcolor{lightgray} \textbf{Demo} & \textbf{Funktion} & \textbf{EUR/Mo} \\
\midrule
E-Mail-Triage & 3 Mail-Sets, KI-Analyse, Antwort-Entwürfe & 1.845 \\
\rowcolor{lightgray} Branchen-Briefing & Live-Aktualisierung, Sparklines, aufklappbar & 1.540 \\
Meisterwissen & Chat mit Typing-Animation, 3 CNC-Szenarien & 40.000+/a \\
\rowcolor{lightgray} Qualitätsprotokoll & Editierbare Messwerte, Toleranzprüfung & 608 \\
Meeting-Protokoll & Audio-Waveform, 4-Step Processing, To-Dos & 535 \\
\midrule
\rowcolor{lightgray} \textbf{Gesamtpotenzial} & \textbf{Monatliche Ersparnis pro Betrieb} & \textbf{$\sim$4.500} \\
\bottomrule
\end{tabularx}


% ============================================
% WAS DAS BEDEUTET
% ============================================
\clearpage
\section{Was das bedeutet}
\ssubtitle{Einordnung für konventionelle Arbeitsweise}

\subsection{Zeitäquivalent}

\begin{tabularx}{\textwidth}{X r r}
\toprule
\rowcolor{lightgray} \textbf{Aufgabe} & \textbf{Konventionell} & \textbf{In dieser Nacht} \\
\midrule
Software-Entwicklung (v18, 6.117 LOC) & 2--3 Wochen & 7 Stunden \\
\rowcolor{lightgray} 9 LaTeX-Reports (85 Seiten, mit Grafiken) & 5--7 Arbeitstage & 3 Stunden \\
5 interaktive HTML-Demos & 3--5 Arbeitstage & 10 Minuten \\
\rowcolor{lightgray} 2 PowerPoint-Decks (49 Slides) & 2--3 Arbeitstage & 40 Minuten \\
8 Research-Dokumente (42 KB) & 2--3 Arbeitstage & 2 Stunden \\
\midrule
\rowcolor{lightgray} \textbf{Gesamt (konventionell)} & \textbf{4--6 Wochen} & \textbf{12 Stunden} \\
\bottomrule
\end{tabularx}

\begin{highlightbox}
\textbf{Faktor 15--20×} -- Das ist keine Effizienzsteigerung. Das ist eine andere Kategorie von Produktivität. Ein Mensch, der weiß wie man KI orchestriert, ersetzt ein ganzes Team für Nacht-Sprints.
\end{highlightbox}

\subsection{Was das ermöglicht hat}

\begin{itemize}
  \item \textbf{Parallele Agenten:} 5 Sub-Agenten bauten gleichzeitig 5 Demos. Gesamtzeit: 3 Minuten statt 5 $\times$ 45 Minuten.
  \item \textbf{Template-Compound:} Jeder Report nutzt das gleiche LaTeX-Template. Report Nr.~9 wurde in 4 Minuten erstellt, weil alle Patterns schon standen.
  \item \textbf{Echtzeit-Validierung:} Jede LaTeX-Kompilierung wurde automatisch auf Warnings geprüft. Null manuelle Korrekturrunden.
  \item \textbf{Kontexttiefe:} MBS-Angebotsdaten, Zeichnungen, Werkstoffpreise -- alles im Arbeitskontext. Kein Copy-Paste zwischen Tools.
\end{itemize}


% ============================================
% VOLLSTÄNDIGE ASSET-LISTE
% ============================================
\section{Vollständige Asset-Liste}
\ssubtitle{24 Deliverables in 12 Stunden}

\begin{tabularx}{\textwidth}{c l l r}
\toprule
\rowcolor{lightgray} & \textbf{Asset} & \textbf{Format} & \textbf{Umfang} \\
\midrule
1 & CNC Planer Pro v18 & HTML & 6.117 LOC \\
\rowcolor{lightgray} 2 & MBS vs CNC Planer Vergleich & PDF & 10 S. \\
3 & Kalkulations-Vergleich 6 Positionen & PDF & 8 S. \\
\rowcolor{lightgray} 4 & Kalkulations-Vergleich 3 Bauteile & PDF & 7 S. \\
5 & HK-Vergleich mit Zeichnungen & PDF & 11 S. \\
\rowcolor{lightgray} 6 & Strategie-Report USP & PDF & 11 S. \\
7 & Vergleichsreport v2 & PDF & 14 S. \\
\rowcolor{lightgray} 8 & Strategie v2 & PDF & 10 S. \\
9 & Kalkulationsausgabe + Mitigationen & PDF & 14 S. \\
\rowcolor{lightgray} 10 & KI-Mittelstand Report & PDF & 8 S. \\
11 & KI Sales Deck & PPTX & 9 Slides \\
\rowcolor{lightgray} 12 & KI-Schulungsdeck & PPTX & 40 Slides \\
13 & Demo: E-Mail-Triage & HTML & 391 LOC \\
\rowcolor{lightgray} 14 & Demo: Branchen-Briefing & HTML & 382 LOC \\
15 & Demo: Meisterwissen & HTML & 581 LOC \\
\rowcolor{lightgray} 16 & Demo: Qualitätsprotokoll & HTML & 341 LOC \\
17 & Demo: Meeting-Protokoll & HTML & 615 LOC \\
\rowcolor{lightgray} 18 & Demo-Index & HTML & 90 LOC \\
19 & Design System Doku & MD & 14 KB \\
\rowcolor{lightgray} 20 & Development Process & MD & 9 KB \\
21 & 8 Research-Dokumente & MD & 42 KB \\
\rowcolor{lightgray} 22 & MBS Line-by-Line Analyse & MD & 34 KB \\
23 & Nutzerdokumentation & MD & 11 KB \\
\rowcolor{lightgray} 24 & One-Pager CNC Planer Pro & MD & 5 KB \\
\bottomrule
\end{tabularx}


\vfill

\begin{center}
\small\textcolor{subtitle}{
Sprint-Report $\cdot$ 5.--6. Februar 2026\\
Florian Ziesche $\cdot$ florian@ziesche.co $\cdot$ +1\,347\,740\,1465\\
KI-Agent: Mia (OpenClaw/Claude Opus 4.6)
}
\end{center}

\end{document}
