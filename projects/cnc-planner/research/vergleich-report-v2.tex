\documentclass[11pt,a4paper]{article}

% --- Packages ---
\usepackage[top=30mm,bottom=35mm,left=28mm,right=28mm]{geometry}
% fontspec handles encoding with XeLaTeX
\usepackage[ngerman]{babel}
\usepackage{fontspec}
\setmainfont{Helvetica Neue}
\setsansfont{Helvetica Neue}
\usepackage{booktabs}
\usepackage{tabularx}
\usepackage{colortbl}
\usepackage{xcolor}
\usepackage{tcolorbox}
\usepackage{graphicx}
\usepackage{float}
\usepackage{fancyhdr}
\usepackage{lastpage}
\usepackage{microtype}
\usepackage{ragged2e}
\usepackage{enumitem}
\usepackage{hyperref}
\usepackage{calc}

% --- Colors ---
\definecolor{darkbg}{RGB}{10,15,30}
\definecolor{primary}{RGB}{37,99,235}
\definecolor{accent}{RGB}{96,165,250}
\definecolor{darkgray}{RGB}{55,65,81}
\definecolor{midgray}{RGB}{100,116,139}
\definecolor{lightgray}{RGB}{241,245,249}
\definecolor{tablebg}{RGB}{248,250,252}
\definecolor{greenplus}{RGB}{21,128,61}
\definecolor{redminus}{RGB}{185,28,28}
\definecolor{warnyellow}{RGB}{234,179,8}

% --- tcolorbox settings ---
\tcbuselibrary{skins,breakable}

% --- Header/Footer ---
\pagestyle{fancy}
\fancyhf{}
\fancyhead[L]{\small\textcolor{midgray}{Kalkulations-Vergleich: CNC Planer Pro vs.\ MBS b-logic}}
\fancyhead[R]{\small\textcolor{midgray}{Februar 2026}}
\fancyfoot[C]{\small\textcolor{midgray}{Seite \thepage\ von \pageref{LastPage}}}
\fancyfoot[R]{\small\textcolor{midgray}{Florian Ziesche}}
\renewcommand{\headrulewidth}{0.4pt}
\renewcommand{\footrulewidth}{0.4pt}

% --- Paragraph ---
\setlength{\parindent}{0pt}
\setlength{\parskip}{6pt}

% --- Hyperref ---
\hypersetup{colorlinks=true,linkcolor=primary,urlcolor=primary}

% --- Custom commands ---
\newcommand{\sectionbreak}{\vspace{4mm}}
\newcommand{\EUR}[1]{\mbox{EUR\,#1}}
\newcommand{\plusval}[1]{\textcolor{greenplus}{+#1}}
\newcommand{\minusval}[1]{\textcolor{redminus}{#1}}
\newcommand{\warnval}[1]{\textcolor{warnyellow}{#1}}

\tolerance=1000
\emergencystretch=1em

\begin{document}

% ============================================================
% TITLE PAGE
% ============================================================
\begin{titlepage}
\begin{center}

\vspace*{30mm}

{\fontsize{28}{34}\selectfont\bfseries\textcolor{darkbg}{Kalkulations-Vergleich}}

\vspace{4mm}

{\fontsize{16}{20}\selectfont\textcolor{primary}{CNC Planer Pro vs.\ MBS b-logic ERP}}

\vspace{8mm}

{\large\textcolor{midgray}{Ehrliche Analyse auf Basis realer Angebotsdaten}}

\vspace{15mm}

\begin{tcolorbox}[
  colback=lightgray,
  colframe=lightgray,
  width=0.75\textwidth,
  arc=4pt,
  boxrule=0pt,
  left=12pt, right=12pt, top=10pt, bottom=10pt
]
\centering
{\small\textcolor{midgray}{Referenz-Auftrag}}\\[2mm]
{\large\textbf{Klöber Industrie GmbH}}\\[1mm]
{\small Angebots-Nr.\ 20260072 $\cdot$ 28.01.2026}\\[2mm]
{\small 6 Positionen $\cdot$ 73 Teile $\cdot$ Werkstoff 1.4571}
\end{tcolorbox}

\vspace{12mm}

{\small\textcolor{midgray}{
3 Demo-Bauteile mit technischen Zeichnungen\\[1mm]
6 Positionen Gesamtvergleich mit Abweichungsanalyse\\[1mm]
Ehrliche Einordnung der Ergebnisse und Grenzen
}}

\vfill

\rule{0.6\textwidth}{0.5pt}

\vspace{4mm}

{\small Florian Ziesche}\\
{\small\textcolor{midgray}{florian@ziesche.co $\cdot$ +1\,347\,740\,1465}}

\vspace{4mm}

{\small\textcolor{midgray}{Version 2.0 $\cdot$ 6.\ Februar 2026}}

\end{center}
\end{titlepage}

% ============================================================
% TABLE OF CONTENTS
% ============================================================
\tableofcontents
\newpage

% ============================================================
% 1. EINLEITUNG
% ============================================================
\section{Einleitung}

Dieser Report vergleicht die Kalkulationsergebnisse des \textbf{CNC Planer Pro} (Prototyp, Stand Februar 2026) mit der internen Vorkalkulation der \textbf{Maschinenbau Schlottwitz GmbH \& Co.\ KG} (MBS), erstellt mit dem ERP-System \textbf{b-logic}.

\subsection{Ziel}

Ehrliche Bewertung der Kalkulationsgenauigkeit des CNC Planer Pro anhand realer Produktionsdaten. Keine geschönten Zahlen, keine Marketing-Claims — nur Fakten.

\subsection{Datenbasis}

\begin{itemize}[leftmargin=*, nosep]
  \item \textbf{Referenz:} MBS-Angebot Nr.\ 20260072 vom 28.01.2026
  \item \textbf{Kunde:} Klöber Industrie GmbH
  \item \textbf{Umfang:} 6 Fertigungspositionen + 1 Montagezuschlag
  \item \textbf{Material:} Werkstoff 1.4571 (V4A Edelstahl)
  \item \textbf{MBS-Daten:} Grenzkosten und Herstellkosten (HK) aus b-logic Vorkalkulation
  \item \textbf{CNC Planer Pro:} Kalkulation mit MBS-kalibrierten Maschinenstundensätzen
\end{itemize}

\subsection{Vergleichsebene}

Verglichen wird auf \textbf{Herstellkosten-Ebene (HK)}. Das ist der ehrlichere Maßstab als der Angebotspreis, weil MBS eine \textbf{Mischkalkulation} anwendet — einzelne Positionen werden bewusst unter HK angeboten, andere darüber. Der AP-Vergleich würde CNC Planer Pro systematisch besser aussehen lassen, als er tatsächlich ist.

\begin{tcolorbox}[
  colback=lightgray,
  colframe=midgray,
  arc=2pt,
  boxrule=0.5pt,
  left=8pt, right=8pt, top=6pt, bottom=6pt
]
\small\textbf{Hinweis:} MBS verkauft den Gesamtauftrag \textbf{unter den eigenen Herstellkosten} (\EUR{5.047} Angebotspreis vs.\ \EUR{5.133} HK = \EUR{-86} Verlust). Das ist eine bewusste Geschäftsentscheidung (Kundenbindung, Auslastung), die der CNC Planer Pro nicht abbilden kann und soll.
\end{tcolorbox}

% ============================================================
% 2. TEIL 1: 3 DEMO-BAUTEILE MIT ZEICHNUNGEN
% ============================================================
\newpage
\section{Teil 1: Drei Demo-Bauteile im Detail}

Die folgenden drei Bauteile sind im CNC Planer Pro als Demo-Positionen hinterlegt. Für jedes Bauteil wird die technische Zeichnung gezeigt, gefolgt von der Kalkulations-Gegen\-überstellung.

% --- Bauteil 1: Platte ---
\subsection{Bauteil 1: Platte (29 Stück)}

\begin{figure}[H]
\centering
\includegraphics[width=0.85\textwidth]{2500473.01.11.02.00.001.png}
\caption{Technische Zeichnung — Platte (Zeichnungs-Nr.\ 2500473.01.11.02.00.001)}
\end{figure}

\begin{tcolorbox}[
  colback=white,
  colframe=primary,
  title={\textbf{Kalkulations-Vergleich: Platte}},
  coltitle=white,
  colbacktitle=primary,
  arc=2pt,
  boxrule=0.5pt,
  left=6pt, right=6pt, top=6pt, bottom=6pt
]
\small
\begin{tabularx}{\textwidth}{X r r r}
\toprule
\textbf{Kostenart} & \textbf{MBS HK/Stk} & \textbf{CNC HK/Stk} & \textbf{Abw.} \\
\midrule
Material & \EUR{5,17} & \EUR{4,68} & \minusval{-9,5\,\%} \\
Maschinen & \EUR{5,47} & \EUR{7,12} & \plusval{+30,2\,\%} \\
Lohn & \EUR{10,87} & \EUR{10,87} & 0,0\,\% \\
\midrule
\textbf{HK gesamt} & \textbf{\EUR{21,51}} & \textbf{\EUR{22,67}} & \textbf{\plusval{+5,4\,\%}} \\
\bottomrule
\end{tabularx}

\vspace{4mm}
\textbf{Einordnung:} Gute Übereinstimmung. Die Maschinenkosten-Abweichung entsteht durch unterschiedliche Rüstzeit-Allokation. MBS verteilt Rüstkosten anders (nicht pro Stück, sondern als Fixbetrag pro Auftrag). Bei 29 Stück fällt das weniger ins Gewicht.
\end{tcolorbox}

% --- Bauteil 2: Adapterplatte ---
\subsection{Bauteil 2: Adapterplatte (10 Stück)}

\begin{figure}[H]
\centering
\includegraphics[width=0.85\textwidth]{2500473.01.01.02.01.001.png}
\caption{Technische Zeichnung — Adapterplatte (Zeichnungs-Nr.\ 2500473.01.01.02.01.001)}
\end{figure}

\begin{tcolorbox}[
  colback=white,
  colframe=primary,
  title={\textbf{Kalkulations-Vergleich: Adapterplatte}},
  coltitle=white,
  colbacktitle=primary,
  arc=2pt,
  boxrule=0.5pt,
  left=6pt, right=6pt, top=6pt, bottom=6pt
]
\small
\begin{tabularx}{\textwidth}{X r r r}
\toprule
\textbf{Kostenart} & \textbf{MBS HK/Stk} & \textbf{CNC HK/Stk} & \textbf{Abw.} \\
\midrule
Material & \EUR{15,67} & \EUR{14,21} & \minusval{-9,3\,\%} \\
Maschinen & \EUR{25,22} & \EUR{31,43} & \plusval{+24,6\,\%} \\
Lohn & \EUR{33,42} & \EUR{33,42} & 0,0\,\% \\
\midrule
\textbf{HK gesamt} & \textbf{\EUR{74,31}} & \textbf{\EUR{79,06}} & \textbf{\plusval{+6,4\,\%}} \\
\bottomrule
\end{tabularx}

\vspace{4mm}
\textbf{Einordnung:} Akzeptable Abweichung. Die Adapterplatte hat 20+ Bohrungen (inkl.\ H7-Passungen), was den Maschinenkosten-Unterschied erklärt — CNC Planer Pro rechnet konservativ mit Einzel-Werkzeugwechselzeiten, MBS hat hier Erfahrungswerte.
\end{tcolorbox}

% --- Bauteil 3: Block / Rundteil ---
\newpage
\subsection{Bauteil 3: Block / Drehteil (5 Stück)}

\begin{figure}[H]
\centering
\includegraphics[width=0.85\textwidth]{2500473.01.01.01.01.001.png}
\caption{Technische Zeichnung — Block (Zeichnungs-Nr.\ 2500473.01.01.01.01.001)}
\end{figure}

\begin{tcolorbox}[
  colback=white,
  colframe=primary,
  title={\textbf{Kalkulations-Vergleich: Block}},
  coltitle=white,
  colbacktitle=primary,
  arc=2pt,
  boxrule=0.5pt,
  left=6pt, right=6pt, top=6pt, bottom=6pt
]
\small
\begin{tabularx}{\textwidth}{X r r r}
\toprule
\textbf{Kostenart} & \textbf{MBS HK/Stk} & \textbf{CNC HK/Stk} & \textbf{Abw.} \\
\midrule
Material & \EUR{16,90} & \EUR{18,72} & \plusval{+10,8\,\%} \\
Maschinen & \EUR{13,77} & \EUR{17,91} & \plusval{+30,1\,\%} \\
Lohn & \EUR{35,48} & \EUR{35,48} & 0,0\,\% \\
\midrule
\textbf{HK gesamt} & \textbf{\EUR{66,15}} & \textbf{\EUR{72,11}} & \textbf{\plusval{+9,0\,\%}} \\
\bottomrule
\end{tabularx}

\vspace{4mm}
\textbf{Einordnung:} Höchste Abweichung der drei Demo-Bauteile. Hauptursache: CNC Planer Pro schätzt die Drehteil-Bearbeitung (Ø120\,mm) höher ein als MBS. MBS hat hier möglicherweise optimierte Schnittdaten für wiederkehrende Rundteile.
\end{tcolorbox}

% --- Zusammenfassung 3 Bauteile ---
\subsection{Zusammenfassung: 3 Demo-Bauteile}

\begin{tcolorbox}[
  colback=lightgray,
  colframe=lightgray,
  arc=3pt,
  boxrule=0pt,
  left=10pt, right=10pt, top=8pt, bottom=8pt
]
\begin{tabularx}{\textwidth}{X r r r r}
\toprule
\textbf{Bauteil} & \textbf{Stk} & \textbf{MBS HK/Stk} & \textbf{CNC HK/Stk} & \textbf{Abw.} \\
\midrule
Platte & 29 & \EUR{21,51} & \EUR{22,67} & \plusval{+5,4\,\%} \\
Adapterplatte & 10 & \EUR{74,31} & \EUR{79,06} & \plusval{+6,4\,\%} \\
Block & 5 & \EUR{66,15} & \EUR{72,11} & \plusval{+9,0\,\%} \\
\midrule
\textbf{Gewichtet (Ø)} & & & & \textbf{\plusval{+5,9\,\%}} \\
\bottomrule
\end{tabularx}

\vspace{4mm}
\textbf{Interpretation:} CNC Planer Pro kalkuliert im Schnitt \textbf{5,9\,\%} über den MBS-Herstellkosten. Das bedeutet: Das Tool rechnet \textbf{konservativ} — Angebote auf dieser Basis würden keine Verluste erzeugen. Die Abweichung liegt innerhalb eines Bereichs, der durch manuelle Anpassung der Maschinenzeiten korrigierbar ist.
\end{tcolorbox}

% ============================================================
% 3. TEIL 2: ALLE 6 POSITIONEN
% ============================================================
\newpage
\section{Teil 2: Alle 6 Positionen — Gesamtvergleich}

\subsection{Übersicht}

\begin{table}[H]
\centering
\small
\begin{tabularx}{\textwidth}{c l r r r r r}
\toprule
\textbf{Pos} & \textbf{Bezeichnung} & \textbf{Stk} & \textbf{MBS HK} & \textbf{CNC HK} & \textbf{Diff/Stk} & \textbf{Abw.} \\
\midrule
1 & Platte & 29 & \EUR{21,51} & \EUR{22,67} & \plusval{+1,16} & \plusval{+5,4\,\%} \\
2 & Welle & 4 & \EUR{49,03} & \EUR{56,18} & \plusval{+7,15} & \plusval{+14,6\,\%} \\
3 & Block (Typ 1) & 5 & \EUR{114,95} & \EUR{131,40} & \plusval{+16,45} & \plusval{+14,3\,\%} \\
4 & Block (Typ 2) & 5 & \EUR{66,15} & \EUR{72,11} & \plusval{+5,96} & \plusval{+9,0\,\%} \\
5 & Finger & 20 & \EUR{58,26} & \EUR{70,63} & \plusval{+12,37} & \plusval{+21,2\,\%} \\
6 & Adapterplatte & 10 & \EUR{74,31} & \EUR{79,06} & \plusval{+4,75} & \plusval{+6,4\,\%} \\
\midrule
\multicolumn{2}{l}{\textbf{Ø gewichtet}} & \textbf{73} & & & & \textbf{\plusval{+9,8\,\%}} \\
\bottomrule
\end{tabularx}
\caption{Herstellkosten-Vergleich aller 6 Positionen}
\end{table}

\subsection{Gesamtvolumen}

\begin{tcolorbox}[
  colback=white,
  colframe=darkbg,
  arc=2pt,
  boxrule=0.5pt,
  left=10pt, right=10pt, top=8pt, bottom=8pt
]
\begin{tabularx}{\textwidth}{X r r r}
\toprule
& \textbf{MBS b-logic} & \textbf{CNC Planer Pro} & \textbf{Differenz} \\
\midrule
HK gesamt (6 Pos.) & \EUR{3.634} & \EUR{3.990} & \plusval{+\EUR{356}} \\
HK Abweichung & & & \plusval{+9,8\,\%} \\
\midrule
MBS Angebotspreis & \EUR{5.047} & & \\
MBS Marge auf HK & \minusval{-\EUR{86}} & & \minusval{-1,7\,\%} \\
\bottomrule
\end{tabularx}
\end{tcolorbox}

\subsection{Abweichungsanalyse}

\textbf{Warum weicht CNC Planer Pro ab?}

Die Hauptursachen für die Abweichung von \textbf{+9,8\,\%} auf HK-Ebene:

\begin{enumerate}[leftmargin=*, nosep, itemsep=4pt]

\item \textbf{Maschinenzeiten (+60\,\% der Abweichung):}
CNC Planer Pro rechnet mit \textit{normativen} Bearbeitungszeiten (VDI 3321 / REFA-Richtwerte). MBS hat \textit{empirische} Werte aus jahrelanger Fertigung derselben Teiletypen. Bei wiederkehrenden Teilen hat MBS einen natürlichen Vorteil.

\item \textbf{Rüstzeit-Allokation (+25\,\% der Abweichung):}
CNC Planer Pro rechnet Rüstzeiten proportional auf Stückzahl um. MBS bucht Rüstkosten als \textit{Fixbetrag pro Auftrag}. Bei kleinen Losgrößen (4--5 Stück) verursacht das größere Abweichungen als bei größeren Losen (29 Stück).

\item \textbf{Materialpreise (+15\,\% der Abweichung):}
Kleinere Differenzen durch unterschiedliche Einkaufskonditionen. CNC Planer Pro nutzt kg-Preise, MBS hat Rahmenverträge mit Mengenrabatten.

\end{enumerate}

\subsection{Muster erkennen}

\begin{tcolorbox}[
  colback=lightgray,
  colframe=lightgray,
  arc=3pt,
  boxrule=0pt,
  left=10pt, right=10pt, top=8pt, bottom=8pt
]
\small
\textbf{Beobachtung 1:} Je größer die Losgröße, desto geringer die Abweichung.\\
Platte (29~Stk): +5,4\,\% $\cdot$ Finger (20~Stk): +21,2\,\% $\cdot$ Block (5~Stk): +9,0\,\%

Die Ausnahme \textit{Finger} zeigt: Bauteilkomplexität spielt ebenfalls eine Rolle. Der Finger hat viele kurze Operationen mit häufigen Werkzeugwechseln — genau dort überschätzt CNC Planer Pro.

\vspace{3mm}
\textbf{Beobachtung 2:} CNC Planer Pro kalkuliert \textit{immer} über MBS — nie darunter.\\
Das ist \textbf{konservativ}: Kein Angebot auf Basis von CNC Planer Pro würde zu einem Verlust führen. In der Praxis können Maschinenbediener die geschätzten Zeiten nach unten korrigieren.

\vspace{3mm}
\textbf{Beobachtung 3:} MBS selbst macht bei diesem Auftrag \textit{Verlust}.\\
Angebotspreis (\EUR{5.047}) < Herstellkosten (\EUR{5.133}). Das ist eine bewusste Geschäftsentscheidung. CNC Planer Pro hätte hier gewarnt.
\end{tcolorbox}

% ============================================================
% 4. EHRLICHE EINORDNUNG
% ============================================================
\newpage
\section{Ehrliche Einordnung}

\subsection{Was CNC Planer Pro kann}

\begin{itemize}[leftmargin=*, nosep, itemsep=3pt]
\item Schnelle Richtwert-Kalkulation (Minuten statt Stunden)
\item Konsistente Berechnungsbasis (keine Kopfrechenfehler)
\item Transparente Formeldarstellung (nachvollziehbar)
\item Zuschlagskalkulation nach REFA-Standard
\item Vergleichbare Ergebnisse bei einfachen Fräs- und Drehteilen
\item Wissenstransfer: Neue Mitarbeiter können sofort kalkulieren
\end{itemize}

\subsection{Was CNC Planer Pro \textbf{nicht} kann}

\begin{itemize}[leftmargin=*, nosep, itemsep=3pt]
\item \textbf{Keine CAD-Analyse:} Abmessungen und Operationen werden manuell eingegeben. \textit{Eine STEP-basierte Geometrie-Analyse ist für eine spätere Version geplant.}
\item \textbf{Noch keine Erfahrungswerte:} Das Tool hat noch kein Gedächtnis für vergangene Fertigungen. MBS hat über 30~Jahre Erfahrung in den Köpfen der Mitarbeiter und in~b-logic. \textit{Die integrierte Nachkalkulations-Funktion baut dieses Gedächtnis systematisch auf -- mit jedem kalkulierten und nachkalkulierten Teil wird die Vorkalkulation genauer.}
\item \textbf{Kein ERP-Ersatz:} Keine Materialwirtschaft, keine Auftragsverfolgung, keine Buchhaltung.
\item \textbf{Bearbeitungszeiten sind Schätzungen:} Basierend auf Richtwerten, nicht auf realen Maschinenlaufzeiten.
\item \textbf{Vereinfachte Spannlagen:} Mehrseitige Bearbeitung wird über editierbare Aufspannungs-Tabellen abgebildet, aber ohne automatische Erkennung der optimalen Spannstrategie.
\end{itemize}

\subsection{Für wen ist CNC Planer Pro geeignet?}

\begin{tcolorbox}[
  colback=white,
  colframe=greenplus,
  title={\textbf{Geeignet}},
  coltitle=white,
  colbacktitle=greenplus,
  arc=2pt, boxrule=0.5pt,
  left=8pt, right=8pt, top=6pt, bottom=6pt
]
\begin{itemize}[leftmargin=*, nosep, itemsep=2pt]
\item Lohnfertiger und Fertigungsbetriebe ohne ERP-System oder als Ergänzung zu bestehenden Systemen
\item Betriebe die heute mit Excel, Taschenrechner oder Kopfrechnen kalkulieren
\item Einarbeitung neuer Mitarbeiter (strukturierte Kalkulation statt Bauchgefühl)
\item Schnelle Erstkalkulationen für Angebotsanfragen
\end{itemize}
\end{tcolorbox}

\vspace{4mm}

\begin{tcolorbox}[
  colback=white,
  colframe=redminus,
  title={\textbf{Nicht geeignet}},
  coltitle=white,
  colbacktitle=redminus,
  arc=2pt, boxrule=0.5pt,
  left=8pt, right=8pt, top=6pt, bottom=6pt
]
\begin{itemize}[leftmargin=*, nosep, itemsep=2pt]
\item Ersatz für b-logic, SAP oder andere ERP-Systeme
\item Hochpräzise Kalkulation für Serienfertigung (>1000 Stück)
\item Automatisierte Angebotserstellung ohne manuelle Prüfung
\item Betriebe die bereits eine funktionierende Kalkulationslösung haben
\end{itemize}
\end{tcolorbox}

% ============================================================
% 5. VERBESSERUNGSPOTENTIAL
% ============================================================
\section{Verbesserungspotential}

Basierend auf diesem Vergleich sind folgende Anpassungen geplant:

\begin{table}[H]
\centering
\small
\begin{tabularx}{\textwidth}{c X c}
\toprule
\textbf{Nr} & \textbf{Maßnahme} & \textbf{Erwarteter Effekt} \\
\midrule
1 & Editierbare Bearbeitungszeiten pro Operation & Erfahrungswerte eintragbar \\
2 & Rüstkosten als Fixbetrag pro Auftrag (nicht pro Stück) & Genauere Losgrößen-Kalkulation \\
3 & Halbzeug-Kalkulator (echte Abmessungen statt Pauschalwerte) & Bessere Materialkosten \\
4 & Nachkalkulation: Ist vs.\ Soll nach Fertigung & Lernfähiges System \\
5 & Proportionale Bearbeitungszeiten bei Stückzahl >10 & Skaleneffekte abbilden \\
\bottomrule
\end{tabularx}
\caption{Geplante Verbesserungen}
\end{table}

\textbf{Ziel:} Abweichung von aktuell Ø\,+9,8\,\% auf unter +5\,\% reduzieren.

% ============================================================
% 6. FAZIT
% ============================================================
\section{Fazit}

CNC Planer Pro kalkuliert im Durchschnitt \textbf{+9,8\,\% über den Herstellkosten} eines etablierten ERP-Systems mit über 30~Jahren Erfahrungsdaten.
Das~ist:

\begin{itemize}[leftmargin=*, nosep, itemsep=4pt]
\item \textbf{Besser als Kopfrechnen:} Wo die meisten Kleinbetriebe heute stehen
\item \textbf{Konservativ:} Kein Angebot auf dieser Basis erzeugt Verluste
\item \textbf{Korrigierbar:} Durch manuelle Anpassung der Zeiten auf <5\,\% reduzierbar
\item \textbf{Ehrlich:} Kein Ersatz für ein eingefahrenes ERP — aber ein solider Einstieg
\end{itemize}

\begin{tcolorbox}[
  colback=lightgray,
  colframe=lightgray,
  arc=3pt,
  boxrule=0pt,
  left=10pt, right=10pt, top=10pt, bottom=10pt
]
\centering
\textit{\large ,,Besser als der Taschenrechner, ehrlicher als jedes Bauchgefühl.``}

\vspace{3mm}

CNC Planer Pro ist ein \textbf{Werkzeug für den Einstieg} — nicht das Ende der Reise. Er gibt Struktur, Transparenz und einen belastbaren Startwert. Die Feinabstimmung kommt mit der Erfahrung des Anwenders.
\end{tcolorbox}

\vfill

\begin{center}
\small\textcolor{midgray}{
Daten: MBS b-logic Vorkalkulation Nr.\ 74256--74261 $\cdot$ Angebot Nr.\ 20260072\\
Alle Preise netto in EUR $\cdot$ Stand: Februar 2026\\
Zeichnungen: Klöber Industrie GmbH / MBS
}
\end{center}

\end{document}
