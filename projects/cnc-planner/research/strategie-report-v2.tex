\documentclass[a4paper,11pt]{article}

% Fonts
\usepackage{fontspec}
\setmainfont{Helvetica Neue}[
  BoldFont=Helvetica Neue Bold,
  ItalicFont=Helvetica Neue Italic
]

% Layout
\usepackage[top=30mm, bottom=35mm, left=28mm, right=28mm]{geometry}
\usepackage{parskip}
\setlength{\parskip}{8pt}
\setlength{\parindent}{0pt}

% Colors
\usepackage{xcolor}
\definecolor{primary}{HTML}{2563EB}
\definecolor{darkbg}{HTML}{0A0F1E}
\definecolor{darkblue}{HTML}{1E3A5F}
\definecolor{bodytext}{HTML}{374151}
\definecolor{heading}{HTML}{111827}
\definecolor{subtitle}{HTML}{64748B}
\definecolor{lightgray}{HTML}{F8F9FA}
\definecolor{border}{HTML}{E5E7EB}
\definecolor{accent}{HTML}{93C5FD}
\definecolor{lightondark}{HTML}{D1D5DB}
\definecolor{darkred}{HTML}{B91C1C}
\definecolor{darkgreen}{HTML}{15803D}
\definecolor{darkyellow}{HTML}{92400E}
\definecolor{lightblue}{HTML}{F0F4FF}
\definecolor{lightred}{HTML}{FEF2F2}
\definecolor{lightgreen}{HTML}{F0FDF4}

% Headers & Footers
\usepackage{fancyhdr}
\pagestyle{fancy}
\fancyhf{}
\renewcommand{\headrulewidth}{0pt}
\renewcommand{\footrulewidth}{0.4pt}
\fancyfoot[L]{\footnotesize\color{subtitle}\textls[50]{CNC PLANER PRO · STRATEGISCHE EINSCHÄTZUNG \& USP}}
\fancyfoot[R]{\footnotesize\color{subtitle}\thepage}

% Headings
\usepackage{titlesec}
\usepackage{needspace}
\titleformat{\section}{\needspace{6\baselineskip}\fontsize{24}{28}\selectfont\bfseries\color{heading}}{}{0em}{}[\vspace{-2pt}]
\titleformat{\subsection}{\needspace{4\baselineskip}\fontsize{14}{18}\selectfont\bfseries\color{heading}}{}{0em}{}
\titlespacing*{\section}{0pt}{0pt}{4pt}
\titlespacing*{\subsection}{0pt}{16pt}{6pt}

% Tables
\usepackage{tabularx}
\usepackage{booktabs}
\usepackage{colortbl}

% Lists
\usepackage{enumitem}
\setlist[itemize]{leftmargin=1.2em, itemsep=2pt, parsep=0pt, topsep=4pt}

% Links
\usepackage{hyperref}
\hypersetup{colorlinks=true, linkcolor=primary, urlcolor=primary}

% Drawing — nur für Cover und Stat-Cards
\usepackage{tikz}
\usetikzlibrary{calc,positioning}

% Boxes
\usepackage{tcolorbox}
\tcbuselibrary{skins,breakable}

% Misc
\usepackage{microtype}
\usepackage{graphicx}
\usepackage{float}
\usepackage{lastpage}

% Language
\usepackage{polyglossia}
\setdefaultlanguage{german}
\tolerance=2000
\emergencystretch=15pt
\hbadness=2000
\widowpenalty=10000
\clubpenalty=10000

% Custom commands
\newcommand{\ssubtitle}[1]{%
  \par\textcolor{subtitle}{\fontsize{12}{16}\selectfont #1}%
  \par\vspace{2pt}\textcolor{border}{\rule{\linewidth}{0.4pt}}\vspace{12pt}%
}

\newtcolorbox{highlightbox}{
  colback=lightblue, colframe=primary,
  leftrule=3pt, rightrule=0pt, toprule=0pt, bottomrule=0pt,
  arc=0pt, outer arc=4pt,
  boxsep=4pt, left=12pt, right=12pt, top=8pt, bottom=8pt,
  fontupper=\fontsize{11}{15}\selectfont\color{darkblue}
}

\newtcolorbox{darkhighlight}{
  colback=darkbg, colframe=primary,
  leftrule=3pt, rightrule=0pt, toprule=0pt, bottomrule=0pt,
  arc=0pt, outer arc=4pt,
  boxsep=4pt, left=12pt, right=12pt, top=8pt, bottom=8pt,
  fontupper=\fontsize{11}{15}\selectfont\color{white}
}

\newtcolorbox{warnbox}{
  colback=lightred, colframe=darkred,
  leftrule=3pt, rightrule=0pt, toprule=0pt, bottomrule=0pt,
  arc=0pt, outer arc=4pt,
  boxsep=4pt, left=12pt, right=12pt, top=8pt, bottom=8pt,
  fontupper=\fontsize{11}{15}\selectfont\color{darkred}
}

\newtcolorbox{greenbox}{
  colback=lightgreen, colframe=darkgreen,
  leftrule=3pt, rightrule=0pt, toprule=0pt, bottomrule=0pt,
  arc=0pt, outer arc=4pt,
  boxsep=4pt, left=12pt, right=12pt, top=8pt, bottom=8pt,
  fontupper=\fontsize{11}{15}\selectfont\color{darkgreen}
}

\newcommand{\statcard}[2]{%
  \begin{tikzpicture}
    \node[fill=lightgray, rounded corners=4pt, minimum width=4.4cm, minimum height=2.2cm, inner sep=6pt, align=center, text width=4cm] {
      {\fontsize{20}{24}\selectfont\bfseries\color{primary}#1}\\[3pt]
      {\fontsize{8}{10}\selectfont\color{subtitle}\MakeUppercase{#2}}
    };
  \end{tikzpicture}%
}

\newcommand{\statcardred}[2]{%
  \begin{tikzpicture}
    \node[fill=lightred, rounded corners=4pt, minimum width=4.4cm, minimum height=2.2cm, inner sep=6pt, align=center, text width=4cm] {
      {\fontsize{20}{24}\selectfont\bfseries\color{darkred}#1}\\[3pt]
      {\fontsize{8}{10}\selectfont\color{subtitle}\MakeUppercase{#2}}
    };
  \end{tikzpicture}%
}

\newcommand{\statcardgreen}[2]{%
  \begin{tikzpicture}
    \node[fill=lightgreen, rounded corners=4pt, minimum width=4.4cm, minimum height=2.2cm, inner sep=6pt, align=center, text width=4cm] {
      {\fontsize{20}{24}\selectfont\bfseries\color{darkgreen}#1}\\[3pt]
      {\fontsize{8}{10}\selectfont\color{subtitle}\MakeUppercase{#2}}
    };
  \end{tikzpicture}%
}

\color{bodytext}

\begin{document}

% ============================================
% COVER PAGE
% ============================================
\thispagestyle{empty}
\begin{tikzpicture}[remember picture, overlay]
  \fill[darkbg] (current page.north west) rectangle (current page.south east);
  \node[anchor=north west, text width=14cm] at ($(current page.north west)+(2.8cm,-4cm)$) {
    {\fontsize{11}{13}\selectfont\color{accent}\textls[100]{STRATEGISCHE EINSCHÄTZUNG}}\\[18pt]
    {\fontsize{36}{40}\selectfont\bfseries\color{white}CNC Planer Pro\\[4pt]USP \& Positionierung}\\[20pt]
    {\fontsize{14}{18}\selectfont\color{lightondark}Belegt mit echten Kalkulationsdaten\\Validiert gegen MBS b-logic ERP\\Ehrliche Bewertung von Stärken und Grenzen}\\[50pt]
    {\fontsize{11}{14}\selectfont\color{accent}Basierend auf}\\[4pt]
    {\fontsize{13}{16}\selectfont\color{white}6 Positionen realer Angebotsvergleich · MBS-Vorkalkulation Nr.~74256--74261}\\[4pt]
    {\fontsize{11}{14}\selectfont\color{lightondark}Wettbewerb: Spanflug · goCAD · up2parts · b-logic · Excel}
  };
  \node[anchor=south west, text width=14cm] at ($(current page.south west)+(2.8cm,3cm)$) {
    {\fontsize{10}{13}\selectfont\color{subtitle}Florian Ziesche · +1\,347\,740\,1465 · florian@ziesche.co}\\[4pt]
    {\fontsize{9}{12}\selectfont\color{subtitle}Version 2.0 · 6. Februar 2026}
  };
\end{tikzpicture}

\clearpage

% ============================================
% 1. EXECUTIVE SUMMARY
% ============================================
\section{Zusammenfassung}
\ssubtitle{Die Lage in 60 Sekunden}

\begin{darkhighlight}
\textbf{CNC Planer Pro kalkuliert Ø\,9,8\,\% über den Herstellkosten eines etablierten ERP-Systems.} Das ist konservativ genug, um kein Geld zu verlieren — und nah genug, um als Einstiegstool nützlich zu sein.

\vspace{4pt}

Als reines Kalkulationstool hat CNC Planer Pro \textbf{keinen USP} gegenüber Spanflug, goCAD oder up2parts. Aber als \textbf{Consulting-Demo + Wissenstransfer-Infrastruktur} für Kleinstbetriebe ohne ERP gibt es eine echte Lücke.
\end{darkhighlight}

\vspace{8pt}

\begin{center}
\statcard{+9,8\,\%}{Abweichung\\zu MBS (HK-Ebene)}
\hspace{6pt}
\statcardgreen{0 Verluste}{Konservative\\Kalkulation}
\hspace{6pt}
\statcardred{Kein USP}{als reines\\Kalkulationstool}
\end{center}

\vspace{6pt}

\textbf{Empfohlene Strategie:} CNC Planer Pro als \textbf{Demo-Asset für KI-Consulting} nutzen (sofortiger Revenue) und parallel den \textbf{Wissenstransfer-Ansatz} bei 5 Betrieben validieren.


% ============================================
% 2. KALKULATIONS-VALIDIERUNG
% ============================================
\clearpage
\section{Kalkulationsqualität: Die Fakten}
\ssubtitle{Was kann CNC Planer Pro tatsächlich — belegt mit echten Daten}

\subsection{Vergleichsbasis}

Verglichen werden die Herstellkosten (HK) aus der MBS b-logic Vorkalkulation mit den CNC Planer Pro Ergebnissen. HK statt Angebotspreis, weil MBS eine Mischkalkulation anwendet (einzelne Positionen unter HK, andere darüber).

\subsection{3 Demo-Bauteile mit technischen Zeichnungen}

\begin{tabularx}{\textwidth}{l r r r r}
\toprule
\rowcolor{lightgray} \textbf{Bauteil} & \textbf{Stk} & \textbf{MBS HK/Stk} & \textbf{CNC HK/Stk} & \textbf{Abweichung} \\
\midrule
Platte & 29 & EUR\,21,51 & EUR\,22,67 & \textcolor{darkyellow}{+5,4\,\%} \\
\rowcolor{lightgray} Adapterplatte & 10 & EUR\,74,31 & EUR\,79,06 & \textcolor{darkyellow}{+6,4\,\%} \\
Block & 5 & EUR\,66,15 & EUR\,72,11 & \textcolor{darkyellow}{+9,0\,\%} \\
\midrule
\rowcolor{lightgray} \textbf{Ø gewichtet} & & & & \textbf{+5,9\,\%} \\
\bottomrule
\end{tabularx}

\subsection{Alle 6 Positionen}

\begin{tabularx}{\textwidth}{c l r r r r}
\toprule
\rowcolor{lightgray} \textbf{Pos} & \textbf{Bezeichnung} & \textbf{Stk} & \textbf{MBS HK/Stk} & \textbf{CNC HK/Stk} & \textbf{Abw.} \\
\midrule
1 & Platte & 29 & EUR\,21,51 & EUR\,22,67 & +5,4\,\% \\
\rowcolor{lightgray} 2 & Welle & 4 & EUR\,49,03 & EUR\,56,18 & +14,6\,\% \\
3 & Block (Typ 1) & 5 & EUR\,114,95 & EUR\,131,40 & +14,3\,\% \\
\rowcolor{lightgray} 4 & Block (Typ 2) & 5 & EUR\,66,15 & EUR\,72,11 & +9,0\,\% \\
5 & Finger & 20 & EUR\,58,26 & EUR\,70,63 & +21,2\,\% \\
\rowcolor{lightgray} 6 & Adapterplatte & 10 & EUR\,74,31 & EUR\,79,06 & +6,4\,\% \\
\midrule
\multicolumn{2}{l}{\textbf{Ø gewichtet}} & \textbf{73} & \textbf{EUR\,3.634} & \textbf{EUR\,3.990} & \textbf{+9,8\,\%} \\
\bottomrule
\end{tabularx}

\subsection{Abweichungsursachen}

\begin{itemize}
  \item \textbf{60\,\% Maschinenzeiten:} CNC Planer Pro nutzt normative REFA/VDI-Richtwerte. MBS hat empirische Werte aus jahrelanger Fertigung.
  \item \textbf{25\,\% Rüstzeit-Allokation:} CNC Planer Pro rechnet Rüstkosten pro Stück. MBS bucht als Fixbetrag pro Auftrag.
  \item \textbf{15\,\% Materialpreise:} Rahmenverträge vs. kg-Richtpreise.
\end{itemize}

\begin{highlightbox}
\textbf{Entscheidendes Muster:} CNC Planer Pro kalkuliert \textbf{immer} über MBS — nie darunter. Kein einziges Angebot auf CNC-Planer-Basis hätte zu einem Verlust geführt. MBS hingegen macht bei diesem Auftrag EUR\,86 Verlust (Angebotspreis EUR\,5.047 < Herstellkosten EUR\,5.133).
\end{highlightbox}


% ============================================
% 3. WETTBEWERB
% ============================================
\clearpage
\section{Wettbewerb: Ehrliche Einordnung}
\ssubtitle{Was andere können — und wir nicht}

\subsection{Die Landschaft}

\begin{tabularx}{\textwidth}{l X l}
\toprule
\rowcolor{lightgray} \textbf{Anbieter} & \textbf{Kernfunktion} & \textbf{Preis} \\
\midrule
\textbf{Spanflug Make} & STEP-basierte Kalkulation + Angebot + AV & EUR\,333/Mo \\
\rowcolor{lightgray} \textbf{goCAD} & KI-basierte Zeichnungsanalyse (CAD/DXF) & individuell \\
\textbf{up2parts} & Ähnlichkeitssuche + autoCAM & individuell \\
\rowcolor{lightgray} \textbf{b-logic / ERP} & Vollständige Auftragsabwicklung & EUR\,5--50K+ \\
\textbf{Excel / Kopf} & Manuelle Kalkulation & EUR\,0 \\
\midrule
\rowcolor{lightgray} \textbf{CNC Planer Pro} & Zuschlagskalkulation + Nachkalkulation & EUR\,49/Mo \\
\bottomrule
\end{tabularx}

\subsection{Wer was besser kann}

\begin{warnbox}
\textbf{Ehrliche Wahrheit:} In einem Feature-Vergleich verliert CNC Planer Pro gegen jeden einzelnen Wettbewerber. Spanflug hat echte CAD-Analyse, goCAD hat KI-Zeichnungserkennung, up2parts hat Ähnlichkeitssuche, b-logic hat echte Betriebsdaten. Sogar Excel ist flexibler.

\vspace{4pt}

Die Frage ist nicht „Was können wir besser?" sondern „Gibt es eine Zielgruppe, für die die anderen zu viel sind?"
\end{warnbox}

\subsection{Was CNC Planer Pro NICHT kann}

\begin{itemize}
  \item \textbf{Keine CAD-Analyse:} Operationen manuell definieren. Kein STEP-Import.
  \item \textbf{Keine Erfahrungsdatenbank:} Kein Gedächtnis für vergangene Fertigungen.
  \item \textbf{Kein ERP-Ersatz:} Keine Materialwirtschaft, keine Auftragsverfolgung.
  \item \textbf{Bearbeitungszeiten sind Richtwerte:} Basierend auf Formeln, nicht auf realen Maschinenlauf\-zeiten.
  \item \textbf{Keine komplexen Spannlagen:} Mehrseitige Bearbeitung vereinfacht.
\end{itemize}


% ============================================
% 4. DIE LÜCKE — USP
% ============================================
\clearpage
\section{Die Lücke: Wo liegt der USP?}
\ssubtitle{Zwischen Excel und Spanflug klafft ein Vakuum}

\subsection{Das Adoptionsproblem}

Für einen 3--5-Mann-Lohnfertiger:

\begin{tabularx}{\textwidth}{l X l}
\toprule
\rowcolor{lightgray} \textbf{Lösung} & \textbf{Problem für KMU} & \textbf{Kosten} \\
\midrule
Excel / Kopf & Funktioniert, aber nicht skalierbar, nicht übertragbar & EUR\,0 \\
\rowcolor{lightgray} Spanflug Make & Braucht STEP-Dateien. Viele KMU haben nur Zeichnungen (PDF). & EUR\,333/Mo \\
goCAD / up2parts & CAD-Infrastruktur vorausgesetzt. Individuelles Pricing. & EUR\,500+/Mo \\
\rowcolor{lightgray} ERP (b-logic etc.) & Jenseits des Budgets und der IT-Kapazität & EUR\,5K--50K+ \\
\bottomrule
\end{tabularx}

\vspace{6pt}

\begin{darkhighlight}
\textbf{Die Lücke:} Es gibt kein Tool für EUR\,50--100/Monat das \textbf{ohne CAD-Dateien} funktioniert und dem Meister eine Struktur gibt, die sein Erfahrungswissen systematisch erfasst. Genau hier sitzt CNC Planer Pro.
\end{darkhighlight}

\subsection{Drei mögliche USPs — bewertet}

\begin{tabularx}{\textwidth}{r X X}
\toprule
\rowcolor{lightgray} \textbf{\#} & \textbf{Kandidat} & \textbf{Bewertung} \\
\midrule
1 & \textbf{Wissenstransfer-System:} Erfasst das Kalkulationswissen des Meisters (Stundensätze, Korrekturfaktoren, Ist-Zeiten) und macht es für Nachfolger verfügbar. & \textcolor{darkgreen}{\textbf{Stärkster USP.}} Kein Wettbewerber positioniert sich so. Emotional stark (Rente, Fachkräftemangel). \\[4pt]
\rowcolor{lightgray} 2 & \textbf{Einstiegstool ohne Hürden:} Keine STEP-Dateien, kein CAD, keine IT-Abteilung. Abmessungen eingeben, kalkulieren. Sofort einsatzbereit. & \textcolor{darkyellow}{\textbf{Nischen-USP.}} Relevant für die „untechnischsten" Betriebe. Abgrenzung zu Spanflug Free (5 Teile kostenlos). \\[4pt]
3 & \textbf{Consulting-Demo:} Zeigt was KI in einer Woche für einen Betrieb bauen kann. Türöffner für EUR\,150--300/h Consulting-Aufträge. & \textcolor{darkgreen}{\textbf{Höchster sofortiger ROI.}} Nicht das Tool ist der USP — der Mensch der es gebaut hat. \\
\bottomrule
\end{tabularx}

\subsection{Der stärkste USP: Wissenstransfer}

\begin{greenbox}
\textbf{Positionierung:} „Wir sichern das Kalkulationswissen Ihres Meisters — bevor er in Rente geht."

\vspace{4pt}

Das ist kein Feature-Argument. Das ist ein \textbf{Business-Problem im Wert von EUR\,40.000+/Jahr}. Ein falsch kalkulierender Nachfolger verliert bei jeder einzelnen Angebotserstellung Geld oder Aufträge.

\vspace{4pt}

Kein Wettbewerber adressiert dieses Problem explizit.
\end{greenbox}


% ============================================
% 5. MARKTGRÖSSE
% ============================================
\clearpage
\section{Marktgröße: Realistisch}
\ssubtitle{Klein aber ehrlich}

\begin{tabularx}{\textwidth}{X r}
\toprule
\rowcolor{lightgray} \textbf{Segment} & \textbf{Anzahl} \\
\midrule
CNC-Fertiger Deutschland gesamt & 2.468 \\
\rowcolor{lightgray} Davon unter 20 MA (ca. 65\,\%) & ca. 1.600 \\
Davon ohne ERP-Kalkulation (ca. 50\,\%) & ca. 800 \\
\rowcolor{lightgray} Davon nicht bei Spanflug/goCAD (ca. 90\,\%) & ca. 720 \\
\textbf{Adressierbarer Markt} & \textbf{500--700} \\
\bottomrule
\end{tabularx}

\vspace{6pt}

\begin{tabularx}{\textwidth}{X r r}
\toprule
\rowcolor{lightgray} \textbf{Szenario} & \textbf{Kunden} & \textbf{ARR bei EUR\,49/Mo} \\
\midrule
Pessimistisch (2\,\% Penetration Y1) & 10--15 & EUR\,6--9K \\
\rowcolor{lightgray} Realistisch (5\,\% Penetration Y1) & 25--35 & EUR\,15--21K \\
Optimistisch (15\,\% Penetration Y3) & 75--105 & EUR\,44--62K \\
\bottomrule
\end{tabularx}

\begin{warnbox}
\textbf{Klartext:} Selbst im optimistischsten Szenario sind EUR\,50--60K ARR die Obergrenze als SaaS in Deutschland. Das ist \textbf{kein VC-Case}. Es ist ein solides Nebenprodukt oder ein Baustein für etwas Größeres (Consulting-Geschäft).
\end{warnbox}


% ============================================
% 6. POSITIONIERUNGS-STRATEGIE
% ============================================
\section{Positionierungsstrategie}
\ssubtitle{Drei Wege — und welcher sich lohnt}

\subsection{Szenario A: Weiterbauen als Kalkulationstool}

\begin{tabularx}{\textwidth}{l X}
\toprule
\rowcolor{lightgray} Aufwand & STEP-Import, Feature-Erkennung, ML-Zeitermittlung. Monate bis Jahre. \\
Konkurrenz & Spanflug (EUR\,12M+ Funding), goCAD, up2parts. Jahrelanger Vorsprung. \\
\rowcolor{lightgray} Empfehlung & \textcolor{darkred}{\textbf{Nicht empfohlen.}} Kein realistischer Pfad zu Wettbewerbsparität. \\
\bottomrule
\end{tabularx}

\subsection{Szenario B: Pivot zu Wissenstransfer-Plattform}

\begin{tabularx}{\textwidth}{l X}
\toprule
\rowcolor{lightgray} Pitch & „Das System das Ihr Kalkulationswissen überlebt." \\
Zielgruppe & Betriebe mit altersbedingt ausscheidendem Meister/AV. \\
\rowcolor{lightgray} Markt & Klein aber unbesetzt. Emotional stark (Fachkräftemangel). \\
Empfehlung & \textcolor{darkyellow}{\textbf{Validieren.}} 5 Gespräche mit Betrieben wo Meister bald in Rente geht. \\
\bottomrule
\end{tabularx}

\subsection{Szenario C: Demo-Asset + KI-Consulting (empfohlen)}

\begin{tabularx}{\textwidth}{l X}
\toprule
\rowcolor{lightgray} Pitch & „Ich habe in einer Woche eine komplette Kalkulationssoftware gebaut. Was könnte ich für Ihren Betrieb bauen?" \\
Revenue & EUR\,150--300/h Consulting. Sofortiger Cash. \\
\rowcolor{lightgray} Beweis & CNC Planer Pro mit +9,8\,\% Genauigkeit vs.\ ERP. In 7 Tagen gebaut. \\
Empfehlung & \textcolor{darkgreen}{\textbf{Höchster ROI.}} Jede Demo generiert Gesprächsanlass. \\
\bottomrule
\end{tabularx}

\begin{darkhighlight}
\textbf{Empfohlene Kombination: C + B parallel.}

\vspace{4pt}

\textbf{Sofort:} CNC Planer Pro als Consulting-Demo nutzen. Es zeigt Kompetenz, generiert Gespräche und Revenue (EUR\,150--300/h). Plus: KI-Workshops für Mittelstand (EUR\,1.500--2.500/Tag).

\vspace{4pt}

\textbf{Parallel (3--6 Monate):} Wissenstransfer-Ansatz in Pilotgesprächen validieren. Wenn positiv: Pivot. Wenn nicht: Tool als Portfolio-Stück behalten.
\end{darkhighlight}


% ============================================
% 7. FAZIT
% ============================================
\clearpage
\section{Fazit}
\ssubtitle{Die Entscheidung auf einer Seite}

\begin{center}
\statcard{+9,8\,\%}{Abweichung\\HK-Ebene}
\hspace{6pt}
\statcardred{Kein USP}{als reines\\Kalkulationstool}
\hspace{6pt}
\statcardgreen{USP}{Wissenstransfer\\+ Consulting-Demo}
\end{center}

\vspace{12pt}

\begin{tabularx}{\textwidth}{l X}
\toprule
\rowcolor{lightgray} \textbf{Frage} & \textbf{Antwort} \\
\midrule
Ist die Kalkulation brauchbar? & \textcolor{darkgreen}{Ja.} Ø\,+9,8\,\% über HK. Konservativ, keine Verluste. Durch Nachkalkulation korrigierbar auf <5\,\%. \\
\rowcolor{lightgray} Hat CNC Planer Pro einen USP? & \textcolor{darkred}{Nein} als Kalkulationstool. \textcolor{darkgreen}{Ja} als Wissenstransfer-Infrastruktur und Consulting-Demo. \\
Kann es Kunden finden? & Begrenzt. Ca. 500--700 adressierbare Betriebe in DE. Max. EUR\,50--60K ARR. \\
\rowcolor{lightgray} Lohnt sich Weiterbau? & Nur als Consulting-Demo oder nach Validierung des Wissenstransfer-Ansatzes. Nicht als eigenständiges SaaS. \\
Was sollte jetzt passieren? & Demo beim Onkel $\rightarrow$ Feedback $\rightarrow$ 5 weitere Gespräche $\rightarrow$ Entscheidung. \\
\bottomrule
\end{tabularx}

\vspace{16pt}

\begin{darkhighlight}
\textbf{Die wichtigste Erkenntnis:}

\vspace{4pt}

CNC Planer Pro ist kein besseres Spanflug. Und das muss es auch nicht sein.

\vspace{4pt}

Es ist ein \textbf{Proof of Concept} das zeigt: Florian Ziesche kann in einer Woche eine vollständige, industrietaugliche Anwendung bauen — inklusive Zuschlagskalkulation, Fertigungsanweisung, Nachkalkulation und Angebotsvorlage. Validiert gegen ein echtes ERP-System. 9,8\,\% Abweichung bei null Vorerfahrung in CNC-Kalkulation.

\vspace{4pt}

\textbf{Das} ist der USP. Nicht das Tool. Der Mensch der es gebaut hat.
\end{darkhighlight}

\vfill

\begin{center}
\small\textcolor{subtitle}{
Daten: MBS b-logic Vorkalkulation Nr.~74256--74261 · Angebot Nr.~20260072\\
Wettbewerbsdaten: Spanflug.de, goCAD.de, up2parts.com · Marktdaten: listflix.de, Destatis\\
Alle Preise netto in EUR · Stand: Februar 2026
}
\end{center}

\end{document}
