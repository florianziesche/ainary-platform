\documentclass[a4paper,11pt]{article}

% Fonts
\usepackage{fontspec}
\setmainfont{Helvetica Neue}[
  BoldFont=Helvetica Neue Bold,
  ItalicFont=Helvetica Neue Italic
]

% Layout
\usepackage[top=28mm, bottom=34mm, left=28mm, right=28mm]{geometry}
\usepackage{parskip}
\setlength{\parskip}{8pt}
\setlength{\parindent}{0pt}

% Colors
\usepackage{xcolor}
\definecolor{primary}{HTML}{2563EB}
\definecolor{darkbg}{HTML}{0A0F1E}
\definecolor{darkblue}{HTML}{1E3A5F}
\definecolor{bodytext}{HTML}{374151}
\definecolor{heading}{HTML}{111827}
\definecolor{subtitle}{HTML}{64748B}
\definecolor{lightgray}{HTML}{F8F9FA}
\definecolor{border}{HTML}{E5E7EB}
\definecolor{accent}{HTML}{93C5FD}
\definecolor{lightondark}{HTML}{D1D5DB}
\definecolor{darkred}{HTML}{B91C1C}
\definecolor{darkgreen}{HTML}{15803D}
\definecolor{darkyellow}{HTML}{92400E}
\definecolor{lightblue}{HTML}{F0F4FF}
\definecolor{lightred}{HTML}{FEF2F2}
\definecolor{lightgreen}{HTML}{F0FDF4}

% Headers & Footers
\usepackage{fancyhdr}
\pagestyle{fancy}
\fancyhf{}
\renewcommand{\headrulewidth}{0pt}
\renewcommand{\footrulewidth}{0.4pt}
\fancyfoot[L]{\footnotesize\color{subtitle}\textls[50]{CNC PLANER PRO · VERGLEICHSREPORT}}
\fancyfoot[R]{\footnotesize\color{subtitle}\thepage}

% Headings
\usepackage{titlesec}
\usepackage{needspace}
\titleformat{\section}{\needspace{6\baselineskip}\fontsize{24}{28}\selectfont\bfseries\color{heading}}{}{0em}{}[\vspace{-2pt}]
\titleformat{\subsection}{\needspace{4\baselineskip}\fontsize{14}{18}\selectfont\bfseries\color{heading}}{}{0em}{}
\titlespacing*{\section}{0pt}{0pt}{4pt}
\titlespacing*{\subsection}{0pt}{16pt}{6pt}

% Tables
\usepackage{tabularx}
\usepackage{booktabs}
\usepackage{colortbl}
\usepackage{multirow}

% Lists
\usepackage{enumitem}
\setlist[itemize]{leftmargin=1.2em, itemsep=2pt, parsep=0pt, topsep=4pt}

% Links
\usepackage{hyperref}
\hypersetup{colorlinks=true, linkcolor=primary, urlcolor=primary}

% Drawing
\usepackage{tikz}
\usetikzlibrary{calc,positioning,shapes.geometric}

% Boxes
\usepackage{tcolorbox}
\tcbuselibrary{skins,breakable}

% Misc
\usepackage{microtype}
\usepackage{setspace}
\usepackage{graphicx}
\usepackage{float}
\usepackage{multicol}
\usepackage{amssymb}

% Language
\usepackage{polyglossia}
\setdefaultlanguage{german}
\tolerance=2000
\emergencystretch=15pt
\hbadness=2000
\hyphenpenalty=50
\widowpenalty=10000
\clubpenalty=10000

% Custom commands
\newcommand{\ssubtitle}[1]{%
  \par\textcolor{subtitle}{\fontsize{12}{16}\selectfont #1}%
  \par\vspace{2pt}\textcolor{border}{\rule{\linewidth}{0.4pt}}\vspace{12pt}%
}

\newtcolorbox{highlightbox}{
  colback=lightblue, colframe=primary,
  leftrule=3pt, rightrule=0pt, toprule=0pt, bottomrule=0pt,
  arc=0pt, outer arc=4pt,
  boxsep=4pt, left=12pt, right=12pt, top=8pt, bottom=8pt,
  fontupper=\fontsize{11}{15}\selectfont\color{darkblue}
}

\newtcolorbox{darkhighlight}{
  colback=darkbg, colframe=primary,
  leftrule=3pt, rightrule=0pt, toprule=0pt, bottomrule=0pt,
  arc=0pt, outer arc=4pt,
  boxsep=4pt, left=12pt, right=12pt, top=8pt, bottom=8pt,
  fontupper=\fontsize{11}{15}\selectfont\color{white}
}

\newcommand{\statcard}[2]{%
  \begin{tikzpicture}
    \node[fill=lightgray, rounded corners=4pt, minimum width=4.4cm, minimum height=2.2cm, inner sep=6pt, align=center, text width=4cm] {
      {\fontsize{20}{24}\selectfont\bfseries\color{primary}#1}\\[3pt]
      {\fontsize{8}{10}\selectfont\color{subtitle}\MakeUppercase{#2}}
    };
  \end{tikzpicture}%
}

\newcommand{\statcardred}[2]{%
  \begin{tikzpicture}
    \node[fill=lightred, rounded corners=4pt, minimum width=4.4cm, minimum height=2.2cm, inner sep=6pt, align=center, text width=4cm] {
      {\fontsize{20}{24}\selectfont\bfseries\color{darkred}#1}\\[3pt]
      {\fontsize{8}{10}\selectfont\color{subtitle}\MakeUppercase{#2}}
    };
  \end{tikzpicture}%
}

\newcommand{\statcardgreen}[2]{%
  \begin{tikzpicture}
    \node[fill=lightgreen, rounded corners=4pt, minimum width=4.4cm, minimum height=2.2cm, inner sep=6pt, align=center, text width=4cm] {
      {\fontsize{20}{24}\selectfont\bfseries\color{darkgreen}#1}\\[3pt]
      {\fontsize{8}{10}\selectfont\color{subtitle}\MakeUppercase{#2}}
    };
  \end{tikzpicture}%
}

\color{bodytext}

\begin{document}

% ============================================
% COVER PAGE
% ============================================
\thispagestyle{empty}
\begin{tikzpicture}[remember picture, overlay]
  \fill[darkbg] (current page.north west) rectangle (current page.south east);
  \node[anchor=north west, text width=14cm] at ($(current page.north west)+(2.8cm,-4cm)$) {
    {\fontsize{11}{13}\selectfont\color{accent}\textls[100]{VERGLEICHSREPORT}}\\[18pt]
    {\fontsize{36}{40}\selectfont\bfseries\color{white}MBS b-logic ERP\\[4pt]vs.\\[4pt]CNC Planer Pro}\\[20pt]
    {\fontsize{14}{18}\selectfont\color{lightondark}Kalkulations-Vergleich auf Basis realer\\Fertigungsdaten — ehrliche Gegenüberstellung}\\[40pt]
    {\fontsize{11}{14}\selectfont\color{accent}Referenzbauteil}\\[4pt]
    {\fontsize{13}{16}\selectfont\color{white}Platte · 29 Stück · 1.4571 Edelstahl · 440\,×\,50\,×\,20\,mm}\\[4pt]
    {\fontsize{11}{14}\selectfont\color{lightondark}Zeichnungsnr. 2500473.01.11.02.00.001}
  };
  \node[anchor=south west, text width=14cm] at ($(current page.south west)+(2.8cm,3cm)$) {
    {\fontsize{10}{13}\selectfont\color{subtitle}Florian Ziesche · CNC Planer Pro · Februar 2026}\\[4pt]
    {\fontsize{9}{12}\selectfont\color{subtitle}Datengrundlage: MBS Maschinenbau Schlottwitz GmbH \& Co. KG, Angebot mit 6 Kalkulationsblättern (b-logic ERP)}
  };
\end{tikzpicture}

\clearpage

% ============================================
% EXECUTIVE SUMMARY
% ============================================
\section{Zusammenfassung}
\ssubtitle{Ehrlicher Vergleich zweier Kalkulationssysteme}

\begin{darkhighlight}
\textbf{Ergebnis:} CNC Planer Pro kalkuliert mit Standardparametern \textbf{Faktor 4,9× höher} als MBS b-logic ERP. Der Angebotspreis für die Referenz-Platte liegt bei EUR\,128,83 statt EUR\,26,30. Die Hauptursachen sind identifiziert und adressierbar.
\end{darkhighlight}

\vspace{8pt}

\begin{center}
\statcardred{4,9×}{Preisabweichung\\(CNC vs. MBS)}
\hspace{8pt}
\statcardred{59,7 min}{CNC Bearbeitungszeit\\(MBS: ca. 12 min)}
\hspace{8pt}
\statcard{46,7\,\%}{Kumulative Zuschläge\\(MBS: 42,6\,\%)}
\end{center}

\vspace{6pt}

Die drei Haupttreiber der Abweichung:

\begin{enumerate}[leftmargin=2em]
  \item \textbf{Bearbeitungszeit} (ca. 70\,\% der Abweichung) — 10 generische OPs statt 3--4 bauteilspezifische
  \item \textbf{Materialberechnung} (ca. 15\,\%) — Vollvolumen × kg-Preis statt Halbzeug-Einkauf
  \item \textbf{Kumulative Zuschläge} (ca. 15\,\%) — Wirken multiplikativ auf bereits zu hohe Basis
\end{enumerate}

\begin{highlightbox}
\textbf{Fazit:} CNC Planer Pro ist in der aktuellen Version kein Ersatz für ein kalibriertes ERP-System. Sein Wert liegt in der \textbf{Strukturierung} der Kalkulation, der \textbf{Wissenssicherung} und der \textbf{Nachkalkulation} — nicht in der absoluten Preisgenauigkeit.
\end{highlightbox}


% ============================================
% KALKULATIONS-VERGLEICH
% ============================================
\clearpage
\section{Kalkulations-Vergleich}
\ssubtitle{Platte, 29 Stück, 1.4571 Edelstahl — Zeile für Zeile}

\subsection{Materialkosten}

\begin{tabularx}{\textwidth}{X r r}
\toprule
\rowcolor{lightgray} \textbf{Position} & \textbf{MBS b-logic} & \textbf{CNC Planer Pro} \\
\midrule
Materialpreis/Stück & EUR\,5,17 & EUR\,18,30 \\
\rowcolor{lightgray} + MGK (MBS: inkl. / CNC: 5\,\%) & — & EUR\,0,92 \\
\textbf{Material-HK pro Stück} & \textbf{EUR\,5,17} & \textbf{EUR\,19,22} \\
\midrule
\textit{Abweichung} & \multicolumn{2}{r}{\textcolor{darkred}{+EUR\,14,05 (+272\,\%)}} \\
\bottomrule
\end{tabularx}

\vspace{8pt}

\textbf{Ursache:} CNC Planer Pro berechnet das Material als Vollvolumen (440\,×\,50\,×\,20\,mm = 3,52\,kg × EUR\,5,20/kg). MBS kalkuliert mit \textbf{Halbzeug-Einkaufspreisen} — z.\,B. Flachstahl 50\,×\,20\,mm als Stangenmaterial, Serienzuschnitt für 29 Stück.

\begin{highlightbox}
\textbf{Lösung:} Hinterlegen des betrieblichen Einkaufspreises pro Bauteil. Wenn kein Einkaufspreis hinterlegt ist, greift die Volumen-Berechnung als Fallback.
\end{highlightbox}

\subsection{Fertigungskosten}

\begin{tabularx}{\textwidth}{X r r}
\toprule
\rowcolor{lightgray} \textbf{Position} & \textbf{MBS b-logic} & \textbf{CNC Planer Pro} \\
\midrule
Maschinenkosten/Stück & EUR\,5,47 & EUR\,69,62 \\
\rowcolor{lightgray} Rückgerechnete Maschinenzeit & ca. 10 min & 59,7 min \\
Lohnkosten/Stück (Grenzk.) & EUR\,8,05 & (im Stundensatz) \\
\rowcolor{lightgray} Rüstkosten/Stück & (inkl.) & EUR\,1,01 \\
Nebenarbeiten/Stück & (inkl.) & EUR\,4,83 \\
\rowcolor{lightgray} + AV-Zuschlag (MBS: 35\,\% GK / CNC: 12\,\%) & (auf Lohn) & EUR\,9,05 \\
\textbf{Fertigungs-HK pro Stück} & \textbf{EUR\,16,34} & \textbf{EUR\,84,51} \\
\midrule
\textit{Abweichung} & \multicolumn{2}{r}{\textcolor{darkred}{+EUR\,68,17 (+417\,\%)}} \\
\bottomrule
\end{tabularx}

\vspace{8pt}

\textbf{Ursache:} CNC Planer Pro generiert einen Operationsplan mit 10 Arbeitsgängen (inkl. Schlichten h5, Feinbohren H7, Entgraten, Qualitätskontrolle) für eine relativ einfache Platte. MBS kalkuliert bauteilspezifisch mit 3--4 Arbeitsgängen und erreicht ca. 10--13 Minuten Gesamtzeit.

\subsection{Gesamtkalkulation}

\begin{tabularx}{\textwidth}{X r r}
\toprule
\rowcolor{lightgray} \textbf{Position} & \textbf{MBS b-logic} & \textbf{CNC Planer Pro} \\
\midrule
Material-HK & EUR\,5,17 & EUR\,19,22 \\
\rowcolor{lightgray} Fertigungs-HK & EUR\,16,34 & EUR\,84,51 \\
\textbf{Herstellkosten} & \textbf{EUR\,21,51} & \textbf{EUR\,103,73} \\
\midrule
\rowcolor{lightgray} + VwGK (CNC: 10\,\%) & (in Lohn-GK) & EUR\,10,37 \\
+ VtGK (CNC: 5\,\%) & (in Lohn-GK) & EUR\,5,19 \\
\rowcolor{lightgray} \textbf{Selbstkosten} & — & \textbf{EUR\,119,29} \\
+ Gewinn (MBS: 18,2\,\% / CNC: 8\,\%) & EUR\,4,79 & EUR\,9,54 \\
\midrule
\rowcolor{lightgray} \textbf{ANGEBOTSPREIS pro Stück} & \textbf{EUR\,26,30} & \textbf{EUR\,128,83} \\
\textit{Abweichung} & \multicolumn{2}{r}{\textcolor{darkred}{+EUR\,102,53 (+389,8\,\%)}} \\
\bottomrule
\end{tabularx}


% ============================================
% METHODIK-VERGLEICH
% ============================================
\clearpage
\section{Methodik-Vergleich}
\ssubtitle{Zwei unterschiedliche Ansätze, zwei unterschiedliche Stärken}

\subsection{Kalkulationsverfahren}

\begin{tabularx}{\textwidth}{l X X}
\toprule
\rowcolor{lightgray} & \textbf{MBS b-logic} & \textbf{CNC Planer Pro} \\
\midrule
\textbf{Verfahren} & ERP-integrierte Vollkostenrechnung & Differenzierende Zuschlagskalkulation \\
\rowcolor{lightgray} \textbf{Material} & Einkaufspreise aus Warenwirtschaft & kg-Preis × Volumen (Fallback) \\
\textbf{Zeiten} & Erfahrungswerte / Nachkalkulation & REFA-Richtwerte / VDI 3321 \\
\rowcolor{lightgray} \textbf{Zuschläge} & Lohn-GK-Zuschlag ca. 35\,\% & MGK, AV, VwGK, VtGK (separat) \\
\textbf{Marge} & Individuell pro Position & Einheitlicher \%-Zuschlag \\
\rowcolor{lightgray} \textbf{Mischkalkulation} & Ja (+18\,\% bis -24,6\,\%) & Nein (gleiche Marge überall) \\
\textbf{Kalibrierung} & Jahrelange Betriebsdaten & Ab Werk: Branchenrichtwerte \\
\bottomrule
\end{tabularx}

\vspace{8pt}

\textbf{Zentrale Erkenntnis:} MBS nutzt eine Mischkalkulation — einfache Teile (Platte: +18,2\,\% Marge) finanzieren komplexe Teile (Finger: -24,6\,\% Marge). CNC Planer Pro kennt dieses Konzept nicht und rechnet jedes Teil mit gleicher Marge.

\subsection{Stundensätze}

\begin{tabularx}{\textwidth}{X r r}
\toprule
\rowcolor{lightgray} \textbf{Komponente} & \textbf{MBS (rückgerechnet)} & \textbf{CNC Planer Pro} \\
\midrule
CNC-Maschinensatz & ca. EUR\,32/h & EUR\,32/h \\
\rowcolor{lightgray} CNC-Lohnsatz & ca. EUR\,38/h & EUR\,38/h \\
\textbf{Gesamt CNC} & \textbf{ca. EUR\,70/h} & \textbf{EUR\,70/h} \\
\midrule
\textit{Bewertung} & \multicolumn{2}{r}{\textcolor{darkgreen}{Stundensätze stimmen überein}} \\
\bottomrule
\end{tabularx}

\vspace{6pt}

Die Stundensätze wurden aus den MBS-Kalkulationsblättern rückgerechnet und als Standardwerte in CNC Planer Pro hinterlegt. \textbf{Die Sätze selbst sind nicht das Problem — die darauf angewendete Zeit ist es.}


% ============================================
% STÄRKEN / SCHWÄCHEN
% ============================================
\clearpage
\section{Ehrliche Bewertung}
\ssubtitle{Was jedes System gut kann — und was nicht}

\subsection{MBS b-logic ERP}

\textbf{\textcolor{darkgreen}{Stärken:}}
\begin{itemize}
  \item Echte Betriebsdaten (Einkaufspreise, Maschinenzeiten, Nachkalkulation)
  \item Mischkalkulation ermöglicht wettbewerbsfähige Gesamtangebote
  \item Jahrelang kalibriert — Preise stimmen
  \item Vollständige Auftragsabwicklung (Angebot → Auftrag → Fertigung → Rechnung)
  \item Fremdleistungen, Baugruppen, Stücklisten integriert
\end{itemize}

\textbf{\textcolor{darkred}{Schwächen:}}
\begin{itemize}
  \item Hohe Einrichtungs- und Lizenzkosten (EUR\,5.000--50.000+)
  \item Wissen steckt im System UND im Kopf des Bedieners
  \item Einarbeitung dauert Wochen bis Monate
  \item Für Kleinstbetriebe (1--5 MA) oft überdimensioniert
  \item Kalkulationslogik für Anwender nicht transparent (Black Box)
\end{itemize}

\subsection{CNC Planer Pro}

\textbf{\textcolor{darkgreen}{Stärken:}}
\begin{itemize}
  \item Sofort einsatzbereit (Browser, kein Setup)
  \item Transparente Kalkulation — jede Formel sichtbar
  \item Zuschlagskalkulation nach Industriestandard (lehrbuchtauglich)
  \item Nachkalkulationssystem für kontinuierliche Verbesserung
  \item Fertigungsanweisung + Angebot aus einer Quelle
  \item Offline-fähig, keine Datenübertragung
  \item EUR\,49/Monat statt EUR\,5.000+ Einrichtung
\end{itemize}

\textbf{\textcolor{darkred}{Schwächen:}}
\begin{itemize}
  \item \textbf{Bearbeitungszeiten Faktor 5× zu hoch} (generischer OP-Plan)
  \item Keine CAD-Analyse — Geometrie wird nicht erkannt
  \item Materialberechnung ohne Halbzeug-Formate (Vollvolumen)
  \item Keine Mischkalkulation (gleiche Marge für alle Positionen)
  \item Keine ERP-Integration (Aufträge, Rechnung, Lagerhaltung)
  \item Keine Fremdleistungen, Baugruppen, Stücklisten
  \item Operationsplan muss manuell geprüft und korrigiert werden
\end{itemize}

\begin{highlightbox}
\textbf{Kernaussage:} MBS b-logic ist das bessere Kalkulationstool. CNC Planer Pro ist das bessere Einstiegstool. Die beiden Systeme bedienen unterschiedliche Zielgruppen und Reifegrade.
\end{highlightbox}


% ============================================
% MARGEN-ANALYSE MBS
% ============================================
\clearpage
\section{MBS Margen-Analyse}
\ssubtitle{Mischkalkulation über 6 Positionen — aus den Kalkulationsblättern}

\begin{tabularx}{\textwidth}{l l r r r r}
\toprule
\rowcolor{lightgray} \textbf{Pos} & \textbf{Artikel} & \textbf{Stk} & \textbf{Angebot/Stk} & \textbf{HK/Stk} & \textbf{Marge} \\
\midrule
1 & Platte & 29 & EUR\,26,30 & EUR\,21,51 & \textcolor{darkgreen}{+18,2\,\%} \\
\rowcolor{lightgray} 2 & Welle & 4 & EUR\,58,00 & EUR\,49,03 & \textcolor{darkgreen}{+15,5\,\%} \\
3 & Block Typ 1 & 5 & EUR\,105,92 & EUR\,114,95 & \textcolor{darkred}{-7,8\,\%} \\
\rowcolor{lightgray} 4 & Block Typ 2 & 5 & EUR\,64,16 & EUR\,66,15 & \textcolor{darkred}{-3,0\,\%} \\
5 & Finger & 20 & EUR\,43,91 & EUR\,58,26 & \textcolor{darkred}{-24,6\,\%} \\
\rowcolor{lightgray} 6 & Platte Typ 2 & 10 & EUR\,72,89 & EUR\,74,31 & \textcolor{darkred}{-1,9\,\%} \\
\midrule
7 & Montage & 1 & EUR\,1.595,10 & — & — \\
\rowcolor{lightgray} & \textbf{Gesamt} & & \textbf{EUR\,5.047,30} & & \\
\bottomrule
\end{tabularx}

\vspace{8pt}

\textbf{Strategie:} MBS fährt eine bewusste Mischkalkulation. Einfache, schnell zu fertigende Teile (Platte, Welle) werden gewinnbringend kalkuliert. Komplexe Teile (Block, Finger) werden teilweise unter Herstellkosten angeboten. Die Montagepauschale (Pos.\,7) kompensiert die Einzelteil-Verluste.

\begin{darkhighlight}
\textbf{Lektion für CNC Planer Pro:} Ein einheitlicher Gewinnzuschlag (z.\,B. 8\,\%) auf alle Positionen bildet die Realität nicht ab. Erfahrene Kalkulatoren vergeben individuelle Margen pro Position — das setzt Branchenkenntnis und Wettbewerbsverständnis voraus, das kein Tool automatisieren kann.
\end{darkhighlight}


% ============================================
% WAS FEHLT
% ============================================
\section{Was CNC Planer Pro fehlt}
\ssubtitle{Offene Lücken — priorisiert nach Impact}

\begin{tabularx}{\textwidth}{l l X}
\toprule
\rowcolor{lightgray} \textbf{Prio} & \textbf{Feature} & \textbf{Auswirkung wenn fehlt} \\
\midrule
P0 & Halbzeug-Kalkulator & Materialkosten 3× zu hoch \\
\rowcolor{lightgray} P0 & Bauteil-proportionale Zeiten & Bearbeitungszeiten 5× zu hoch \\
P0 & Editierbare OP-Zeiten & Nutzer kann Zeiten nicht korrigieren \\
\midrule
\rowcolor{lightgray} P1 & Einkaufspreis pro Bauteil & Nur kg-Fallback verfügbar \\
P1 & Individuelle Margen pro Position & Keine Mischkalkulation möglich \\
\rowcolor{lightgray} P1 & CAD-Import (STEP) & Geometrie manuell eingeben \\
P1 & Fremdleistungen & Beschichtung, Härten nicht kalkulierbar \\
\midrule
\rowcolor{lightgray} P2 & ERP-Export/Import & Kein Datenaustausch \\
P2 & Stücklisten / Baugruppen & Nur Einzelteile \\
\rowcolor{lightgray} P2 & Ähnlichkeitssuche & Keine historische Referenz \\
\bottomrule
\end{tabularx}


% ============================================
% ZIELGRUPPE
% ============================================
\clearpage
\section{Positionierung}
\ssubtitle{Für wen CNC Planer Pro einen echten Mehrwert liefert}

\begin{darkhighlight}
\textbf{Zielgruppe:} CNC-Lohnfertiger die schneller kalkulieren wollen als im Kopf, aber kein ERP brauchen. Strukturiert kalkulieren statt schätzen. Wissen sichern statt im Kopf behalten.
\end{darkhighlight}

\vspace{8pt}

\subsection{Idealkunde}

\begin{tabularx}{\textwidth}{l X}
\toprule
\rowcolor{lightgray} \textbf{Kriterium} & \textbf{Ausprägung} \\
\midrule
Betriebsgröße & 1--10 Mitarbeiter \\
\rowcolor{lightgray} Kalkulations-Methode heute & Kopf, Taschenrechner, Excel \\
ERP-System & Keins oder nicht für Kalkulation genutzt \\
\rowcolor{lightgray} Fertigungsprofil & Prismatische 3-Achs-Frästeile, Einzelteil/Kleinserie \\
Anfragen/Woche & 5--15 \\
\rowcolor{lightgray} Schmerz & Meister ist Flaschenhals, Wissen nur im Kopf \\
\bottomrule
\end{tabularx}

\subsection{Das Wissenstransfer-Argument}

\begin{center}
\statcardred{19,5 Mio}{Arbeitskräfte verlassen\\den Markt bis 2036}
\hspace{8pt}
\statcardred{72\,\%}{der Führungskräfte:\\Know-how nicht gesichert}
\hspace{8pt}
\statcardred{14\,h/Woche}{verbringen MA mit\\Wissen-Weitergabe}
\end{center}

\vspace{6pt}

\textit{Quellen: IW-Studie / tagesschau (Apr. 2025), Destatis (Aug. 2025), Flip/Workplace Intelligence Studie (Dez. 2025)}

\vspace{8pt}

Der Meister (58) kalkuliert seit 30 Jahren. Er weiß, dass die Platte 12 Minuten braucht, nicht 45. Er weiß, welcher Kunde immer verhandelt. Er weiß, welche Maschine Spiel auf der Y-Achse hat.

\textbf{Nichts davon ist dokumentiert.}

Wenn er in Rente geht, fängt der Nachfolger bei Null an — nicht bei den Maschinen, sondern bei der Kalkulation. CNC Planer Pro speichert die Erfahrungswerte systematisch: Stundensätze, Nachkalkulationsdaten, betriebsspezifische Zuschläge. \textbf{Der Nachfolger bekommt nicht unsere Zahlen, sondern seine.}


% ============================================
% FAZIT
% ============================================
\section{Fazit}
\ssubtitle{Was daraus folgt}

\begin{tabularx}{\textwidth}{l X X}
\toprule
\rowcolor{lightgray} & \textbf{MBS b-logic} & \textbf{CNC Planer Pro} \\
\midrule
\textbf{Besser bei} & Preisgenauigkeit, ERP-Integration, Mischkalkulation, Betriebsdaten & Einstieg, Transparenz, Schnelligkeit, Wissenssicherung, Kosten \\
\rowcolor{lightgray} \textbf{Zielgruppe} & Betriebe mit >10 MA und AV-Abteilung & Betriebe mit 1--10 MA ohne ERP \\
\textbf{Einrichtung} & Wochen -- Monate & Minuten \\
\rowcolor{lightgray} \textbf{Kosten} & EUR\,5.000--50.000+ & EUR\,49/Monat \\
\textbf{Preisgenauigkeit} & Hoch (kalibriert) & Richtwert (±30\,\% ohne Kalibrierung) \\
\bottomrule
\end{tabularx}

\vspace{12pt}

\begin{highlightbox}
\textbf{CNC Planer Pro ist nicht die bessere Kalkulation. Es ist der bessere Einstieg.}

Für Betriebe, die heute gar nicht strukturiert kalkulieren, ist der Schritt von \glqq{}im Kopf\grqq{} zu \glqq{}transparent und nachvollziehbar\grqq{} der größere Hebel als der Schritt von \glqq{}±30\,\%\grqq{} zu \glqq{}±5\,\%\grqq{}.
\end{highlightbox}

\end{document}
