\documentclass[a4paper,11pt]{article}

\usepackage{fontspec}
\setmainfont{Helvetica Neue}[BoldFont=Helvetica Neue Bold, ItalicFont=Helvetica Neue Italic]

\usepackage[top=25mm, bottom=30mm, left=22mm, right=22mm]{geometry}
\usepackage{parskip}
\setlength{\parskip}{7pt}
\setlength{\parindent}{0pt}

\usepackage{xcolor}
\definecolor{primary}{HTML}{2563EB}
\definecolor{darkbg}{HTML}{0A0F1E}
\definecolor{darkblue}{HTML}{1E3A5F}
\definecolor{bodytext}{HTML}{374151}
\definecolor{heading}{HTML}{111827}
\definecolor{subtitle}{HTML}{64748B}
\definecolor{lightgray}{HTML}{F8F9FA}
\definecolor{border}{HTML}{E5E7EB}
\definecolor{accent}{HTML}{93C5FD}
\definecolor{lightondark}{HTML}{D1D5DB}
\definecolor{lightblue}{HTML}{F0F4FF}
\definecolor{lightgreen}{HTML}{F0FDF4}
\definecolor{darkgreen}{HTML}{15803D}
\definecolor{screenborder}{HTML}{D1D5DB}

\usepackage{fancyhdr}
\pagestyle{fancy}
\fancyhf{}
\renewcommand{\headrulewidth}{0pt}
\renewcommand{\footrulewidth}{0.4pt}
\fancyfoot[L]{\footnotesize\color{subtitle}\textls[50]{CNC PLANER PRO $\cdot$ BENUTZERHANDBUCH}}
\fancyfoot[R]{\footnotesize\color{subtitle}\thepage}

\usepackage{titlesec}
\usepackage{needspace}
\titleformat{\section}{\needspace{6\baselineskip}\fontsize{20}{24}\selectfont\bfseries\color{heading}}{}{0em}{}[\vspace{-2pt}]
\titleformat{\subsection}{\needspace{4\baselineskip}\fontsize{12}{16}\selectfont\bfseries\color{heading}}{}{0em}{}
\titlespacing*{\section}{0pt}{0pt}{4pt}
\titlespacing*{\subsection}{0pt}{12pt}{4pt}

\usepackage{tabularx}
\usepackage{booktabs}
\usepackage{colortbl}
\usepackage{enumitem}
\setlist[itemize]{leftmargin=1.2em, itemsep=2pt, parsep=0pt, topsep=3pt}
\setlist[enumerate]{leftmargin=1.8em, itemsep=3pt, parsep=0pt, topsep=3pt}

\usepackage{hyperref}
\hypersetup{colorlinks=true, linkcolor=primary, urlcolor=primary}

\usepackage{tikz}
\usetikzlibrary{calc,positioning}

\usepackage{tcolorbox}
\tcbuselibrary{skins,breakable}

\usepackage{microtype}
\usepackage{graphicx}
\usepackage{float}

\usepackage{polyglossia}
\setdefaultlanguage{german}
\tolerance=2000
\emergencystretch=15pt
\hbadness=2000
\widowpenalty=10000
\clubpenalty=10000

\newcommand{\ssubtitle}[1]{%
  \par\textcolor{subtitle}{\fontsize{11}{14}\selectfont #1}%
  \par\vspace{2pt}\textcolor{border}{\rule{\linewidth}{0.4pt}}\vspace{8pt}%
}

\newtcolorbox{highlightbox}{
  colback=lightblue, colframe=primary,
  leftrule=3pt, rightrule=0pt, toprule=0pt, bottomrule=0pt,
  arc=0pt, outer arc=4pt,
  boxsep=3pt, left=10pt, right=10pt, top=6pt, bottom=6pt,
  fontupper=\fontsize{10}{13}\selectfont\color{darkblue}
}

\newtcolorbox{tipbox}{
  colback=lightgreen, colframe=darkgreen,
  leftrule=3pt, rightrule=0pt, toprule=0pt, bottomrule=0pt,
  arc=0pt, outer arc=4pt,
  boxsep=3pt, left=10pt, right=10pt, top=6pt, bottom=6pt,
  fontupper=\fontsize{10}{13}\selectfont\color{bodytext}
}

% Screenshot command: full width, border, caption below
\newcommand{\screenshot}[2]{%
  \begin{figure}[H]
  \centering
  \setlength{\fboxsep}{0pt}%
  \setlength{\fboxrule}{0.5pt}%
  \fcolorbox{screenborder}{white}{\includegraphics[width=0.98\textwidth]{#1}}%
  \par\vspace{3pt}%
  {\small\color{subtitle}#2}%
  \end{figure}%
}

\color{bodytext}

\begin{document}

% ============================================
% COVER
% ============================================
\thispagestyle{empty}
\begin{tikzpicture}[remember picture, overlay]
  \fill[darkbg] (current page.north west) rectangle (current page.south east);
  \node[anchor=north west, text width=14cm] at ($(current page.north west)+(2.2cm,-3.5cm)$) {
    {\fontsize{11}{13}\selectfont\color{accent}\textls[100]{BENUTZERHANDBUCH}}\\[16pt]
    {\fontsize{38}{44}\selectfont\bfseries\color{white}CNC Planer Pro}\\[8pt]
    {\fontsize{16}{20}\selectfont\color{lightondark}Version 0.18.0-beta}\\[28pt]
    {\fontsize{13}{17}\selectfont\color{lightondark}Anleitung zur Kalkulation, Angebotserstellung\\und Fertigungsplanung}\\[50pt]
    {\fontsize{11}{14}\selectfont\color{accent}Funktionsumfang}\\[6pt]
    {\fontsize{12}{15}\selectfont\color{white}Zuschlagskalkulation nach REFA\\
    Fertigungsanweisung mit Operationsfolge\\
    Angebotsvorschlag mit Kundenverwaltung\\
    Nachkalkulation und Soll-Ist-Vergleich\\
    Deckungsbeitragsrechnung}
  };
  \node[anchor=south west, text width=14cm] at ($(current page.south west)+(2.2cm,2.5cm)$) {
    {\fontsize{10}{13}\selectfont\color{subtitle}Florian Ziesche $\cdot$ florian@ziesche.co $\cdot$ +1\,347\,740\,1465}\\[4pt]
    {\fontsize{9}{12}\selectfont\color{subtitle}Stand: 6. Februar 2026}
  };
\end{tikzpicture}

\clearpage
\tableofcontents
\clearpage

% ============================================
% 1. ÜBERSICHT
% ============================================
\section{Übersicht}
\ssubtitle{Was ist CNC Planer Pro?}

CNC Planer Pro ist ein webbasiertes Kalkulationstool für CNC-Fertiger. Es berechnet Angebotspreise nach dem Prinzip der \textbf{Zuschlagskalkulation} (REFA-Standard) und erstellt Fertigungsanweisungen sowie Angebotsvorschläge.

\begin{tabularx}{\textwidth}{l X}
\toprule
\rowcolor{lightgray} \textbf{Bereich} & \textbf{Funktion} \\
\midrule
Kalkulation & Zuschlagskalkulation (MEK $\rightarrow$ HK $\rightarrow$ SK $\rightarrow$ AP) \\
\rowcolor{lightgray} Fertigung & Operationsfolge mit Schnittdaten und Werkzeugen \\
Angebot & Angebotsvorschlag mit Positionstabelle \\
\rowcolor{lightgray} Nachkalkulation & Soll-Ist-Vergleich und Erkenntnisse \\
Deckungsbeitrag & DB-Rechnung pro Stück und Auftrag \\
\rowcolor{lightgray} Einstellungen & Stundensätze, Zuschläge, Firmendaten \\
\bottomrule
\end{tabularx}

\begin{highlightbox}
\textbf{Wichtig:} CNC Planer Pro liefert Richtwerte für die schnelle Angebotserstellung -- kein ERP-Ersatz. Eine Nachkalkulation nach Fertigung wird empfohlen.
\end{highlightbox}


% ============================================
% 2. STARTSEITE
% ============================================
\clearpage
\section{Bauteil auswählen}
\ssubtitle{Startseite -- Bauteilkarten und Navigation}

\screenshot{screenshots/01-startseite.png}{Startseite mit Bauteil-Übersicht, Upload-Bereich und Werkstück-Konfiguration}

Beim Öffnen zeigt CNC Planer Pro die \textbf{Bauteilkarten} (oben) und die \textbf{Werkstück-Konfiguration} (unten). Klicken Sie auf ein Bauteil, um es zu laden.

\textbf{Links:} Die Navigation gliedert sich in Vorkalkulation, Fertigung, Ausgabe, Auswertung und Konfiguration.


% ============================================
% 3. KONFIGURATION
% ============================================
\clearpage
\section{Werkstück konfigurieren}
\ssubtitle{Maße, Werkstoff, Stückzahl und Aufspannungen}

\screenshot{screenshots/02-config.png}{Werkstück-Konfiguration mit Aufspannungs-Tabelle und Berechnen-Button}

Nach der Bauteilauswahl prüfen Sie die Parameter:

\begin{itemize}
  \item \textbf{Werkstoff:} S235JR, AlMg3, 1.4571, 1.4301, 42CrMo4, C45
  \item \textbf{Rohmaße:} Länge $\times$ Breite $\times$ Höhe in mm
  \item \textbf{Stückzahl:} Beeinflusst die Rüstkostenverteilung pro Stück
  \item \textbf{Aufspannungen:} Editierbare Tabelle -- Spannmittel und Rüstzeit pro Aufspannung
\end{itemize}

Klicken Sie auf \textbf{``Berechnen $\rightarrow$''} um die Kalkulation zu starten.

\begin{tipbox}
\textbf{Tipp:} Je höher die Stückzahl, desto geringer die Rüstkosten pro Teil. Testen Sie verschiedene Losgrößen um den Einfluss zu sehen.
\end{tipbox}


% ============================================
% 4. KALKULATIONSERGEBNIS
% ============================================
\clearpage
\section{Kalkulationsergebnis}
\ssubtitle{Zuschlagskalkulation mit aufklappbaren Details}

\screenshot{screenshots/03-kalkulation.png}{Ergebnisseite mit Angebotspreis, Konfidenz und aufklappbarer Kostengliederung}

Das Ergebnis zeigt:

\begin{enumerate}
  \item \textbf{Angebotspreis pro Stück} mit Konfidenz-Einschätzung
  \item \textbf{Kostengliederung:} Material, Fertigung, Rüsten, Werkzeug, Gemeinkosten, Gewinn
  \item \textbf{Aufklappbare Details:} Klick auf jede Zeile zeigt Formeln und Zwischenwerte
  \item \textbf{Deckungsbeitragsrechnung:} DB I und DB II pro Stück und Auftrag
\end{enumerate}

\begin{highlightbox}
\textbf{Transparenz:} Jede Kalkulations-Zeile zeigt die vollständige Berechnungsformel. Keine Black Box -- alles nachvollziehbar.
\end{highlightbox}


% ============================================
% 5. FERTIGUNGSANWEISUNG
% ============================================
\clearpage
\section{Fertigungsanweisung}
\ssubtitle{Operationsfolge mit Schnittdaten}

\screenshot{screenshots/04-fertigung.png}{Fertigungsanweisung mit Kopfdaten, Werkstück-Infos und Operationsfolge}

Die Fertigungsanweisung enthält:

\begin{itemize}
  \item \textbf{Kopfdaten:} Bauteilname, Zeichnungsnummer, Freigabedatum, Version
  \item \textbf{Werkstück-Infos:} Werkstoff, Rohmaße, Gewicht, Maschine, Steuerung
  \item \textbf{Operationsfolge} (OP10--OP100): Jede Operation mit Werkzeug, Schnittdaten (n, vf, ap) und geschätzter Zeit
  \item \textbf{Prüfmerkmale:} Editierbar, CSV-Export möglich
\end{itemize}

\begin{tipbox}
\textbf{Hinweis:} Die Fertigungsanweisung ist ein Vorschlag -- keine verbindliche Fertigungsplanung. Schnittdaten basieren auf VDI 3321 Richtwerten.
\end{tipbox}


% ============================================
% 6. ANGEBOT
% ============================================
\clearpage
\section{Angebotsvorschlag}
\ssubtitle{Positionstabelle mit Kundendaten}

\screenshot{screenshots/05-angebot.png}{Angebotsvorschlag mit Absender, Empfänger, Positionstabelle und Bedingungen}

Der Angebotsvorschlag enthält:

\begin{itemize}
  \item \textbf{Absender/Empfänger} -- aus Einstellungen und Kundeneingabe
  \item \textbf{Positionstabelle} -- Artikelnr., Bezeichnung, Zeichnungsnr., Menge, Preise
  \item \textbf{Summe} -- Netto, MwSt 19\,\%, Brutto
  \item \textbf{Gültigkeit} -- automatisch (Standard: 4 Wochen, freibleibend)
  \item \textbf{Bedingungen} -- Preiskalkulations-Hinweis, Genauigkeitshinweis
  \item \textbf{Export:} ``Drucken/PDF'' und ``Kopieren'' Buttons
\end{itemize}


% ============================================
% 7. NACHKALKULATION
% ============================================
\clearpage
\section{Nachkalkulation}
\ssubtitle{Soll-Ist-Vergleich und kontinuierliche Verbesserung}

\screenshot{screenshots/06-nachkalkulation.png}{Nachkalkulations-Historie mit Soll-Ist-Abweichung und Ergebnis-Status}

Vier Unter-Tabs:

\begin{enumerate}
  \item \textbf{Historie} -- Alle Nachkalkulationen mit Datum, Erfasser, Kalk./Ist-Zeit, Delta, Ergebnis
  \item \textbf{Soll-Ist-Vergleich} -- Detaillierter Vergleich einer Position
  \item \textbf{Erkenntnisse} -- Freitext für Lessons Learned
  \item \textbf{Dashboard} -- Genauigkeits-Übersicht über die Zeit
\end{enumerate}

\begin{highlightbox}
\textbf{Warum wichtig:} Jede Nachkalkulation macht die Vorkalkulation genauer. Nach 50 Teilen konvergiert die Abweichung erfahrungsgemäß auf $<$5\,\%.
\end{highlightbox}


% ============================================
% 8. EINSTELLUNGEN
% ============================================
\clearpage
\section{Einstellungen}
\ssubtitle{Firmendaten, Stundensätze und Angebots-Parameter}

\screenshot{screenshots/07-einstellungen.png}{Einstellungen: Firmendaten, Angebotsbedingungen, Logo-Upload, Darstellung}

Konfigurierbare Parameter:

\begin{tabularx}{\textwidth}{l X}
\toprule
\rowcolor{lightgray} \textbf{Bereich} & \textbf{Felder} \\
\midrule
Firmendaten & Name, Ansprechpartner, Adresse, Telefon, E-Mail, Steuernr., IBAN \\
\rowcolor{lightgray} Angebot & Gültigkeit (Tage), Standard-Lieferzeit, Zahlungsziel \\
Logo & PNG/JPG Upload (max.~200$\times$100\,px) \\
\rowcolor{lightgray} Darstellung & Sprache (Deutsch), Währung (EUR) \\
\bottomrule
\end{tabularx}

\begin{tipbox}
\textbf{Preise \& Sätze} (separater Tab): Stundensätze (Lohn + Maschine), Zuschläge (MGK, AV, VwGK, VtGK, Gewinn) und Verschnitt. Änderungen wirken sofort auf alle Kalkulationen.
\end{tipbox}


% ============================================
% KALKULATIONSLOGIK
% ============================================
\clearpage
\section{Kalkulationslogik}
\ssubtitle{5-stufige Zuschlagskalkulation nach REFA}

\begin{tabularx}{\textwidth}{c X l}
\toprule
\rowcolor{lightgray} \textbf{St.} & \textbf{Beschreibung} & \textbf{Formel} \\
\midrule
1 & \textbf{Materialeinzelkosten (MEK)} & Gewicht $\times$ EUR/kg $\times$ Verschn. \\
& + Materialgemeinkosten (MGK) & MEK $\times$ MGK-Satz \\
\rowcolor{lightgray} & \textbf{= Materialkosten} & \\
\midrule
2 & \textbf{Fertigungseinzelkosten (FEK)} & Zeit $\times$ Stundensatz \\
& + Rüstkosten & Rüstzeit $\times$ Rate $\div$ Stk \\
& + Nebenzeiten & Sägen, Entgraten, Prüfung \\
\rowcolor{lightgray} & + Arbeitsvorbereitung (AV) & FEK $\times$ AV-Satz \\
& \textbf{= Fertigungskosten} & \\
\midrule
\rowcolor{lightgray} 3 & \textbf{HERSTELLKOSTEN (HK)} & Material + Fertigung \\
\midrule
4 & + Verwaltungs-GK (VwGK) & HK $\times$ VwGK-Satz \\
\rowcolor{lightgray} & + Vertriebs-GK (VtGK) & HK $\times$ VtGK-Satz \\
& \textbf{= SELBSTKOSTEN (SK)} & \\
\midrule
\rowcolor{lightgray} 5 & + Gewinnzuschlag & SK $\times$ Gewinn-Satz \\
& \textbf{= ANGEBOTSPREIS (AP)} & \\
\bottomrule
\end{tabularx}

\subsection{Bekannte Einschränkungen}

\begin{enumerate}
  \item \textbf{Materialkosten:} Berechnung aus Volumen $\times$ Dichte $\times$ kg-Preis (keine Halb\-zeug-Op\-ti\-mie\-rung)
  \item \textbf{Bearbeitungszeiten:} Referenzwerte mit Volumenskalierung, nicht geometriebasiert
  \item \textbf{Keine CAD-Integration:} Abmessungen werden manuell eingegeben
  \item \textbf{Werkzeugkosten:} Im Maschinenstundensatz enthalten
  \item \textbf{Oberflächenbehandlung:} Nicht berücksichtigt (Härten, Beschichten)
  \item \textbf{Sonderspannung:} Vorrichtungskosten nicht separat erfasst
  \item \textbf{Transportkosten:} Nicht inkludiert
\end{enumerate}

\begin{highlightbox}
\textbf{Empfehlung:} CNC Planer Pro liefert zuverlässige Richtwerte für die Erstkal\-ku\-la\-tion. Nutzen Sie die Nachkalkulations-Funktion um die Genauigkeit über die Zeit zu verbessern.
\end{highlightbox}

\vfill

\begin{center}
\small\textcolor{subtitle}{
CNC Planer Pro v0.18.0-beta $\cdot$ Benutzerhandbuch\\
Florian Ziesche $\cdot$ florian@ziesche.co $\cdot$ +1\,347\,740\,1465\\
Stand: 6. Februar 2026
}
\end{center}

\end{document}
