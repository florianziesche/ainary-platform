\documentclass[a4paper,11pt]{article}

% Fonts
\usepackage{fontspec}
\setmainfont{Helvetica Neue}[
  BoldFont=Helvetica Neue Bold,
  ItalicFont=Helvetica Neue Italic
]

% Layout
\usepackage[top=28mm, bottom=34mm, left=28mm, right=28mm]{geometry}
\usepackage{parskip}
\setlength{\parskip}{8pt}
\setlength{\parindent}{0pt}

% Colors
\usepackage{xcolor}
\definecolor{primary}{HTML}{2563EB}
\definecolor{darkbg}{HTML}{0A0F1E}
\definecolor{darkblue}{HTML}{1E3A5F}
\definecolor{bodytext}{HTML}{374151}
\definecolor{heading}{HTML}{111827}
\definecolor{subtitle}{HTML}{64748B}
\definecolor{lightgray}{HTML}{F8F9FA}
\definecolor{border}{HTML}{E5E7EB}
\definecolor{accent}{HTML}{93C5FD}
\definecolor{lightondark}{HTML}{D1D5DB}
\definecolor{darkred}{HTML}{B91C1C}
\definecolor{darkgreen}{HTML}{15803D}
\definecolor{darkyellow}{HTML}{92400E}
\definecolor{lightblue}{HTML}{F0F4FF}
\definecolor{lightred}{HTML}{FEF2F2}
\definecolor{lightgreen}{HTML}{F0FDF4}

% Headers & Footers
\usepackage{fancyhdr}
\pagestyle{fancy}
\fancyhf{}
\renewcommand{\headrulewidth}{0pt}
\renewcommand{\footrulewidth}{0.4pt}
\fancyfoot[L]{\footnotesize\color{subtitle}\textls[50]{CNC PLANER PRO · STRATEGISCHE ANALYSE}}
\fancyfoot[R]{\footnotesize\color{subtitle}\thepage}

% Headings
\usepackage{titlesec}
\usepackage{needspace}
\titleformat{\section}{\needspace{6\baselineskip}\fontsize{24}{28}\selectfont\bfseries\color{heading}}{}{0em}{}[\vspace{-2pt}]
\titleformat{\subsection}{\needspace{4\baselineskip}\fontsize{14}{18}\selectfont\bfseries\color{heading}}{}{0em}{}
\titlespacing*{\section}{0pt}{0pt}{4pt}
\titlespacing*{\subsection}{0pt}{16pt}{6pt}

% Tables
\usepackage{tabularx}
\usepackage{booktabs}
\usepackage{colortbl}
\usepackage{multirow}

% Lists
\usepackage{enumitem}
\setlist[itemize]{leftmargin=1.2em, itemsep=2pt, parsep=0pt, topsep=4pt}

% Links
\usepackage{hyperref}
\hypersetup{colorlinks=true, linkcolor=primary, urlcolor=primary}

% Drawing
\usepackage{tikz}
\usetikzlibrary{calc,positioning,shapes.geometric}

% Boxes
\usepackage{tcolorbox}
\tcbuselibrary{skins,breakable}

% Misc
\usepackage{microtype}
\usepackage{setspace}
\usepackage{graphicx}
\usepackage{float}
\usepackage{multicol}
\usepackage{amssymb}

% Language
\usepackage{polyglossia}
\setdefaultlanguage{german}
\tolerance=2000
\emergencystretch=15pt
\hbadness=2000
\hyphenpenalty=50
\widowpenalty=10000
\clubpenalty=10000

% Custom commands
\newcommand{\ssubtitle}[1]{%
  \par\textcolor{subtitle}{\fontsize{12}{16}\selectfont #1}%
  \par\vspace{2pt}\textcolor{border}{\rule{\linewidth}{0.4pt}}\vspace{12pt}%
}

\newtcolorbox{highlightbox}{
  colback=lightblue, colframe=primary,
  leftrule=3pt, rightrule=0pt, toprule=0pt, bottomrule=0pt,
  arc=0pt, outer arc=4pt,
  boxsep=4pt, left=12pt, right=12pt, top=8pt, bottom=8pt,
  fontupper=\fontsize{11}{15}\selectfont\color{darkblue}
}

\newtcolorbox{darkhighlight}{
  colback=darkbg, colframe=primary,
  leftrule=3pt, rightrule=0pt, toprule=0pt, bottomrule=0pt,
  arc=0pt, outer arc=4pt,
  boxsep=4pt, left=12pt, right=12pt, top=8pt, bottom=8pt,
  fontupper=\fontsize{11}{15}\selectfont\color{white}
}

\newcommand{\statcard}[2]{%
  \begin{tikzpicture}
    \node[fill=lightgray, rounded corners=4pt, minimum width=4.4cm, minimum height=2.2cm, inner sep=6pt, align=center, text width=4cm] {
      {\fontsize{20}{24}\selectfont\bfseries\color{primary}#1}\\[3pt]
      {\fontsize{8}{10}\selectfont\color{subtitle}\MakeUppercase{#2}}
    };
  \end{tikzpicture}%
}

\newcommand{\statcardred}[2]{%
  \begin{tikzpicture}
    \node[fill=lightred, rounded corners=4pt, minimum width=4.4cm, minimum height=2.2cm, inner sep=6pt, align=center, text width=4cm] {
      {\fontsize{20}{24}\selectfont\bfseries\color{darkred}#1}\\[3pt]
      {\fontsize{8}{10}\selectfont\color{subtitle}\MakeUppercase{#2}}
    };
  \end{tikzpicture}%
}

\newcommand{\statcardgreen}[2]{%
  \begin{tikzpicture}
    \node[fill=lightgreen, rounded corners=4pt, minimum width=4.4cm, minimum height=2.2cm, inner sep=6pt, align=center, text width=4cm] {
      {\fontsize{20}{24}\selectfont\bfseries\color{darkgreen}#1}\\[3pt]
      {\fontsize{8}{10}\selectfont\color{subtitle}\MakeUppercase{#2}}
    };
  \end{tikzpicture}%
}

\color{bodytext}

\begin{document}

% ============================================
% COVER PAGE
% ============================================
\thispagestyle{empty}
\begin{tikzpicture}[remember picture, overlay]
  \fill[darkbg] (current page.north west) rectangle (current page.south east);
  \node[anchor=north west, text width=14cm] at ($(current page.north west)+(2.8cm,-4cm)$) {
    {\fontsize{11}{13}\selectfont\color{accent}\textls[100]{STRATEGISCHE ANALYSE}}\\[18pt]
    {\fontsize{36}{40}\selectfont\bfseries\color{white}CNC Planer Pro\\[4pt]Produkt-Markt-Fit}\\[20pt]
    {\fontsize{14}{18}\selectfont\color{lightondark}Hat das Produkt einen USP?\\Kann es Kunden finden?\\Und wenn ja — welche?}\\[50pt]
    {\fontsize{11}{14}\selectfont\color{accent}Wettbewerbsanalyse}\\[4pt]
    {\fontsize{13}{16}\selectfont\color{white}Spanflug Make · goCAD · up2parts · b-logic ERP · Excel}\\[4pt]
    {\fontsize{11}{14}\selectfont\color{lightondark}Marktgröße, Positionierung, Go-to-Market}
  };
  \node[anchor=south west, text width=14cm] at ($(current page.south west)+(2.8cm,3cm)$) {
    {\fontsize{10}{13}\selectfont\color{subtitle}Florian Ziesche · Februar 2026}\\[4pt]
    {\fontsize{9}{12}\selectfont\color{subtitle}Research: Spanflug.de, goCAD.de, up2parts.com, listflix.de, Destatis, IW-Studie, Flip/Workplace Intelligence}
  };
\end{tikzpicture}

\clearpage

% ============================================
% EXECUTIVE SUMMARY
% ============================================
\section{Die ehrliche Antwort}
\ssubtitle{Hat CNC Planer Pro einen USP? Kann es Kunden finden?}

\begin{darkhighlight}
\textbf{Kurzfassung:} Als Kalkulationstool hat CNC Planer Pro \textbf{keinen USP} gegenüber Spanflug, goCAD oder up2parts. Diese analysieren echte CAD-Dateien, haben Millionen Datenpunkte und jahrelangen Vorsprung.

\vspace{4pt}

Als \textbf{Wissenstransfer- und Strukturierungstool} für Kleinstbetriebe ohne ERP gibt es eine Lücke im Markt — aber sie ist schmal und die Zahlungsbereitschaft unklar.
\end{darkhighlight}

\vspace{8pt}

\begin{center}
\statcard{2.468}{CNC-Fertiger\\in Deutschland}
\hspace{8pt}
\statcard{ca. 1.600}{davon unter\\20 Mitarbeiter}
\hspace{8pt}
\statcardred{EUR\,333/Mo}{Spanflug Make\\(Jahresabo)}
\end{center}


% ============================================
% WETTBEWERB
% ============================================
\clearpage
\section{Wettbewerbslandschaft}
\ssubtitle{Was existiert — und was es kann}

\subsection{Die etablierten Player}

\begin{tabularx}{\textwidth}{l X X r}
\toprule
\rowcolor{lightgray} \textbf{Anbieter} & \textbf{Kernfunktion} & \textbf{Technologie} & \textbf{Preis} \\
\midrule
\textbf{Spanflug Make} & CAD-basierte Kalkulation + Angebot + AV & STEP-Analyse, Mio. Datenpunkte & EUR\,333/Mo \\
\rowcolor{lightgray} \textbf{goCAD} & CAD/DXF-Analyse + Kalkulation & KI-gestützte Zeichnungsauswertung & individuell \\
\textbf{up2parts} & Arbeitsplan + Kalkulation + autoCAM & Geometrische Ähnlichkeitssuche & individuell \\
\rowcolor{lightgray} \textbf{b-logic/ERP} & Vollständige Auftragsabwicklung & Betriebsdaten + Nachkalkulation & EUR\,5K--50K+ \\
\textbf{Excel/Kopf} & Manuelle Kalkulation & Erfahrungswissen des Meisters & EUR\,0 \\
\midrule
\rowcolor{lightgray} \textbf{CNC Planer Pro} & Zuschlagskalkulation + Nachkalk. & Abmessungen + Formeln & EUR\,49/Mo \\
\bottomrule
\end{tabularx}

\subsection{Was jeder besser kann als wir}

\begin{tabularx}{\textwidth}{l X}
\toprule
\rowcolor{lightgray} \textbf{Anbieter} & \textbf{Überlegenheit gegenüber CNC Planer Pro} \\
\midrule
\textbf{Spanflug} & Echte CAD-Analyse (STEP), tagesaktuelle Materialpreise, Arbeitsplan aus Geometrie, Millionen kalibrierte Bauteile, Halbzeugbeschaffung integriert \\
\rowcolor{lightgray} \textbf{goCAD} & CAD/DXF-Analyse, KI-Auswertung von Zeichnungen, Webshop-Integration, Kunststoff-Kalkulation \\
\textbf{up2parts} & Geometrische Ähnlichkeitssuche (historische Teile wiederverwenden), autoCAM-Integration, ERP-Anbindung \\
\rowcolor{lightgray} \textbf{b-logic} & Echte Betriebsdaten, Mischkalkulation, Auftragssteuerung, Fremdleistungen, Baugruppen \\
\textbf{Excel} & Kostenlos, bereits vorhanden, anpassbar, keine Lernkurve \\
\bottomrule
\end{tabularx}

\begin{highlightbox}
\textbf{Brutale Wahrheit:} In einem Feature-Vergleich verliert CNC Planer Pro gegen jeden einzelnen Wettbewerber. Die Frage ist nicht \glqq{}Was können wir besser?\grqq{} sondern \glqq{}Gibt es eine Zielgruppe, für die die anderen zu viel sind?\grqq{}
\end{highlightbox}


% ============================================
% MARKTLÜCKE
% ============================================
\clearpage
\section{Die Lücke}
\ssubtitle{Zwischen Excel und Spanflug klafft ein Vakuum}

\subsection{Das Adoptionstrichter-Problem}

\vspace{4pt}

\begin{tikzpicture}[scale=1]
  % Funnel
  \fill[lightgray] (0,0) -- (12,0) -- (10.5,1.5) -- (1.5,1.5) -- cycle;
  \fill[lightblue] (1.5,1.5) -- (10.5,1.5) -- (9.5,3) -- (2.5,3) -- cycle;
  \fill[primary!15] (2.5,3) -- (9.5,3) -- (8.8,4.5) -- (3.2,4.5) -- cycle;
  \fill[primary!30] (3.2,4.5) -- (8.8,4.5) -- (8.3,6) -- (3.7,6) -- cycle;
  
  % Labels
  \node[anchor=west] at (12.2,0.75) {\footnotesize\color{bodytext}\textbf{Excel / Kopf} — ca. 1.000+ Betriebe};
  \node[anchor=west] at (10.7,2.25) {\footnotesize\color{bodytext}\textbf{???} — Hier ist die Lücke};
  \node[anchor=west] at (9.7,3.75) {\footnotesize\color{bodytext}\textbf{Spanflug/goCAD} — EUR\,300+/Mo};
  \node[anchor=west] at (9,5.25) {\footnotesize\color{bodytext}\textbf{ERP} — EUR\,5K--50K+};
  
  % Arrow
  \draw[->, very thick, darkred] (6,0.75) -- (6,2.25);
  \node[anchor=west] at (6.3,1.5) {\footnotesize\color{darkred}\textbf{Dieser Sprung ist zu groß}};
\end{tikzpicture}

\vspace{8pt}

Die Realität für einen 3-Mann-Betrieb:

\begin{itemize}
  \item \textbf{Excel/Kopf:} Funktioniert. Kostet nichts. Aber nicht skalierbar und nicht übertragbar.
  \item \textbf{Spanflug Make:} EUR\,333/Monat. Erfordert STEP-Dateien. Viele Kleinstbetriebe arbeiten nur mit Zeichnungen (PDF).
  \item \textbf{goCAD/up2parts:} Individuelles Pricing. CAD-Infrastruktur vorausgesetzt.
  \item \textbf{ERP:} Jenseits des Budgets und der IT-Kapazität.
\end{itemize}

\begin{darkhighlight}
\textbf{Die Lücke:} Es gibt kein Tool für EUR\,50--100/Monat das \textbf{ohne CAD-Dateien} funktioniert und dem Meister eine Struktur gibt, die sein Erfahrungswissen systematisch erfasst. Genau hier könnte CNC Planer Pro sitzen.
\end{darkhighlight}

\subsection{Marktgröße der Lücke}

\begin{tabularx}{\textwidth}{X r}
\toprule
\rowcolor{lightgray} \textbf{Segment} & \textbf{Geschätzte Anzahl} \\
\midrule
CNC-Fertiger Deutschland gesamt & 2.468 \\
\rowcolor{lightgray} Davon unter 20 MA (ca. 65\,\%) & ca. 1.600 \\
Davon ohne ERP-Kalkulation (ca. 50\,\%) & ca. 800 \\
\rowcolor{lightgray} Davon nicht bei Spanflug/goCAD (ca. 90\,\%) & ca. 720 \\
\textbf{Adressierbarer Markt (TAM)} & \textbf{ca. 500--700 Betriebe} \\
\midrule
\rowcolor{lightgray} Bei 5\,\% Penetration (realistisch Y1) & 25--35 Kunden \\
Bei EUR\,49/Mo & \textbf{EUR\,15K--21K ARR} \\
\rowcolor{lightgray} Bei 15\,\% Penetration (optimistisch Y3) & 75--105 Kunden \\
Bei EUR\,49/Mo & \textbf{EUR\,44K--62K ARR} \\
\bottomrule
\end{tabularx}

\begin{highlightbox}
\textbf{Realistisch:} Der Markt ist \textbf{klein}. Selbst bei optimistischer Penetration sind EUR\,50--60K ARR die Obergrenze in Deutschland. Das ist kein VC-Case. Es ist ein solides Nebenprodukt oder ein Baustein für etwas Größeres.
\end{highlightbox}


% ============================================
% USP-ANALYSE
% ============================================
\clearpage
\section{USP-Analyse}
\ssubtitle{Was könnte ein echtes Alleinstellungsmerkmal werden?}

\subsection{Was KEIN USP ist}

\begin{tabularx}{\textwidth}{l X l}
\toprule
\rowcolor{lightgray} \textbf{Behauptung} & \textbf{Warum kein USP} & \textbf{Wer es besser kann} \\
\midrule
\glqq{}Schnelle Kalkulation\grqq{} & Spanflug: Sekunden mit CAD & Spanflug \\
\rowcolor{lightgray} \glqq{}Automatische Zeitermittlung\grqq{} & Unsere Zeiten sind Faktor 5× daneben & Alle \\
\glqq{}Transparente Formeln\grqq{} & Excel ist auch transparent & Excel \\
\rowcolor{lightgray} \glqq{}Professionelles Angebot\grqq{} & Spanflug, goCAD haben das auch & Alle \\
\glqq{}Günstiger Preis\grqq{} & Spanflug Free: 5 Teile/Mo kostenlos & Spanflug \\
\bottomrule
\end{tabularx}

\subsection{Was ein USP SEIN KÖNNTE}

\begin{tabularx}{\textwidth}{r X X}
\toprule
\rowcolor{lightgray} \textbf{\#} & \textbf{Potenzieller USP} & \textbf{Bewertung} \\
\midrule
1 & \textbf{Wissenstransfer-System:} Erfasst Erfahrungswissen des Meisters (Stundensätze, Ist-Zeiten, Korrekturfaktoren) und macht es für den Nachfolger verfügbar. & \textcolor{darkgreen}{Einzigartig.} Kein Wettbewerber positioniert sich so. Emotional starkes Argument (Rente, Fachkräftemangel). \\
\rowcolor{lightgray} 2 & \textbf{Nachkalkulation für KMU ohne ERP:} Strukturierter Soll-Ist-Vergleich, der ohne ERP-Integration funktioniert. & \textcolor{darkyellow}{Interessant.} up2parts hat Ähnlichkeitssuche, aber keine standalone Nachkalkulation. \\
3 & \textbf{Kein CAD erforderlich:} Funktioniert mit Abmessungen aus der Zeichnung. Für Betriebe die keine STEP-Dateien haben. & \textcolor{darkyellow}{Nischen-USP.} Vorteil für die \glqq{}untechnischsten\grqq{} Betriebe. Begrenzte Zielgruppe. \\
\rowcolor{lightgray} 4 & \textbf{Lernende Kalibrierung:} Je mehr Nachkalkulationsdaten, desto genauer werden die Vorkalkulationen. Konvergenz über Zeit. & \textcolor{darkgreen}{Starker Moat} — aber nur mit echtem ML/Feedback-Loop, nicht mit manueller Eingabe. \\
5 & \textbf{Offline + Datenschutz:} Keine Cloud, keine Datenübertragung. Alles lokal. & \textcolor{darkyellow}{Nischen-USP.} Relevant für sicherheitsbewusste Betriebe, aber kein Kaufgrund allein. \\
\bottomrule
\end{tabularx}

\begin{darkhighlight}
\textbf{Der stärkste USP-Kandidat:} \glqq{}Wir sichern das Kalkulationswissen Ihres Meisters — bevor er in Rente geht.\grqq{}

\vspace{4pt}

Das ist kein Feature. Das ist ein Business-Problem im Wert von EUR\,40.000+/Jahr (ein falsch kalkulierender Nachfolger pro Maschine). Und kein Wettbewerber adressiert es.
\end{darkhighlight}


% ============================================
% GO-TO-MARKET
% ============================================
\clearpage
\section{Go-to-Market}
\ssubtitle{Wie findet das Produkt Kunden — realistisch?}

\subsection{Kanäle und ihre Erfolgswahrscheinlichkeit}

\begin{tabularx}{\textwidth}{l X l l}
\toprule
\rowcolor{lightgray} \textbf{Kanal} & \textbf{Taktik} & \textbf{Kosten} & \textbf{Erfolg} \\
\midrule
\textbf{Direktansprache} & Emails an Lohnfertiger aus listflix.de/IndustryArena & Gering & \textcolor{darkyellow}{Mittel} \\
\rowcolor{lightgray} \textbf{Fachforen} & IndustryArena, Zerspanungsbude, CNC-Arena & Null & \textcolor{darkyellow}{Mittel} \\
\textbf{Messen} & Intec, AMB, Metav — Stand oder Besuch & Hoch & \textcolor{darkgreen}{Hoch} \\
\rowcolor{lightgray} \textbf{Handwerkskammern} & Workshops, Digitalisierungsberatung & Gering & \textcolor{darkyellow}{Mittel} \\
\textbf{SEO/Content} & Blog: \glqq{}Maschinenstundensatz berechnen\grqq{} etc. & Zeit & \textcolor{darkgreen}{Langfristig} \\
\rowcolor{lightgray} \textbf{Empfehlung} & Zufriedene Pilotkunden & Null & \textcolor{darkgreen}{Höchste} \\
\textbf{Cold Calls} & Direkt anrufen, Demo anbieten & Zeit & \textcolor{darkyellow}{Mittel} \\
\bottomrule
\end{tabularx}

\subsection{Empfohlene Reihenfolge}

\begin{enumerate}[leftmargin=2em]
  \item \textbf{Validate first (jetzt):} 5 Demos bei echten Betrieben. Nicht verkaufen — Feedback sammeln. Frage: \glqq{}Würden Sie EUR\,49/Monat dafür zahlen? Warum (nicht)?\grqq{}
  \item \textbf{Pilot (Monat 1--2):} 3--5 kostenlose Pilotinstallationen. Bedingung: Nachkalkulationsdaten teilen. Damit: Zeiten kalibrieren, Product verbessern.
  \item \textbf{Content (parallel):} SEO-Artikel zu \glqq{}Maschinenstundensatz berechnen\grqq{}, \glqq{}Zuschlagskalkulation Beispiel\grqq{}, \glqq{}Nachkalkulation CNC\grqq{} — diese Keywords haben Suchvolumen und null Software-Konkurrenz.
  \item \textbf{Paid Pilots (Monat 3+):} Erste zahlende Kunden. EUR\,49/Mo. Nur wenn Feedback von Phase 1+2 positiv.
  \item \textbf{Messen (Monat 6+):} Wenn Produkt validiert. AMB 2026, Intec 2027.
\end{enumerate}


% ============================================
% DREI SZENARIEN
% ============================================
\clearpage
\section{Drei Szenarien}
\ssubtitle{Weiterbauen, pivoten oder parken?}

\subsection{Szenario A: Weiterbauen als Kalkulationstool}

\begin{tabularx}{\textwidth}{l X}
\toprule
\rowcolor{lightgray} \textbf{Aufwand} & STEP-Import, Feature-Erkennung, ML-Zeitermittlung, Halbzeug-Kalkulator. Monate bis Jahre Entwicklung. \\
\textbf{Markt} & Direkte Konkurrenz zu Spanflug (EUR\,12M+ Funding), goCAD, up2parts. \\
\rowcolor{lightgray} \textbf{Risiko} & Hoch. Wettbewerber haben jahrelangen Vorsprung und Datenvorteil. \\
\textbf{Chance} & Wenn Preispunkt EUR\,49 und gute Qualität: Einstiegssegment. \\
\rowcolor{lightgray} \textbf{Empfehlung} & \textcolor{darkred}{Nicht empfohlen als Hauptstrategie.} \\
\bottomrule
\end{tabularx}

\subsection{Szenario B: Pivot zu Wissenstransfer-Plattform}

\begin{tabularx}{\textwidth}{l X}
\toprule
\rowcolor{lightgray} \textbf{Positionierung} & \glqq{}Das System das Ihr Kalkulationswissen überlebt.\grqq{} \\
\textbf{Kernfunktionen} & Nachkalkulation, Erfahrungswerte-Datenbank, Stundensatz-Kalibrierung, Wissens-Export für Nachfolger. \\
\rowcolor{lightgray} \textbf{Zielgruppe} & Betriebe mit altersbedingt ausscheidendem Kalkulationsexperten (Meister, AV). \\
\textbf{Markt} & Kleiner aber unbesetzt. Emotionaler Trigger (Fachkräftemangel). \\
\rowcolor{lightgray} \textbf{Risiko} & Zahlungsbereitschaft unklar. \glqq{}Wissensmanagement\grqq{} klingt abstrakt. \\
\textbf{Chance} & First Mover in einer echten Lücke. Fördermittel möglich (Digitalisierung KMU). \\
\rowcolor{lightgray} \textbf{Empfehlung} & \textcolor{darkyellow}{Validieren. 5 Gespräche mit Meistern die bald in Rente gehen.} \\
\bottomrule
\end{tabularx}

\subsection{Szenario C: Demo-Asset + Consulting-Hebel}

\begin{tabularx}{\textwidth}{l X}
\toprule
\rowcolor{lightgray} \textbf{Positionierung} & CNC Planer Pro als \textbf{Demo-Projekt} für AI-Consulting-Geschäft. \\
\textbf{Pitch} & \glqq{}Ich habe in einer Woche eine voll funktionsfähige Vorkalkulations-App gebaut. Was könnte ich für Ihren Betrieb in einem Monat bauen?\grqq{} \\
\rowcolor{lightgray} \textbf{Zielgruppe} & Mittelständische Fertiger die Digitalisierung wollen aber nicht wissen wo anfangen. \\
\textbf{Markt} & Größer als SaaS-Markt. EUR\,150--300/h Consulting. \\
\rowcolor{lightgray} \textbf{Risiko} & Kein recurring Revenue aus dem Tool selbst. \\
\textbf{Chance} & Sofortiger Cash. Zeigt Kompetenz. Jedes Consulting-Projekt kann zu einem SaaS-Kunden werden. \\
\rowcolor{lightgray} \textbf{Empfehlung} & \textcolor{darkgreen}{Höchster ROI kurzfristig. Nutze die Demo beim Onkel als Proof of Concept.} \\
\bottomrule
\end{tabularx}

\begin{highlightbox}
\textbf{Empfohlene Strategie: C + B parallel.}

\vspace{4pt}

\textbf{Kurzfristig (jetzt):} CNC Planer Pro als Consulting-Demo nutzen. Zeigt was du kannst. Generiert Gespräche und Revenue (EUR\,150--300/h).

\vspace{4pt}

\textbf{Mittelfristig (3--6 Monate):} Wenn Pilotgespräche zeigen dass Wissenstransfer ein Kaufargument ist → Pivot zu Szenario B. Wenn nicht → Tool als Portfolio-Stück behalten, nicht weiter investieren.
\end{highlightbox}


% ============================================
% FAZIT
% ============================================
\clearpage
\section{Fazit}
\ssubtitle{Die eine Folie für die Entscheidung}

\begin{center}
\statcardred{Kein USP}{als reines\\Kalkulationstool}
\hspace{8pt}
\statcardgreen{Lücke}{Wissenstransfer\\für KMU ohne ERP}
\hspace{8pt}
\statcard{EUR\,50--60K}{Max. ARR in DE\\(optimistisch)}
\end{center}

\vspace{12pt}

\begin{tabularx}{\textwidth}{l X}
\toprule
\rowcolor{lightgray} \textbf{Frage} & \textbf{Antwort} \\
\midrule
Hat CNC Planer Pro einen USP? & \textcolor{darkred}{Nein} — als Kalkulationstool. \textcolor{darkgreen}{Potenziell ja} — als Wissenstransfer-System. \\
\rowcolor{lightgray} Kann es Kunden finden? & Ja, aber wenige. Ca. 500--700 Betriebe in DE. Davon erreichbar: 50--100. \\
Lohnt sich Weiterbau? & Nur als Consulting-Demo (sofortiger ROI) oder nach Validierung des Wissenstransfer-Ansatzes. \\
\rowcolor{lightgray} Was sollte JETZT passieren? & Demo beim Onkel → Feedback → 5 weitere Gespräche → Entscheidung. \\
\bottomrule
\end{tabularx}

\vspace{16pt}

\begin{darkhighlight}
\textbf{Die wichtigste Erkenntnis:}

\vspace{4pt}

CNC Planer Pro ist kein besseres Spanflug. Und das muss es auch nicht sein.

\vspace{4pt}

Es ist ein \textbf{Proof of Concept} das zeigt: Florian Ziesche kann in einer Woche eine vollständige, industrietaugliche Anwendung bauen — inklusive Zuschlagskalkulation, Fertigungsanweisung, NC-Code, Nachkalkulation und professionellem Angebots-Generator.

\vspace{4pt}

\textbf{Das} ist der USP. Nicht das Tool. Der Mensch der es gebaut hat.
\end{darkhighlight}

\end{document}
